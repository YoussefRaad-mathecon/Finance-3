\documentclass[12pt]{article}
\usepackage[utf8]{inputenc}
\usepackage[margin=1in]{geometry}
\usepackage[titletoc,title]{appendix}
\usepackage{tabularx}
\usepackage{fancyhdr}
\usepackage{array}
\usepackage{booktabs}
\usepackage{hyperref}
\usepackage{xcolor}
\usepackage{listings}
\usepackage{amsmath,amssymb,amsfonts,mathrsfs,latexsym}
\usepackage{cleveref}
\numberwithin{equation}{section}
\usepackage{graphicx} 
\usepackage{enumerate}
\usepackage{caption}
\usepackage{mathtools}
\usepackage{mdframed}
\usepackage{amsthm}
\DeclareUnicodeCharacter{2212}{-}
\usepackage{changepage}
\usepackage{mdframed}
\usepackage{enumitem}
\usepackage[utf8]{inputenc}
\usepackage[parfill]{parskip}
\usepackage[english]{babel}
\usepackage[nottoc,numbib]{tocbibind}
\usepackage{lastpage} 
\usepackage{float}
\usepackage[bb=boondox]{mathalfa}
\usepackage{graphicx}
\usepackage[round]{natbib}
\usepackage{amsmath,amsfonts,amssymb,mathtools}

\usepackage{graphicx,float}
\renewcommand{\thetable}{\textbf{\arabic{table}}}
\usepackage{pgffor}
\usepackage[ruled,vlined]{algorithm2e}
\usepackage{algorithmic}
\setlength{\parindent}{20pt}

\def\s#1{\mathscr{#1}}
\def\c#1{\mathcal{#1}}
\def\f#1{\mathfrak{#1}}
\foreach \x in {A,...,Z}{
  \expandafter\xdef\csname \x\x\endcsname{\noexpand\mathbb{\x}}
}
\newcommand{\lp}{\left(}
\newcommand{\rp}{\right)}

\newcommand{\lc}{\left\{}
\newcommand{\rc}{\right\}}

\pagestyle{fancy}
\begin{document}
\begin{titlepage}
    \begin{center}
        \vspace*{1cm}
        
        
        \Huge
        \textbf{Hand-In \#3}
        
        \vspace{0.5cm}
        \LARGE
        Continuous Time Finance 2 (Finkont2)
        
        \vspace{0.5cm}
        
        Youssef Raad (zfw568)
    
        
        \vspace{0.5cm}
         April  14, 2024
        
    \end{center}
    \vspace{2.5cm}
    \begin{center}
        \includegraphics[scale=0.50]{ku_segl.png}
    \end{center}
    \vspace{5 cm}
    \begin{center}
        \textit{Do not go gentle into that good night}
    \end{center}
\end{titlepage}
\captionsetup[figure]{labelfont=bf} \captionsetup[table]{labelfont=bf}
\fancyhead[L]{Youssef Raad (zfw568) \\ Continuous Time Finance 2
\\ Hand-In \#3}
\fancyhead[R]{ }
\setlength{\headheight}{40pt}
\newpage
\tableofcontents
\cfoot{\thepage\ / \pageref{LastPage}}


\newpage
\section{Characteristic functions in the Heston model}
The single digit referencing to equation (x) is for equations in the HandIn \#3
description. If equations are refered within this document we will use a double
digit referencing (x.y) where x is the section number and y the equation number. 
\subsection{Changing variables and guessing the form of solution}
$\textcolor{blue}{\star}$ Let $\Psi_j:=\Psi_{x(T)}^j(x,v,t;u)$, $j=1,2$ and $\tau=T-t$. Let the
reparametrization of $\Psi_j$ through $\tau=T-t$ be given by
$\Psi_{x(T)}^j(x,v,\tau;u)$. The transformations of termins in (4) is then given
by:
\begin{align*}
    \frac{\partial }{\partial t}\Psi_{x(T)}^j(x,v,t;u)&=-\frac{\partial }{\partial \tau}\Psi_{x(T)}^j(x,v,\tau;u),\\
    \frac{\partial }{\partial x}\Psi_{x(T)}^j(x,v,t;u)&=\frac{\partial }{\partial x}\Psi_{x(T)}^j(x,v,\tau;u),\\
    \frac{\partial^2 }{\partial x^2}\Psi_{x(T)}^j(x,v,t;u)&=\frac{\partial ^2}{\partial x^2}\Psi_{x(T)}^j(x,v,\tau;u),\\
    \frac{\partial }{\partial v}\Psi_{x(T)}^j(x,v,t;u)&=\frac{\partial }{\partial v}\Psi_{x(T)}^j(x,v,\tau;u),\\
    \frac{\partial^2 }{\partial v^2}\Psi_{x(T)}^j(x,v,t;u)&=\frac{\partial^2 }{\partial v^2}\Psi_{x(T)}^j(x,v,\tau;u),\\
    \frac{\partial^2 }{\partial v \partial x}\Psi_{x(T)}^j(x,v,t;u)&=\frac{\partial^2 }{\partial v \partial x}\Psi_{x(T)}^j(x,v,\tau;u).
\end{align*}
As the only term that differs from (4) through the reparametrization is
$\frac{\partial }{\partial t}\Psi_j=-\frac{\partial }{\partial
\tau}\Psi_{x(T)}^j(x,v,\tau;u)$ we achieve (6) from substitution of the
different term into (4):
\begin{align*}
    0&=-\frac{\partial \Psi_j}{\partial \tau}+(r+u_jv)\frac{\partial \Psi_j}{\partial x}+
    \frac{1}{2}v\frac{\partial^2 \Psi_j}{\partial x^2}+\rho\sigma v \frac{\partial^2 \Psi_j}{\partial v \partial x}+
    \frac{1}{2}\sigma^2v \frac{\partial^2 \Psi_j}{\partial v^2}+(a-b_j v) \frac{\partial \Psi_j}{\partial v},
\end{align*}
as desired.

The terminal condition follows simply from observering that at the boundary $t=T$
and thus $\tau=T-t=T-T=0$. Hence, through the reparametrization using $\tau$ we
achieve (7):
\begin{align*}
    e^{iux}=\Psi_{x(T)}^j(x,v,\tau;u) \quad \forall (x, v) \in \mathbb{R} \times (0, +\infty),
\end{align*}
as desired.
\newpage
$\textcolor{blue}{\star}$ (8) is given by:
\begin{align*}
    \Psi_{x(T)}^j(x,v,t;u)=\exp \lc C_j (\tau; u) + D_j (\tau; u)v + i  u x \rc.
\end{align*}
We insert the ansatz (the guessed solution) (8) into (6) and (7) using the shorthands
$C_j (\tau; u):=C_j$ and $D_j (\tau; u)=D_j$ yielding the PDE:
\begin{align*}
    0&=-\frac{\partial \Psi_j}{\partial \tau}+(r+u_jv)\frac{\partial \Psi_j}{\partial x}+
    \frac{1}{2}v\frac{\partial^2 \Psi_j}{\partial x^2}+\rho\sigma v \frac{\partial^2 \Psi_j}{\partial v \partial x}+
    \frac{1}{2}\sigma^2v \frac{\partial^2 \Psi_j}{\partial v^2}+(a-b_j v) \frac{\partial \Psi_j}{\partial v}\\
    &\overset{\dagger}{=}-\left( \frac{\partial  C_j}{\partial \tau}+ \frac{\partial D_j}{\partial \tau}v\right)\Psi_j +(r+u_jv)(iu) \Psi_j\\&\quad+ \frac{1}{2}v(iu)^2\Psi_j+\rho\sigma v(D_j)(iu)\Psi_j+
    \frac{1}{2}\sigma^2v (D_j)^2\Psi_j+(a-b_j v)( D_j)\Psi_j\\
    &\overset{\dagger\dagger}{=} \Psi_j \lc \underbrace{ -\frac{\partial C_j}{\partial \tau} +aD_j+rui}_{(*)} \rc + \Psi_j v \lc\underbrace{ -\frac{\partial D_j}{\partial \tau}+u_jui-\frac{1}{2}u^2+\rho\sigma u i D_j+\frac{1}{2}\sigma^2 D_j^2 - b_jD_j}_{(**)} \rc,
\end{align*}
where $\dagger$ follows from differentiation and
$\dagger \dagger$ from the fact that $i^2= (\sqrt{-1})^2=-1$ and thus
$(iu)^2=(-1 )\cdot u^2=-u^2$.
\subsection{Transforming the PDE to a system of ODEs and solving it}
$\textcolor{blue}{\star}$ The PDE that we derived in Subsection 1.1 is satisfied if and only if both
ordinary differential equations $(*)$ and $(**)$ equal $0$ for all $x,v)\in (0,\infty)$. Consequently, the
two ordinary differential equations from $(*)$ and $(**)$ can be solved
seperately. The system of two ordinary differential equations is then found by dropping
$\Psi_j$-terms, rewriting and thus given by $(*)$:
\begin{align*}
    0&=-\frac{\partial C_j}{\partial \tau}+aD_j+rui\\&\iff \\ \frac{\partial C_j}{\partial \tau}&=aD_j+rui,
\end{align*}
and $(**)$:
\begin{align*}
    0&=-\frac{\partial D_j}{\partial \tau}+u_jui-\frac{1}{2}u^2+\rho\sigma u iD_j+\frac{1}{2}\sigma^2D_j^2-b_jD_j\\ &\iff \\ \frac{\partial D_j}{\partial \tau}&=u_jui-\frac{1}{2}u^2+\rho\sigma u iD_j+\frac{1}{2}\sigma^2D_j^2-b_jD_j.
\end{align*}
At $\tau=0$, the initial conditions are
zero-initial conditions, i.e such that $D_j(0;u)=0$ and $C_j(0;u) =0$. This is
because when the expiry is reached ($\tau = 0$), the value of $x_T = \log S_T$ is
established, eliminating the expectation in the equality:
\begin{align*}
    \Psi_j&=\EE^{\PP_j} \left [ e^{iux}  \mid x(t)=x,v(t)=v \right]\\
    &=\exp \lc C_j (\tau; u) + D_j (\tau; u) \cdot v + i  u x \rc \quad j=1,2.
\end{align*}
As a result, the second equality has to simplifiy to $e^{iu x}$ as the
expectation vanishes at $\tau=0$. This leads to
the conclusion that the initial conditions at expiry for $D_j (0,\phi) = 0$
and $C_j (0,\phi) = 0$. The equation denoted by (**) outlines a Riccati equation
in $D_j$ that, importantly, does not depend on $C_j$, while the
equation marked by (*) represents an ordinary differential equation for $C_j$.
This ordinary differential equation can be readily solved through direct
integration after determining $D_j$.

$\textcolor{blue}{\star}$ We now proceed to solving the Riccati equation as of
the beforementioned reasons, i.e that it contains terms involving only the
functions $D_j$ and not $C_j$. We have that:
\begin{align*}
    \frac{\partial D_j}{\partial \tau}&=u_jui-\frac{1}{2}u^2+\rho\sigma u iD_j+\frac{1}{2}\sigma^2D_j^2-b_jD_j\\
    &=u_jui-\frac{1}{2}u^2-(b_j-\rho\sigma u i)D_j+\frac{\sigma^2}{2}D_j^2\\
    &=\alpha-\beta D_j+\frac{1}{2}\sigma^2D_j^2,
\end{align*}
with:
\begin{align*}
    \alpha = u_j ui - \frac{1}{2}u^2 \quad \text{and} \quad \beta=b_j-\rho \sigma u i.
\end{align*}
The solution given in (9) can then be written exactly as:
\begin{align*}
    D_j(\tau;u)=-\frac{\frac{\partial E_j(\tau;u)}{\partial \tau}}{\frac{1}{2}\sigma^2E_j(\tau;u)}.
\end{align*}
If we now differentiate the RHS of (9) wrt. $\tau$ we achieve:
\begin{align*}
        \frac{\partial D_j}{\partial \tau}&=-\frac{\frac{\partial^2 E_j}{\partial\tau^2 }E_j\frac{1}{2}\sigma^2- \left ( \frac{\partial E_j}{\partial \tau}\right )^2 \frac{1}{2}\sigma^2}{\left ( \frac{1}{2}\sigma^2 E_j\right )^2}\\
        &=\frac{\left ( \frac{\partial E_j}{\partial \tau}\right )^2-\frac{\partial^2 E_j}{\partial\tau^2 }E_j}{\frac{1}{2}\sigma^2 E_j^2}.
\end{align*}
Substituting (9) and its just found derivatives into the Ricatti equation
yields:
\begin{align*}
    \frac{\partial D_j}{\partial \tau}=\alpha-\beta D_j + \frac{1}{2}\sigma^2D_j^2\\
    &\iff\\
    \frac{\left ( \frac{\partial E_j}{\partial \tau}\right )^2-\frac{\partial^2 E_j}{\partial\tau^2 }E_j}{\frac{1}{2}\sigma^2 E_j^2}&=\alpha-\beta\frac{-\frac{\partial E_jD}{\partial \tau}}{\frac{1}{2}\sigma^2 E_j}+\frac{1}{2}\sigma^2 \left ( \frac{-\frac{\partial E_j}{\partial \tau}}{\frac{1}{2}\sigma^2 E_j}\right )^2\\
    &\iff\\
    \frac{\left ( \frac{\partial E_j}{\partial \tau}\right )^2-\frac{\partial^2 E_j}{\partial\tau^2 }E_j}{\frac{1}{2}\sigma^2 E_j^2}&=\alpha+\beta\frac{\frac{\partial E_jD}{\partial \tau}}{\frac{1}{2}\sigma^2 E_j}+ \frac{\left ( \frac{\partial E_j}{\partial \tau}\right ) ^2}{\frac{1}{2}\sigma^2 E_j}\\
    &\iff\\
    \left ( \frac{\partial E_j}{\partial \tau} \right )^2 - \frac{\partial^2 E_j}{\partial \tau^2} &= \alpha \frac{1}{2} \sigma^2 E_j^2 + \beta \frac{ \frac{\partial E_j}{\partial \tau} }{\frac{1}{2}\sigma^2E_j} \frac{1}{2}\sigma^2 E_j^2+ \frac{\left ( \frac{\partial E_j}{\partial \tau}\right ) ^2}{\frac{1}{2}\sigma^2 E_j}\frac{1}{2}\sigma^2 E_j^2\\
    &\iff\\
    \left ( \frac{\partial E_j}{\partial \tau} \right )^2 - \frac{\partial^2 E_j}{\partial \tau^2} &= \alpha \frac{1}{2} \sigma^2 E_j^2 + \beta \frac{\partial E_j}{\partial \tau}E_j+\left ( \frac{\partial E_j }{\partial \tau} \right )^2\\
    &\iff\\
    - \frac{\partial^2 E_j}{\partial \tau^2}E_j &= \alpha \frac{1}{2} \sigma^2 E_j^2 + \beta \frac{\partial E_j}{\partial \tau}E_j\\
    &\iff\\
    - \frac{\partial^2 E_j}{\partial \tau^2} &= \alpha \frac{1}{2} \sigma^2 E_j + \beta \frac{\partial E_j}{\partial \tau}E_j\\
    &\iff\\
    \frac{\partial^2 E_j}{\partial \tau^2}+\alpha \frac{1}{2} \sigma^2 E_j + \beta \frac{\partial E_j}{\partial \tau}E_j&=0\\
    &\overset{\dagger}{\iff}\\
    \frac{\partial^2 E_j}{\partial \tau^2} - (\rho u \sigma i - b_j) \frac{\partial E_j}{\partial \tau} + \frac{\sigma^2}{2} \left( -\frac{1}{2}u^2 + u_j ui \right) E_j &= 0,
\end{align*}
where $\dagger$ follows from $\alpha = u_j ui - \frac{1}{2}u^2$ and
$\beta=b_j-\rho \sigma u i$.

$\textcolor{blue}{\star}$  We showed the initial condition for $\tau=0$ was
specified as
$D_j(0;u)=0$. We exploit this to show that the respective conditions for $E_j$ are
given by:
\begin{align*}
    D_j(0;u)=-\frac{\left. \frac{\partial E_j(\tau;u)}{\partial \tau}\right|_{\tau=0}}{\frac{1}{2}\sigma^2E_j(0;u)}&=0\\
    &\iff\\
    \left. \frac{\partial E_j(\tau;u)}{\partial \tau}\right|_{\tau=0}&=0.
\end{align*}
Using this result and that we just showed the resulting ODE for $E_j$ we now have that:
\begin{align*}
    \left ( \left. \frac{\partial^2 E_j(\tau;u)}{\partial \tau^2}\right|_{\tau=0} \right )-(\rho u \sigma i-b_j)\underbrace{\left (\left. \frac{\partial E_j(\tau;u)}{\partial \tau}\right|_{\tau=0} \right )}_{= 0} + \frac{\sigma^2}{2} \left( -\frac{1}{2}u^2 + u_j ui \right) E_j(0;u)&= 0\\
    &\iff\\
    \left ( \left. \frac{\partial^2 E_j(\tau;u)}{\partial \tau^2}\right|_{\tau=0} \right ) + \frac{\sigma^2}{2} \left( -\frac{1}{2}u^2 + u_j ui \right) E_j(0;u) &= 0\\
    &\iff\\
    -\frac{2}{\sigma^2}\frac{\left.\frac{\partial^2 E_j(\tau;u)}{\partial \tau^2}\right|_{\tau=0}}{\left( -\frac{1}{2}u^2 + u_j ui \right)}&=E_j(0;u).
\end{align*}
yielding the condition for $E_j$ at $\tau=0$ (i.e expiry).

$\textcolor{blue}{\star}$ For $\tau=0$, we achieve the following form of
equation (10):
\begin{align*}
    E_j(0;u)&=A_je^{x_{j,+}\cdot 0}+B_je^{x_{j,-}\cdot 0}\\&
    =A_j\cdot 1+B_j \cdot 1\\
    &=A_j + B_j,
\end{align*}
showing the first property.

We now find the derivative of equation (10) to find the initial condition of
said derivative:
\begin{align*}
    \frac{\partial E_j(\tau;u)}{\partial \tau}=x_{j,+}A_je^{x_{j,+}\cdot\tau}+x_{j,-}B_je^{x_{j,-}\cdot\tau}.
\end{align*}
This now yields the initial condition of the derivative of $E_j$:
\begin{align*}
    \left. \frac{\partial E_j(\tau;u)}{\partial \tau} \right|_{\tau=0}&=x_{j,+}A_je^{x_{j,+}\cdot 0}+x_{j,-}B_j e^{x_{j,-}\cdot 0}\\
    &=x_{j,+} \cdot 1 \cdot A_j+x_{j,-} \cdot 1 \cdot B_j \\
    &=x_{j,+} A_j+x_{j,-} B_j
\end{align*}

$\textcolor{blue}{\star}$ Once again, we use the initial conditions for $D_j$,
namely, $D_j(0;u)=0$, for $\tau=0$. For equation (9) to hold, it must then be the case
that:
\begin{align*}
    \left. \frac{\partial E_j(\tau;u)}{\partial \tau} \right|_{\tau=0}=0.
\end{align*}
This gives us the bi-implications:
\begin{align*}
    \left. \frac{\partial E_j(\tau;u)}{\partial \tau} \right|_{\tau=0}&=0\\
    &\:\iff\\
    x_{j,+} A_j+x_{j,-} B_j&=0\\
    \:&\:\overset{\dagger}{\iff}\\
    -A_j=\frac{x_{j,-}}{x_{j,+}}B_j=g_jB_j &\quad \text{and}\quad B_j=\frac{x_{j,+}}{x_{j,-}}A_j=-g_j^{-1}A_j,
\end{align*}
where $\dagger$ follows from definition, namely, $g_j=\frac{x_{j,-}}{x_{j,+}}$.
Using our just showed result, $E_j(0;u)=A_j+B_j$, we find that:
\begin{align*}
    -A_j=g_jB_j&=g_j(E_j(0;u)-A_j)\\
    &\iff\\
    A_j(g_j-1)&=g_jE_j(0;u)\\
    &\iff\\
    A_j&=\frac{g_jE_j(0;u)}{g_j-1},
\end{align*}
implying by the found relations of $A_j,B_j$:
\begin{align*}
    B_j=-g_j^{-1}A_j=-g_j^{-1}\frac{g_jE_j(0;u)}{g_j-1}=-\frac{E_j(0;u)}{g_j-1}.
\end{align*}

$\textcolor{blue}{\star}$ It now follows quite simply and nicely from
substitution of the
previous results that:
\begin{align*}
    E_j(\tau;u)&=A_je^{x_{j,+}\cdot \tau}+B_j e^{x_{j,-}\cdot \tau}\\
    &=\frac{g_jE_j(0;u)}{g_j-1}e^{x_{j,+}\cdot \tau}-\frac{E_j(0;u)}{g_j-1}e^{x_{j,-}\cdot \tau}\\
    &=\frac{E_j(0;u)}{g_j-1}\left ( g_je^{x_{j,+}\cdot \tau}-e^{x_{j,-}\cdot \tau}\right ).
\end{align*}

$\textcolor{blue}{\star}$ We use the found relations for $A_j,B_j$ to find: 
\begin{align*}
    \frac{\partial E_j(\tau;u)}{\partial \tau}&=x_{j,+}A_je^{x_{j,+}\cdot \tau}+x_{j,-}B_je^{x_{j,-}\cdot\tau}\\
    &=x_{j,+}\frac{g_jE_j(0;u)}{g_j-1}e^{x_{j,+}\tau}-x_{j,-}\frac{E_j(0;u)}{g_j-1}e^{x_{j,-}\cdot\tau}\\
    &=\frac{E_j(0;u)}{g_j-1}\left (x_{j,+}g_je^{x_{j,+}\cdot \tau} -x_{j,-}e^{x_{j,-}\cdot \tau}\right ).
\end{align*}

$\textcolor{blue}{\star}$ Substituting $E_j$ and $(\partial E_j)/(\partial
\tau)$ into equation (9) for $D_j$
now yields the altered expression:
\begin{align*}
    D_j(\tau;u)&=-\frac{\frac{\partial E_j(\tau;u)}{\partial \tau}}{\frac{1}{2}\sigma^2E_j(\tau;u)}\\
    &=-\frac{\frac{E_j(0;u)}{g_j-1}\left (x_{j,+}g_je^{x_{j,+}\cdot \tau} -x_{j,-}e^{x_{j,-}\cdot \tau}\right )}{\frac{1}{2}\sigma^2 \frac{E_j(0;u)}{g_j-1}\left ( g_je^{x_{j,+}\cdot \tau}-e^{x_{j,-}\cdot \tau}\right )}
\end{align*}
Now, substituting in equation (11) and (15) into the altered equation (9) yields:
\begin{align*}
    D_j(\tau ; u)&=-\frac{\frac{E_j(0 ; u)}{g_j-1}\left(x_{j,+} g_j \exp\left\{x_{j,+} \cdot \tau\right\}-x_{j,-} \exp\left\{x_{j,-} \cdot \tau\right\}\right)}{\frac{\sigma^2}{2} \frac{E_j(0;u)}{g_j-1}\left(g_j \exp\left\{x_{j,+} \cdot \tau\right\}-\exp\left\{x_{j,-} \cdot \tau\right\}\right)} \\
    & =-\frac{\frac{E_j(0 ; u)}{g_j-1}\left(\frac{\rho \sigma u i-b_j+d_j}{2} g_j \exp\left\{\frac{\rho \sigma u i-b_j+d_j}{2} \tau\right\}-\frac{\rho \sigma u i-b_j-d_j}{2} \exp\left\{\frac{\rho \sigma u i-b_j-d_j}{2} \tau\right\}\right)}{\frac{\sigma^2}{2} \frac{E_j(0 ; u)}{g_j-1}\left(g_j \exp\left\{\frac{\rho \sigma u i-b_j+d_j}{2} \tau\right\}-\exp\left\{\frac{\rho \sigma u i-b_j-d_j}{2} \tau\right\}\right)} \\
    & =-\frac{\frac{\rho \sigma u i-b_j+d_j}{2} g_j \exp\left\{\frac{\rho \sigma u i-b_j+d_j}{2} \tau\right\}-\frac{\rho \sigma u i-b_j-d_j}{2} \exp\left\{\frac{\rho \sigma u i-b_j-d_j}{2} \tau\right\}}{\frac{\sigma^2}{2}\left(g_j \exp\left\{\frac{\rho \sigma u i-b_j+d_j}{2} \tau\right\}-\exp\left\{\frac{\rho \sigma u i-b_j-d_j}{2} \tau\right\}\right)} \\
    & =-\frac{\frac{\rho \sigma u i-b_j+d_j}{2} g_j \exp\left\{\frac{d_j}{2} \tau\right\}-\frac{\rho \sigma u i-b_j-d_j}{2} \exp\left\{-\frac{d_j}{2} \tau\right\}}{\frac{\sigma^2}{2}\left(g_j \exp\left\{\frac{d_j}{2} \tau\right\}-\exp\left\{-\frac{d_j}{2} \tau\right\}\right)} \\
    & =-\frac{\left(\rho \sigma u i-b_j+d_j\right) g_j \exp\left\{\frac{d_j}{2} \tau\right\}-\left(\rho \sigma u i-b_j-d_j\right) \exp\left\{-\frac{d_j}{2} \tau\right\}}{\sigma^2\left(g_j \exp\left\{\frac{d_j}{2} \tau\right\}-\exp\left\{-\frac{d_j}{2} \tau\right\}\right)} \\
    & \overset{\dagger}{=}-\frac{\left(\rho \sigma u i-b_j+d_j\right) g_j \exp\left\{\frac{d_j}{2} \tau\right\} \exp\left\{\frac{d_j}{2} \tau\right\}-\left(\rho \sigma u i-b_j-d_j\right) \exp\left\{-\frac{d_j}{2} \tau\right\} \exp\left\{\frac{d_j}{2} \tau\right\}}{\sigma^2\left(g_j \exp\left\{\frac{d_j}{2} \tau\right\} \exp\left\{\frac{d_j}{2} \tau\right\}-\exp\left\{-\frac{d_j}{2} \tau\right\} \exp\left\{\frac{d_j}{2} \tau\right\}\right)} \\
    & =-\frac{\left(\rho \sigma u i-b_j+d_j\right) g_j \exp\left\{d_j \tau\right\}-\left(\rho \sigma u i-b_j-d_j\right)}{\sigma^2\left(g_j \exp\left\{d_j \tau\right\}-1\right)} \\
    & \overset{\dagger\dagger}{=}-\frac{\left(\rho \sigma u i-b_j+d_j\right) \frac{\rho \sigma u i-b_j-d_j}{\rho \sigma u i-b_j+d_j} \exp\left\{d_j \tau\right\}-\left(\rho \sigma u i-b_j-d_j\right)}{\sigma^2\left(g_j \exp\left\{d_j \tau\right\}-1\right)} \\
    & =-\frac{\left(\rho \sigma u i-b_j-d_j\right) \exp\left\{d_j \tau\right\}-\left(\rho \sigma u i-b_j-d_j\right)}{\sigma^2\left(g_j \exp\left\{d_j \tau\right\}-1\right)} \\
    & \overset{\dagger\dagger\dagger}{=}-\frac{\left(b_j-\rho \sigma u i+d_j\right) \exp\left\{d_j \tau\right\}-\left(b_j-\rho \sigma u i+d_j\right)}{\sigma^2\left(1-g_j \exp\left\{d_j \tau\right\}\right)} \\
    & =\frac{\left(b_j-\rho \sigma u i+d_j\right)\left(1-\exp\left\{d_j \tau\right\}\right)}{\sigma^2\left(1-g_j \exp\left\{d_j \tau\right\}\right)} \\
    & =\frac{b_j-\rho \sigma u i + d_j}{\sigma^2} \frac{1-\exp\left\{d_j \tau\right\}}{1-g_j \exp\left\{d_j \tau\right\}},
\end{align*}
where $\dagger$ follows by multiplication of $\exp{\left \{\frac{d_j}{2}\tau
\right \}}$, $\dagger\dagger$ from $g:=x_{j,-}/x_{j,+}$ and
$\dagger\dagger\dagger$ by multiplication of $-1$.

$\textcolor{blue}{\star}$  As stated at the very start, finding the solution of
$D_j$ is advantageous to derive the solution for $C_j$ as the ODE for $C_j$ was
given by $(*)$ containing $D_j$ and yielding by integration:
\begin{align*}
    \frac{\partial C_j}{\partial \tau}&=rui+aD_j\\
    &\iff\\
    C_j&=\int_0^\tau rui+aD_j(s;u)\:ds\\
    &=rui\int_0^\tau \:ds+a\int_0^\tau D_j(s;u)\:ds\\
    &=rui\lc \tau-0\rc+a\int_0^\tau D_j(s;u)\:ds\\
    &=rui\tau+a\int_0^\tau D_j(s;u)\:ds.
\end{align*}
As we wanted by solving the ODE of $D_j$ first we only have to find the integral
involving $D_j$. We proceed with finding the integral involving $D_j$:
\begin{align*}
    \int_0^\tau D_j(s;u)\:ds&\overset{\dagger}{=}\int_0^\tau \frac{b_j-\rho \sigma u i + d_j}{\sigma^2} \frac{1-\exp\left\{d_j \tau\right\}}{1-g_j \exp\left\{d_j \tau\right\}} \:ds\\
    &= \frac{b_j-\rho \sigma u i + d_j}{\sigma^2}\cdot \int_0^\tau\frac{1-\exp\left\{d_j \tau\right\}}{1-g_j \exp\left\{d_j \tau\right\}} \:ds\\
    &\overset{\dagger\dagger}{=}\frac{b_j-\rho \sigma u i+d j}{\sigma^2} \frac{1}{d_j} \cdot\int_{1}^{\exp\left\{d_j \tau\right\}} \frac{1-y}{1-g_j y} \frac{1}{y} \:dy \\
    & =\frac{b_j-\rho \sigma u i+d j}{\sigma^2} \frac{1}{d_j} \cdot\int_{1}^{\exp\left\{d_j \tau\right\}} \frac{1-y}{y\left(1-g_j y\right)} \:dy \\
    & =\frac{b_j-\rho \sigma u i+d j}{\sigma^2} \frac{1}{d_j}\cdot \int_{1}^{\exp\left\{d_j \tau\right\}} \frac{1-g_j y-\left(y-g_j y\right)}{y\left(1-g_j y\right)} \:dy \\
    & =\frac{b_j-\rho \sigma u i+d j}{\sigma^2} \frac{1}{d_j} \cdot\int_{1}^{\exp\left\{d_j \tau\right\}} \frac{1}{y}-\frac{1-g_j}{1-g_j y} \:dy \\
    & =\frac{b_j-\rho \sigma u i+d j}{\sigma^2} \frac{1}{d_j}\cdot\left(\int_{1}^{\exp\left\{d_j \tau\right\}} \frac{1}{y} d y-\int_{1}^{\exp\left\{d_j \tau\right\}} \frac{1-g_j}{1-g_j y} d y\right)\\
    &\overset{\dagger\dagger\dagger}{=}\frac{b_j-\rho \sigma u i+d j}{\sigma^2} \frac{1}{d_j}\cdot \left ( \left [ \log{(y)}\right ]_1^{\exp\left\{d_j \tau\right\}}+\frac{1-g_j}{g_j} \left [ \log{(1-g_jy)}\right ]_1^{\exp\left\{d_j \tau\right\}}\right )\\
    & =\frac{b_j-\rho \sigma u i+d j}{\sigma^2} \frac{1}{d_j}\\&\quad \cdot \left(\left[\log \left(\exp\left\{d_j \tau\right\} \right)-\log{(1)}\right]+\frac{1-g_j}{g_j}\left[\log \left(1-g_j \exp\left\{d_j \tau\right\}\right)-\log \left(1-g_j \right)\right]\right) \\
    &=\frac{b_j-\rho \sigma u i+d j}{\sigma^2} \frac{1}{d_j}\cdot\left(d_j \tau+\frac{1-g_j}{g_j} \log \left(\frac{1-g_j \exp\left\{d_j \tau\right\}}{1-g_j}\right)\right),
\end{align*}
where $\dagger$ follows from our just found expression for $D_j$,
$\dagger\dagger$ from substition using:
\begin{align*}
    y=\exp\left\{d_j s\right\}, \quad dy=d_j\exp\left\{d_j s\right\}ds,\quad ds=\frac{dy}{d_j y},
\end{align*}
and the fact that when at $\tau \Rightarrow \exp\left\{d_j \tau\right\}$ and at
zero $\Rightarrow \exp\left\{d_j 0 \right\}=1$. Lastly, $\dagger\dagger\dagger$ follows from:
\begin{align*}
    \frac{d}{dy}\frac{1-g_j}{g_j}\log{(1-g_j y)}&=\frac{1}{1-g_jy}\frac{1-g_j}{g_j}(-g_j)\\
    &=-\frac{1-g_j}{1-g_jy}\\
    &\Rightarrow\\
    -\int\frac{1-g_j}{1-g_jy}\:dy&=\frac{1-g_j}{g_j}\log{(1-g_jy)}(+C).
\end{align*}
\newpage
We are not quite done as we still need to substitute the found integral back
into the expression for $C_j$:
\begin{align*}
    C_j&=rui\tau+a\int_0^\tau D_j(s;u)\:ds.\\
    & =rui\tau+a \frac{b_j-\rho \sigma u i+d_j}{\sigma^2} \cdot\frac{1}{d_j}\left(d_j \tau+\frac{1-g_j}{g_j} \log \left(\frac{1-g_j \exp\left\{d_j \tau\right\}}{1-g_j}\right)\right) \\
    & =rui\tau+\frac{a}{\sigma^2}\cdot\left(\left(b_j-\rho \sigma u i+d_j\right) \tau+\frac{1}{d_j}\left(b_j-\rho \sigma u i+d_j\right) \frac{1-g_j}{g_j} \log \left(\frac{1-g_j \exp\left\{d_j \tau\right\}}{1-g_j}\right)\right) \\
    & \overset{\dagger}{=}rui\tau+\frac{a}{\sigma^2}\cdot\left(\left(b_j-\rho \sigma u i+d_j\right) \tau+\frac{1}{d_j}\left(b_j-\rho \sigma u i+d_j\right) \frac{\rho-\frac{\rho \sigma u i-b_j-d_j}{\rho \sigma u i-b_j+d_j}}{\left.\frac{\rho \sigma u i-b_j-d_j}{\rho \sigma u i-b_j+d_j} \log \left(\frac{1-g_j \exp\left\{d_j \tau\right\}}{1-g_j}\right)\right)}\right. \\
    &=rui\tau+\frac{a}{\sigma^2}\cdot\left(\left(b_j-\rho \sigma u i+d_j\right) \tau+\frac{1}{d_j}\left(b_j-\rho \sigma u i+d_j\right) \frac{\left(\rho \sigma u i-b_j+d_j\right)-\left(\rho \sigma u i-b_j-d_j\right)}{\rho \sigma u i-b_j-d_j} \log \left(\frac{1-g_j \exp\left\{d_j \tau\right\}}{1-g_j}\right)\right) \\
    & =rui\tau+\frac{a}{\sigma^2}\cdot\left(\left(b_j-\rho \sigma u i+d_j\right) \tau+\frac{1}{d_j}\left(b_j-\rho \sigma u i+d_j\right) \frac{2 d_j}{\rho \sigma u i-b_j-d_j} \log \left(\frac{1-g_j \exp\left\{d_j \tau\right\}}{1-g_j}\right)\right) \\
    & =rui\tau+\frac{a}{\sigma^2}\cdot\left(\left(b_j-\rho \sigma u i+d_j\right) \tau-\frac{1}{d_j} 2 d_j \frac{\rho \sigma u i-b_j-d_j}{\rho \sigma u i-b_j-d_j} \log \left(\frac{1-g_j \exp\left\{d_j \tau\right\}}{1-g_j}\right)\right) \\
    &=rui\tau+\frac{a}{\sigma^2}\cdot\left(\left(b_j-\rho \sigma u i+d_j\right) \tau-2 \log \left(\frac{1-g_j \exp\left\{d_j \tau\right\}}{1-g_j}\right)\right),
\end{align*}

where $\dagger$ follows from the definition of $g_j=x_{j,-}/x_{j,+}$.
\newpage
\subsection{Numerical implementation of characteristic functions
}
$\textcolor{blue}{\star}$ All the code can be viewed in \autoref{chap:code}. 

$\textcolor{blue}{\star}$ From \cite{Havrylenko2024}, pp. 59-65 we define a
function to compute the characteristic functions $\Psi_j$, $j=1,2$, given by:
\begin{align*}
    \Psi_{x(T)}^j(x,v,t;u)=\exp \lc C_j (\tau; u) + D_j (\tau; u) \cdot v + i  u x \rc,
\end{align*}
with:
\begin{align*}
    C(\tau ; u) &= rui\tau + \frac{a}{\sigma^2}\left( (b_j - \rho\sigma ui + d_j) \tau - 2 \log \left( \frac{1 - g_j e^{d_j \tau}}{1 - g_j} \right) \right), \\
    D_j(\tau ; u) &= \frac{b_j - \rho \sigma ui + d_j}{\sigma^2} \frac{1 - e^{d_j \tau}}{1 - g_j e^{d_j \tau}}, \\
    g_j &= \frac{b_j - \rho\sigma ui + d_j}{b_j - \rho\sigma ui - d_j}, \\
    d_j &= \sqrt{(\rho\sigma ui - b_j)^2 - \sigma^2(2u_jui - u^2)},\\
    u_1 &= \frac{1}{2}, \quad u_2 = -\frac{1}{2}, \quad a = \kappa\theta, \quad b_1 = \kappa + \lambda - \rho\sigma, \quad b_2 = \kappa + \lambda,
\end{align*}
that takes parameters seen in Table 1 with given specific values.
\begin{table}[!h]
    \centering
    \begin{tabular}{lll}
    \toprule
    \textbf{Description} & \textbf{Symbol} & \textbf{Value} \\
    \midrule
    Current Stock Price & $S_t$ & 100 \\
    Current Variance & $v_t$ & 0.06 \\
    Risk-free Interest Rate & $r$ & 0.05 \\
    Speed of Mean Reversion & $\kappa$ & 1 \\
    Long-term Mean Variance & $\theta$ & 0.06 \\
    Volatility of the Volatility & $\sigma$ & 0.3 \\
    Correlation Coefficient & $\rho$ & -0.5 \\
    Market Price of Risk & $\lambda$ & 0.01 \\
    Time to Expiry & $\tau$ & 1 \\
    \bottomrule
    \end{tabular}
    \caption{Parameters for Computing Characteristic Functions $\Psi_1$, $\Psi_2$.}
    \label{table:characteristic_functions}
\end{table}

\newpage
The plot seen in Figure 1 is of the real and imaginary part of
$\Psi_j$, $j=1,2$ computed using the above defined function in \texttt{Python}.
The pattern for the real and imaginary part is seen, as oscillating with peaks
around the middle of the interval
for $u\in [-20,20]$, i.e near $u \approx 0$, and lower amplitude oscillations around the
ends, i.e near $u\approx \pm 20$. More specifically, we see that the real part
of the characteristic functions is even but the imaginary part is odd. The graph
of $\mathfrak{Im}(\Psi_j)$ has odd symmetry, meaning it has a rotational
symmetry about the origin, and the graph of $\mathfrak{Re}(\Psi_j)$ has even
symmetry, meaning it has symmetry about the second axis - although it requires
some \textit{imagination} as the second axis is not centered.
\begin{figure}[!h]
    \begin{adjustwidth}{-2cm}{-2cm}  % Adjust these values as needed
      \centering
      \includegraphics[width=1\linewidth]{1.3.png} % You can adjust this width as needed
      \label{Figure 1}
    \end{adjustwidth}
    \captionsetup{skip=-25pt} % Adjust the space between the figure and the caption
    \caption{Real and imaginary part of $\Psi_j$, $j=1,2$ for $u\in [-20,20]$
    computed using the \texttt{characteristicFunctionHeston} function in
    \texttt{Python} with code avaliable in \autoref{chap:code}.}
\end{figure}

\newpage
\section{Option pricing in the context of the Heston model}
We will report results with five significant figures throughout this section of
the HandIn. When references are given to (x.y) (for example 2.1) it is to be understood as
subsection (x.y) (subsection 2.1) unless otherwise stated.
\subsection{Monte Carlo and Euler discretization Scheme}
$\textcolor{blue}{\star}$ All the code can be viewed in \autoref{chap:code}.

$\textcolor{blue}{\star}$ We simulate $S_t$ over the time interval $[0, T]$ (to
and including expiry),
which we assume to be discretized as $0 = t_1 < t_2 < \ldots < t_m = T$, where
the time increments are equally spaced with width $d t$. For $n\in\NN$ we use
the Euler discretization on an equidistan time grid $\lc t_i=\frac{iT}{n}\mid
i=0,\ldots,n\rc$, where $n$ is the number of time points after $t=0$. From
\cite{Rouah2024}, p. 4 we start with the initial
values $S_0$ for the stock price and $v_0$ for the variance. Given a value for
$v_t$ at time $t$, we first obtain the discretization of $v_t$ by $v_{t+dt}$ from:
\begin{align*}
    v_{t+dt} = v_t + \kappa(\theta - v_t)dt + \sigma \sqrt{v_t dt}Z_v,
\end{align*}
and we obtain the discretization of $S_t$ by $S_{t+dt}$ from:
\begin{align*}
    S_{t+dt} = S_t + rS_tdt + \sqrt{v_t dt}S_tZ_s,
\end{align*}
where to generate $Z_v$ and $Z_s$ with correlation $\rho$, we first generate two independent standard normal variables $Z_1$ and $Z_2$, and set $Z_v = Z_1$ and $Z_s = \rho Z_1 + \sqrt{1 - \rho^2} Z_2$.
For an arbitrary but fixed $\lambda$ (i.e no estimation of $\lambda$ needed) we use the adjusments:
\begin{align*}
    \tilde{\kappa}=\kappa+\lambda \quad \text{ and } \quad \tilde{\theta}=\frac{\kappa\theta}{\kappa + \lambda},
\end{align*}
and thus:
\begin{align*}
    v_{t+dt} = v_t +\tilde{\kappa}(\tilde{\theta} - v_t)dt + \sigma \sqrt{v_t dt}Z_v,
\end{align*}
yielding that the price under the equivalent martingale measure
$\tilde{\QQ}(\lambda)$ is unique, i.e choosing a different arbitrary
$\lambda_1\neq \lambda$ would yield another equivalent martingale measure
$\tilde{\QQ}(\lambda_1)$ among the infinite number of measures.

Lastly, we note that there might occure negative $v$-values at some of the time
points due to discretization errors even if the Feller condition is met (see
\cite{Havrylenko2024}, p. 28). When this occurs we use a full truncation
scheme, namely, $\max\lc 0,v_t\rc$ but firstly count it towards
our zero-variance-count for later reporting.
\newpage
The simulation in \texttt{Python} is done with the parameters seen in Table 2.
\begin{table}[!h]
    \centering
    \begin{tabular}{lll}
    \toprule
    \textbf{Parameter} & \textbf{Symbol} & \textbf{Value} \\
    \midrule
    Initial Stock Price & $S_0$ & 100 \\
    Initial Variance & $v_0$ & 0.06 \\
    Risk-free Interest Rate & $r$ & 0.05 \\
    Speed of Mean Reversion & $\kappa$ & 1 \\
    Long-term Mean Variance & $\theta$ & 0.06 \\
    Volatility of the Volatility & $\sigma$ & 0.3 \\
    Correlation Coefficient & $\rho$ & -0.5 \\
    Market Price of Risk & $\lambda$ & 0.01 \\
    Expiry Time & $T$ & 1 \\
    Strike Price & $K$ & 100 \\
    Number of Time Points & $n$ & 100 \\
    Number of Paths & $N$ & 1000 \\
    \bottomrule
    \end{tabular}
    \caption{Simulation and model parameters for the Heston model to calculate the European call option price Monte Carlo and Euler discretization scheme.}
    \label{table:call_option_pricing_parameters1}
    \end{table}

The results from simulation in \texttt{Python} for three different seeds
(\texttt{np.random.seed()}) is seen in Table 3.

\begin{table}[ht]
    \centering
    \begin{tabular}{|l|c|c|c|}
    \hline
    \multicolumn{4}{|c|}{\textbf{Euler Monte Carlo}} \\ \hline
    \textbf{}  & \textbf{Seed A} & \textbf{Seed B} & \textbf{Seed C} \\ \hline
    \textbf{Option Price} & 12.247 & 11.849 & 11.781 \\
    \textbf{Standard Deviation} & 0.51792 & 0.51211 & 0.52947 \\
    \textbf{Computing Time} & 0.70570 & 0.61735 & 0.70212 \\
    \textbf{0-Variance-Count} & 52 & 47 & 36 \\ \hline
    \end{tabular}
    \caption{Simulation results using Euler Monte Carlo discretization for seeds $\text{\textbf{A}}=1$, $\text{\textbf{B}}=10$ and $\text{\textbf{C}}=100$.}
    \label{table:simulation_results1}
\end{table}


Observe that the option price depends upon the seed-number which implies that the
Monte Carlo, and specifically Euler discretization, scheme might not be precise enough.
Furthermore, we kept the number of simulations fixed, as requested. If this
constraint was lifted and simulation was done at a higher number of paths the
results for every seed would differ as well. These findings for Monte Carlo
follow exactly as given by
\cite{Havrylenko2024}, p. 73.
\newpage
\subsection{Monte Carlo and Milstein discretization scheme}
$\textcolor{blue}{\star}$ All the code can be viewed in \autoref{chap:code}.

$\textcolor{blue}{\star}$ We refer to the logical build up for the functions
used in 2.1 but shortly explain the implementation of Monte Carlo method but now with
the Milstein discretization scheme. From \cite{Rouah2024}, p. 8 given a value for
$v_t$ at time $t$, we first obtain the discretization of $v_t$ by $v_{t+dt}$ from:
\begin{align*}
    v_{t+dt} = v_t + \kappa(\theta - v_t)dt + \sigma \sqrt{v_tdt}Z_v+\frac{1}{4}\sigma^2dt\left (Z_v^2-1 \right ),
\end{align*}
and we obtain the discretization of $S_t$ by $S_{t+dt}$ from:
\begin{align*}
    S_{t+dt} = S_t + rS_tdt + \sqrt{v_t dt}S_tZ_s+\frac{1}{4}S_t^2dt(Z_s^2-1 ),
\end{align*}
where to generate $Z_v$ and $Z_s$ with correlation $\rho$, we first generate two independent standard normal variables $Z_1$ and $Z_2$, and set $Z_v = Z_1$ and $Z_s = \rho Z_1 + \sqrt{1 - \rho^2} Z_2$.
For an arbitrary but fixed $\lambda$ (i.e no estimation of $\lambda$ needed) we use the adjusments:
\begin{align*}
    \tilde{\kappa}=\kappa+\lambda \quad \text{ and } \quad \tilde{\theta}=\frac{\kappa\theta}{\kappa + \lambda},
\end{align*}
and thus:
\begin{align*}
    v_{t+dt} = v_t +\tilde{\kappa}(\tilde{\theta} - v_t)dt + \sigma \sqrt{v_tdt}Z_v+\frac{1}{4}\sigma^2dt\left (Z_v^2-1 \right ).
\end{align*}
The simulation in \texttt{Python} is done with the parameters seen in Table 4.
\begin{table}[!h]
    \centering
    \begin{tabular}{lll}
    \toprule
    \textbf{Parameter} & \textbf{Symbol} & \textbf{Value} \\
    \midrule
    Initial Stock Price & $S_0$ & 100 \\
    Initial Variance & $v_0$ & 0.06 \\
    Risk-free Interest Rate & $r$ & 0.05 \\
    Speed of Mean Reversion & $\kappa$ & 1 \\
    Long-term Mean Variance & $\theta$ & 0.06 \\
    Volatility of the Volatility & $\sigma$ & 0.3 \\
    Correlation Coefficient & $\rho$ & -0.5 \\
    Market Price of Risk & $\lambda$ & 0.01 \\
    Expiry Time & $T$ & 1 \\
    Strike Price & $K$ & 100 \\
    Number of Time Points & $n$ & 100 \\
    Number of Paths & $N$ & 1000 \\
    \bottomrule
    \end{tabular}
    \caption{Simulation and model parameters for the Heston model to calculate the European call option price Monte Carlo and Milstein discretization scheme.}
    \label{table:call_option_pricing_parameters2}
    \end{table}
\newpage
The results from simulation in \texttt{Python} for three different seeds
(\texttt{np.random.seed()}) is seen in Table 5.

\begin{table}[ht]
    \centering
    \begin{tabular}{|l|c|c|c|}
    \hline
    \multicolumn{4}{|c|}{\textbf{Milstein Monte Carlo}} \\ \hline
    \textbf{} & \textbf{Seed A} & \textbf{Seed B} & \textbf{Seed C} \\ \hline
    \textbf{Option Price} & 12.253 & 11.827 & 11.778 \\
    \textbf{Standard Deviation} & 0.51785 & 0.51029 & 0.52915 \\
    \textbf{Computing Time} & 0.75349 & 0.78604 & 0.92265 \\
    \textbf{0-Variance-Count} & 0 & 0 & 0 \\ \hline
    \end{tabular}
    \caption{Simulation results using Monte Carlo and Milstein discretization for seeds $\text{\textbf{A}}=1$, $\text{\textbf{B}}=10$ and $\text{\textbf{C}}=100$.}
    \label{table:simulation_results2}
\end{table}


The option prices, standard deviations and computing time are of the same ball park as the ones
given in 2.1 (EulerMC). It should be noted that computation time is highly
reliant on the state of your computer at that given moment and the drag on your CPU/GPU. However, note that the 0-Variance-Count is $0$ for
all three seeds. This means that the Milstein discretization scheme seems to
produce roughly the same issues as the Euler discretization with the important
difference of producing far fewer (here, none) instances of a variance having to
be truncated. These findings for Monte Carlo
follow as given by
\cite{Havrylenko2024}, p. 73 and the observation for the 0-Variance-count is
exactly as mentioned by \cite{Rouah2024}, p. 7. The reason for this difference
is that the key to the Milstein Scheme is that the accuracy of the
discretization is increased by considering expansion of specifically $\mu_t=\mu(v_t)$ and
$\sigma_t=\sigma(v_t)$ by Ito's lemma. This adds the correction which differs
from EulerMC (2.1):
\begin{align*}
    v_{t+dt} = v_t +\tilde{\kappa}(\tilde{\theta} - v_t)dt + \sigma \sqrt{v_tdt}Z_v\underbrace{+\frac{1}{4}\sigma^2dt\left (Z_v^2-1 \right )}_{\text{correction}}.
\end{align*}
However, even though no instances required truncation under these three seeds, the full
truncation scheme must still be applied to $v_{t+dt}$ as this is not a guarantee.
\subsection{Heston’s original formula}
$\textcolor{blue}{\star}$ All the code can be viewed in \autoref{chap:code}.

$\textcolor{blue}{\star}$ In this subsection we employ Heston's original formula
to compute the price of the European call option on the underlying asset $S$
following the Heston model. From \cite{Havrylenko2024}, pp. 59-66, analogous to
(1.3), we define a function \texttt{characteristicFunctionHeston} to compute the
characteristic functions $\Psi_j$, $j=1,2$, given by:
\begin{align*}
    \Psi_{x(T)}^j(x,v,t;u)=\exp \lc C_j (\tau; u) + D_j (\tau; u) \cdot v + i  u x \rc,
\end{align*}
with:
\begin{align*}
    C_j(\tau ; u) &= rui\tau + \frac{a}{\sigma^2}\left( (b_j - \rho\sigma ui + d_j) \tau - 2 \log \left( \frac{1 - g_j e^{d_j \tau}}{1 - g_j} \right) \right), \\
    D_j(\tau ; u) &= \frac{b_j - \rho \sigma ui + d_j}{\sigma^2} \frac{1 - e^{d_j \tau}}{1 - g_j e^{d_j \tau}}, \\
    g_j &= \frac{b_j - \rho\sigma ui + d_j}{b_j - \rho\sigma ui - d_j}, \\
    d_j &= \sqrt{(\rho\sigma ui - b_j)^2 - \sigma^2(2u_jui - u^2)},\\
    u_1 &= \frac{1}{2}, \quad u_2 = -\frac{1}{2}, \quad a = \kappa\theta, \quad b_1 = \kappa + \lambda - \rho\sigma, \quad b_2 = \kappa + \lambda.
\end{align*}
Using the characteristic function, the price of the European call option under
$\tilde{\QQ}$ in the Heston model corresponding to $\lambda$ using Heston's
original formula is given by \cite{Havrylenko2024}, p. 66 as:
\begin{align*}
    Call^{\text{OriginalFT}}(t)=S_tQ_1(\log(S(t)),v(t),t;ln(K))-Ke^{-r\tau}Q_2(\log(S(t)),v(t),t;\log(K)),
\end{align*}
where for $j=1,2$:
\begin{align*}
    Q_j(\log(S(t)),v(t),t;ln(K))=\frac{1}{2}+\frac{1}{\pi}\int_0^{+\infty} \mathfrak{Re} \left ( \frac{e^{-iu\log(K)}\Psi^j_{\log(S(T))}(\log(S(t)),v(t),t;u)}{iu}\right )du.
\end{align*}
Note that in theory we do not need to consider the real part of the integrand
but as we need to truncate the integral the need arises as there will be an
imaginary part. 

Pricing of the call option is then calculated using the parameters
seen in
Table 6.
\begin{table}[!h]
    \centering
    \begin{tabular}{lll}
    \toprule
    \textbf{Parameter} & \textbf{Symbol} & \textbf{Value} \\
    \midrule
    Initial Stock Price & $S_0$ & 100 \\
    Initial Variance & $v_0$ & 0.06 \\
    Risk-free Interest Rate & $r$ & 0.05 \\
    Speed of Mean Reversion & $\kappa$ & 1 \\
    Long-term Mean Variance & $\theta$ & 0.06 \\
    Volatility of the Volatility & $\sigma$ & 0.3 \\
    Correlation Coefficient & $\rho$ & -0.5 \\
    Market Price of Risk & $\lambda$ & 0.01 \\
    Time to Expiry & $\tau$ & 1 \\
    Strike Price & $K$ & 100 \\
    \bottomrule
    \end{tabular}
    \caption{Model parameters for the Heston model to calculate the European call option price using Heston's original formula.}
    \label{table:call_option_pricing_parameters3}
    \end{table}

However, as noted in \cite{Havrylenko2024}, p. 66, computation of the above
integrals $Q_j$, $j=1,2$, can be numerically tricky because of the imaginary
part of $\log$ in
$C_j$ as the imaginary part of $\log$ has a branch cut along the negative real
axis. To overcome this obstacle and evaluate the integral numerically
we employ the \texttt{quad}-function. The \texttt{quad}-function in
\texttt{Python}'s \texttt{scipy.integrate}-module \cite{2020SciPy-NMeth} performs numerical integration
of $Q_j$, $j=1,2$ over a specified (truncated) interval  using adaptive quadrature methods
from the \texttt{QUADPACK}-library. We evaluate both integrals in increase upper
bounds and then compare the European call price and computation time with the
results obtained from 2.1 (EulerMC) and 2.2 (MilsteinMC).

\newpage
The results can be seen in Table 7 with different increasing upper bounds.

\begin{table}[ht]
    \centering
    \begin{tabular}{|l|c|c|c|c|}
    \hline
    \multicolumn{5}{|c|}{\textbf{Heston's Original Formula}} \\ \hline
    \textbf{} & \textbf{Bounds $[0,10]$} & \textbf{Bounds $[0,50]$} & \textbf{Bounds $[0,100]$} & \textbf{Bounds $[0,1000]$} \\ \hline
    \textbf{Option Price} & 11.8122 & 11.936 & 11.936 & 11.936 \\
    \textbf{Computing Time} & $1.0023\times 10^{-3}$ & $2.9919 \times 10^{-3}$ &
    $4.9834 \times 10^{-3}$
    & $7.9796 \times 10^{-3}$ \\ \hline
    \end{tabular}
    \caption{Heston's original formula results using numerical integration with different (truncated) upper bounds: $[0,10]$, $[0,50]$, $[0,100]$  and $[0,1000]$.}
    \label{table:results3}
\end{table}

Table 7 results show that the option prices computed by Heston's original formula
through numerical integration are rougly the same as the ones from 2.1 (EulerMC)
and 2.2 (MilsteinMC). An important notice is with the increasing upper bound we
seem to (at five significant figures) achieve the same option price. Furthermore
this option price is not reliant on a seed. Aiming for a unique price for the
European call option, the features of not depending on a seed and achieving a
quickly stabilizing price as the upper bound increases are advantageous.  Both
of these two properties were lacking in 2.1 (EulerMC)
and 2.2 (MilsteinMC), however, do note that the call price is in the same ball
park which does mean they could be used for exotic option pricing.

The major difference in Heston's original formula compared to 2.1 (EulerMC)
and 2.2 (MilsteinMC) lies in the computating time. We see differences of up to a
factor of $100$ between the computing times of Hestons original formula relative to 2.1 (EulerMC)
and 2.2 (MilsteinMC). Note however the
computation time is indeed, and as expected, increasing in the bounds of
integration but so would the increase in paths for 2.1 (EulerMC)
and 2.2 (MilsteinMC). Advantagously, observe how precise Heston's original formula is
for a relatively small upper bound and increasing the upper bound
further seems to not effect the option price up to five significant figures.


It should again be noted that computation time is highly
reliant on the state of your computer at that given moment and the drag on your CPU/GPU.
\newpage
\subsection{Carr-Madan formula}
$\textcolor{blue}{\star}$ All the code can be viewed in \autoref{chap:code}.

$\textcolor{blue}{\star}$ In this subsection we employ Carr-Madan approach to
Fourier transform pricing
to compute the price of the European call option on the underlying asset $S$
following the Heston model. From \cite{Havrylenko2024}, 59-66, (somehow) analogous to
(1.3) and (2.3), we have the characteristic functions $\Psi_j$, $j=1,2$, given by:
\begin{align*}
    \Psi_{x(T)}^j(x,v,t;u)=\exp \lc C_j (\tau; u) + D_j (\tau; u) \cdot v + i  u x \rc,
\end{align*}
with:
\begin{align*}
    C_j(\tau ; u) &= rui\tau + \frac{a}{\sigma^2}\left( (b_j - \rho\sigma ui + d_j) \tau - 2 \log \left( \frac{1 - g_j e^{d_j \tau}}{1 - g_j} \right) \right), \\
    D_j(\tau ; u) &= \frac{b_j - \rho \sigma ui + d_j}{\sigma^2} \frac{1 - e^{d_j \tau}}{1 - g_j e^{d_j \tau}}, \\
    g_j &= \frac{b_j - \rho\sigma ui + d_j}{b_j - \rho\sigma ui - d_j}, \\
    d_j &= \sqrt{(\rho\sigma ui - b_j)^2 - \sigma^2(2u_jui - u^2)},\\
    u_1 &= \frac{1}{2}, \quad u_2 = -\frac{1}{2}, \quad a = \kappa\theta, \quad b_1 = \kappa + \lambda - \rho\sigma, \quad b_2 = \kappa + \lambda.
\end{align*}
However, using the characteristic function the price of the European call option under
$\tilde{\QQ}$ in the Heston model corresponding to $\lambda$ using Carr-Madan's
Fourier transform formula is
given on \cite{Havrylenko2024}, p. 77 by:
\begin{align*}
    Call^{\text{CarrMadan}}(k) = \frac{e^{-\alpha k}}{\pi} \mathfrak{Re} \left( \int_{0}^{+\infty} \frac{e^{-iu k} e^{-rT} \Psi_{x(T)}\left(u - (1 + \alpha)i\right)}{\alpha^{2} + \alpha - u^{2} + i(1 + 2\alpha)u} \, du \right).
\end{align*}
In other words, only one integration scheme is required in opposition to two in
Heston's original formula (2.3). Note from \cite{Havrylenko2024}, pp. 56-57 we
see that $Q_2$ is related to the measure $\tilde{\QQ}$ whereas for $Q_1$ we change
measure to $\QQ^S$. i.e we have that the price of a European call option can be
writtein as (by change of numeraire):
\begin{align*}
    Call(t)&=e^{-r\tau} \EE^{\tilde{\QQ}}\lc S_T 1_{S_T > K} \rc - K e^{-r\tau} \mathbb{E}^{\tilde{\QQ}} \lc 1_{S_T > K} \rc
    \\&=S_t\QQ^S(S_T>K)-Ke^{-r\tau}\tilde{\QQ}(S_T>K)\\
    &=S_tQ_1-Ke^{-r\tau}Q_2
\end{align*}
The measure $\tilde{\QQ}$ uses the bond $B_t$ as the numeraire, while the measure $\QQ^S$ uses the stock price $S$.
Resultingly, $Q_2$ is the correct function as it relatex to $\tilde{\QQ}$ and thus $\Psi_2$
the correct characteristic function.
Consequently, when implementing the \texttt{characteristicFunctionHeston} into
the evaluation of the integral in the Carr-Madan Fourier transform pricing
formula for the European call option, we restrict \texttt{j=2} in the function parameter and the
horizontal shift for the characteristic function argument to  \texttt{(u-(1+alpha)i)} as opposed to \texttt{u}.

Note that in theory we do not need to consider the real part of the integrand
but as we need to truncate the integral for numerical evaluation the need arises as there will be an
imaginary part. 

To evaluate the integral numerically we employ the \texttt{quad}-function in
\texttt{Python}, again (see subsection 2.4 for the same description and branch
cuts of the imaginary part of the $\log$-function).

Pricing of the call option is then calculated using the parameters seen in Table 8.
\begin{table}[!h]
    \centering
    \begin{tabular}{lll}
    \toprule
    \textbf{Parameter} & \textbf{Symbol} & \textbf{Value} \\
    \midrule
    Initial Stock Price & $S_0$ & 100 \\
    Initial Variance & $v_0$ & 0.06 \\
    Risk-free Interest Rate & $r$ & 0.05 \\
    Speed of Mean Reversion & $\kappa$ & 1 \\
    Long-term Mean Variance & $\theta$ & 0.06 \\
    Volatility of the Volatility & $\sigma$ & 0.3 \\
    Correlation Coefficient & $\rho$ & -0.5 \\
    Market Price of Risk & $\lambda$ & 0.01 \\
    Time to Expiry & $\tau$ & 1 \\
    Strike Price & $K$ & 100 \\
    Damping Factor & $\alpha$ & 0.3 \\
    \bottomrule
    \end{tabular}
    \caption{Model parameters for the Heston's model European call option pricing using Carr-Madan's Fourier transfom approach}
    \label{table:call_option_pricing_parameters4}
\end{table}

The results can be seen in Table 9  with different increasing upper bounds and
specifically $u=50$ as requested in the HandIn.
\begin{table}[ht]
    \centering
    \begin{tabular}{|l|c|c|c|c|}
    \hline
    \multicolumn{5}{|c|}{\textbf{Carr-Madan's Fourier Transform Formula}} \\ \hline
    \textbf{} & \textbf{Bounds $[0,10]$} & \textbf{Bounds $[0,50]$} & \textbf{Bounds $[0,100]$} & \textbf{Bounds $[0,1000]$} \\ \hline
    \textbf{Option Price} & 11.886 & 11.936 & 11.936 & 11.936 \\
    \textbf{Computing Time} & $3.9701 \times 10^{-3}$ & $6.0089 \times 10^{-3}$ &
    $5.985 \times 10^{-3}$
    & $8.9772 \times 10^{-3}$ \\ \hline
    \end{tabular}
    \caption{Carr-Madan's Fourier transform results using numerical integration with different (truncated) upper bounds: $[0,10]$, $[0,50]$, $[0,100]$  and $[0,1000]$.}
    \label{table:results4}
\end{table}

Firstly, observe that the results of (2.4) are almost identical as thoes of
(2.3). We will however comment on why this might not be a general result when we
increase the number of call option calculations.

If we were to price a large number of options we would expect the
Carr-madan method (2.4) to be faster in relative terms in computing time
compared to that of (2.3) and especially thoes of EulerMC (2.1) and MilsteinMC
(2.2). To see why consider both call option pricing formulas for (2.3):
\begin{align*}
    Call^{\text{OriginalFT}}(t)=S_tQ_1(\log(S(t)),v(t),t;ln(K))-Ke^{-r\tau}Q_2(\log(S(t)),v(t),t;\log(K)),
\end{align*}
and for 2.4:
\begin{align*}
    Call^{\text{CarrMadan}}(k) = \frac{e^{-\alpha k}}{\pi} \mathfrak{Re} \left( \int_{0}^{+\infty} \frac{e^{-iu k} e^{-rT} \Psi_{x(T)}\left(u - (1 + \alpha)i\right)}{\alpha^{2} + \alpha - u^{2} + i(1 + 2\alpha)u} \, du \right).
\end{align*}
Notice that the denominator of the Carr-Madan formula (2.3) decays the integral
relatively faster compared to that of Heston's original formula (2.4). This is
because the integrand enters in the denominator squared in Carr-Madan formula (2.3)
where as it enters linearly in the denominator in Heston's original formula
(2.4). Another obvious advantageous is the fact that we have one integration
scheme in Carr-Madan formula (2.3) as opposed to two in Heston's original formula
(2.4).

The key difference between Carr-Madan's formula and the methods EulerMC (2.1)
and MilsteinMC (2.2)  is the computation time, with Carr-Madan showing up to a
$100$ times faster computation compared to (2.1) and (2.2). Despite the expected
increase in computation time with wider integration bounds, similar to the
increase in paths for EulerMC (2.1) and Carr-Madan's (2.2) precision remains high for small
upper bounds without significantly affecting option prices up to five decimal
places. Increasing the upper bound theoretically accelerates the integral's
decay for the reasons mentioned before compared to Heston's original formula (2.3).

It should again be noted as it is not negliable whatsoever that the computation time is highly
reliant on the state of your computer at that given moment and the drag on your CPU/GPU.

\newpage
\section{Asset allocation under the Heston model}
\subsection{Derivation of the Hamilton-Jacobi-Bellman PD}
\subsection{Solution to the Hamilton-Jacobi-Bellman PDE}
\subsection{Interpretation of the results}


\newpage
\appendix
\section{Appendix: Code}\label{chap:code}
\href{https://github.com/YoussefRaad-mathecon/Handin-3}{GitHub profile
with the code in \texttt{Python} for Option pricing in the context of the Heston
model and in R for asset allocation under the Heston model. We frequently refer
to and use the parameter $"\lambda"$. It's crucial to understand that this
"\texttt{lambda}" parameter should not be confused with \texttt{Python}'s
built-in \texttt{lambda}-function. \texttt{Python}'s \texttt{lambda} function
are a feature that allows for creating small, anonymous functions at runtime.
They are defined using the \texttt{lambda} keyword, followed by a list of
arguments etc.}

\newpage
\bibliographystyle{abbrvnat}

\bibliography{mybib.bib}
\end{document}