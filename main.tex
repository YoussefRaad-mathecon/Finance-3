\documentclass[11pt]{article}
\usepackage[utf8]{inputenc}
\usepackage[margin=1in]{geometry}
\usepackage[titletoc,title]{appendix}
\usepackage{tabularx}
\usepackage{fancyhdr}
\usepackage{array}
\usepackage{booktabs}
\usepackage{hyperref}
\usepackage{xcolor}
\usepackage{listings}
\usepackage{amsmath,amssymb,amsfonts,mathrsfs,latexsym}
\usepackage{cleveref}
\numberwithin{equation}{section}
\usepackage{graphicx} 
\usepackage{enumerate}
\usepackage{caption}
\usepackage{mathtools}
\usepackage{mdframed}
\usepackage{amsthm}
\DeclareUnicodeCharacter{2212}{-}
\usepackage{changepage}
\usepackage{mdframed}
\usepackage{enumitem}
\usepackage[utf8]{inputenc}
\usepackage[parfill]{parskip}
\usepackage[english]{babel}
\usepackage[nottoc,numbib]{tocbibind}
\usepackage{lastpage} 
\usepackage{float}
\usepackage[bb=boondox]{mathalfa}
\usepackage{graphicx}
\usepackage[round]{natbib}
\usepackage{amsmath,amsfonts,amssymb,mathtools}

\usepackage{graphicx,float}
\renewcommand{\thetable}{\textbf{\arabic{table}}}
\usepackage{pgffor}
\usepackage[ruled,vlined]{algorithm2e}
\usepackage{algorithmic}
\setlength{\parindent}{20pt}

\def\s#1{\mathscr{#1}}
\def\c#1{\mathcal{#1}}
\def\f#1{\mathfrak{#1}}
\foreach \x in {A,...,Z}{
  \expandafter\xdef\csname \x\x\endcsname{\noexpand\mathbb{\x}}
}
\newcommand{\lp}{\left(}
\newcommand{\rp}{\right)}

\newcommand{\lc}{\left\{}
\newcommand{\rc}{\right\}}

\newcommand{\lb}{\left\[}

\newcommand{\rb}{\right\]}

\pagestyle{fancy}
\begin{document}
\begin{titlepage}
    \begin{center}
        \vspace*{1cm}
        
        
        \Huge
        \textbf{Hand-In \#3}
        
        \vspace{0.5cm}
        \LARGE
        Continuous Time Finance 2 (FinKont2)
        
        \vspace{0.5cm}
        
        Youssef Raad (zfw568)
    
        
        \vspace{0.5cm}
         April  14, 2024
        
    \end{center}
    \vspace{2.5cm}
    \begin{center}
        \includegraphics[scale=0.50]{ku_segl.png}
    \end{center}
    \vspace{5.5cm}
    \begin{center}
        \textit{\href{https://www.youtube.com/watch?v=jXD98S28qT8}{"Dying For The Right Cause. It's The Most Human Thing We Can Do."}}
    \end{center}
\end{titlepage}
\captionsetup[figure]{labelfont=bf} \captionsetup[table]{labelfont=bf}
\fancyhead[L]{Youssef Raad (zfw568) \\ Continuous Time Finance 2
\\ Hand-In \#3}
\fancyhead[R]{ }
\setlength{\headheight}{40pt}
\newpage
\tableofcontents
\cfoot{\thepage\ / \pageref{LastPage}}

\begin{center}
    $\dagger$\textbf{Note}$\dagger$
\end{center}
\begin{center}
I have completed the handin assignment as per the instructions provided in the
assignment description. However, I encountered an issue where certain questions
appear to be missing:
\begin{itemize}
    \item Showing respective conditions for $E_j$ is \textbf{left in} as it is removed in the
    latest version but no notice was given (Subsection 1.2).
    \item Showing the solutions of ODE's equation (29) and (30) are given by equation (31)
    and (32), respectively, are \textbf{left out} since notice was given (Subsection 3.2).
\end{itemize}

\end{center}

\newpage
\section{Characteristic functions in the Heston model}
The single digit referencing to equation (x) is for equations in the HandIn \#3
description. If equations are refered within this document we will use a double
digit referencing (x.y) where x is the section number and y the equation number. 
\subsection{Changing variables and guessing the form of solution}
$\textcolor{blue}{\star}$ Let $\Psi_j:=\Psi_{x(T)}^j(x,v,t;u)$, $j=1,2$ and $\tau=T-t$. Let the
reparametrization of $\Psi_j$ through $\tau=T-t$ be given by
$\Psi_{x(T)}^j(x,v,\tau;u)$. The transformations of terms in equation (4) is then given
by noticing only the first term depends on $\tau$:
\begin{align*}
    \frac{\partial }{\partial t}\Psi_{x(T)}^j(x,v,t;u)&=-\frac{\partial }{\partial \tau}\Psi_{x(T)}^j(x,v,\tau;u),\\
    \frac{\partial }{\partial x}\Psi_{x(T)}^j(x,v,t;u)&=\frac{\partial }{\partial x}\Psi_{x(T)}^j(x,v,\tau;u),\\
    \frac{\partial^2 }{\partial x^2}\Psi_{x(T)}^j(x,v,t;u)&=\frac{\partial ^2}{\partial x^2}\Psi_{x(T)}^j(x,v,\tau;u),\\
    \frac{\partial }{\partial v}\Psi_{x(T)}^j(x,v,t;u)&=\frac{\partial }{\partial v}\Psi_{x(T)}^j(x,v,\tau;u),\\
    \frac{\partial^2 }{\partial v^2}\Psi_{x(T)}^j(x,v,t;u)&=\frac{\partial^2 }{\partial v^2}\Psi_{x(T)}^j(x,v,\tau;u),\\
    \frac{\partial^2 }{\partial v \partial x}\Psi_{x(T)}^j(x,v,t;u)&=\frac{\partial^2 }{\partial v \partial x}\Psi_{x(T)}^j(x,v,\tau;u).
\end{align*}
As the only transformation that differs from equation (4) through the
reparametrization is the first, namely:
\begin{align*}
    \frac{\partial }{\partial t}\Psi_j=-\frac{\partial }{\partial
\tau}\Psi_{x(T)}^j(x,v,\tau;u),
\end{align*}
we achieve equation (6) from substitution of the
differing term into equation (4) using the shorhand notation:
\begin{align*}
    0&=-\frac{\partial \Psi_j}{\partial \tau}+(r+u_jv)\frac{\partial \Psi_j}{\partial x}+
    \frac{1}{2}v\frac{\partial^2 \Psi_j}{\partial x^2}+\rho\sigma v \frac{\partial^2 \Psi_j}{\partial v \partial x}+
    \frac{1}{2}\sigma^2v \frac{\partial^2 \Psi_j}{\partial v^2}+(a-b_j v) \frac{\partial \Psi_j}{\partial v},
\end{align*}
as desired.

The terminal condition follows simply from observering that at $t=T$
we have $\tau=T-t=T-T=0$. Hence, through the reparametrization using $\tau$ we
achieve equation (7):
\begin{align*}
    e^{iux}=\Psi_{x(T)}^j(x,v,0;u) \quad \forall (x, v) \in \mathbb{R} \times (0, +\infty),
\end{align*}
as desired.

\newpage
$\textcolor{blue}{\star}$ Equation (8) is given by:
\begin{align*}
    \Psi_{x(T)}^j(x,v,t;u)=\exp \lc C_j (\tau; u) + D_j (\tau; u)v + i  u x \rc.
\end{align*}
We substitute in the ansatz (the guessed solution) (8) into equation (6) and equation (7) using the shorthands
$C_j (\tau; u):=C_j$ and $D_j (\tau; u)=D_j$ yielding the PDE:
\begin{align*}
    0&=-\frac{\partial \Psi_j}{\partial \tau}+(r+u_jv)\frac{\partial \Psi_j}{\partial x}+
    \frac{1}{2}v\frac{\partial^2 \Psi_j}{\partial x^2}+\rho\sigma v \frac{\partial^2 \Psi_j}{\partial v \partial x}+
    \frac{1}{2}\sigma^2v \frac{\partial^2 \Psi_j}{\partial v^2}+(a-b_j v) \frac{\partial \Psi_j}{\partial v}\\
    &\overset{\dagger}{=}-\left( \frac{\partial  C_j}{\partial \tau}+ \frac{\partial D_j}{\partial \tau}v\right)\Psi_j +(r+u_jv)(iu) \Psi_j\\&\quad+ \frac{1}{2}(iu)^2\Psi_j+\rho\sigma v(D_j)(iu)\Psi_j+
    \frac{1}{2}\sigma^2v (D_j)^2\Psi_j+(a-b_j v)( D_j)\Psi_j\\
    &\overset{\dagger\dagger}{=} \Psi_j \lc \underbrace{ -\frac{\partial C_j}{\partial \tau} +aD_j+rui}_{(*)} \rc + \Psi_j v \lc\underbrace{ -\frac{\partial D_j}{\partial \tau}+u_jui-\frac{1}{2}u^2+\rho\sigma u i D_j+\frac{1}{2}\sigma^2 D_j^2 - b_jD_j}_{(**)} \rc,
\end{align*}
where $\dagger$ follows from differentiation and
$\dagger \dagger$ from the fact that $i^2= (\sqrt{-1})^2=-1$ and thus
$(iu)^2=(-1 )\cdot u^2=-u^2$.
\subsection{Transforming the PDE to a system of ODEs and solving it}
$\textcolor{blue}{\star}$ The PDE that we derived in Subsection 1.1 is satisfied if and only if both
ordinary differential equations $(*)$ and $(**)$ equal $0$ for all $(x,v)\in \RR
\times (0,\infty)$. Consequently, the
two ordinary differential equations from $(*)$ and $(**)$ can be solved
seperately. i.e a system of two ODE's. The system of two ordinary differential equations is then found by dropping
$\Psi_j$-terms, rewriting and thus given by $(*)$:
\begin{align*}
    0&=-\frac{\partial C_j}{\partial \tau}+aD_j+rui\\&\iff \\ \frac{\partial C_j}{\partial \tau}&=aD_j+rui,
\end{align*}
and $(**)$:
\begin{align*}
    0&=-\frac{\partial D_j}{\partial \tau}+u_jui-\frac{1}{2}u^2+\rho\sigma u iD_j+\frac{1}{2}\sigma^2D_j^2-b_jD_j\\ &\iff \\ \frac{\partial D_j}{\partial \tau}&=u_jui-\frac{1}{2}u^2+\rho\sigma u iD_j+\frac{1}{2}\sigma^2D_j^2-b_jD_j.
\end{align*}
At $\tau=0$, the initial conditions are
zero-initial conditions, i.e such that $D_j(0;u)=0$ and $C_j(0;u) =0$. This is
because when the expiry is reached ($\tau = 0$), the value of $x_T = \log S_T$ is
established, eliminating the expectation in the equality:
\begin{align*}
    \Psi_j&:=\EE^{\PP_j} \left [ e^{iux}  \mid x(t)=x,v(t)=v \right]\\
    &=\exp \lc C_j (\tau; u) + D_j (\tau; u) \cdot v + i  u x \rc \quad j=1,2.
\end{align*}
As a result, the second equality has to simplifiy to $e^{iu x}$ as the
expectation vanishes at $\tau=0$:
\begin{align*}
    \Psi_{x_{(T)}}^j(x,v,0;u)&=e^{iux}\\
    &\Rightarrow\\
    \exp \lc C_j (0; u) + D_j (0; u) \cdot v + i  u x \rc&=e^{iux}\\
    &\iff\\
    C_j (0; u) + D_j (0; u) \cdot v &=0
\end{align*}
This leads to the conclusion that the sufficient initial conditions at $\tau=0$ are $D_j (0,u) = 0$
and $C_j (0,u) = 0$. The equation denoted by (**) outlines a Riccati equation
in $D_j$ that, importantly, does not depend on $C_j$, while the
equation marked by (*) represents an ordinary differential equation for $C_j$.
This ordinary differential equation can be readily solved through direct
integration of $D_j$ after determining $D_j$.

$\textcolor{blue}{\star}$ We now proceed to solving the Riccati equation as of
the beforementioned reasons, i.e that it contains terms involving only the
functions $D_j$ and not $C_j$. We have that:
\begin{align*}
    \frac{\partial D_j}{\partial \tau}&=u_jui-\frac{1}{2}u^2+\rho\sigma u iD_j+\frac{1}{2}\sigma^2D_j^2-b_jD_j\\
    &=u_jui-\frac{1}{2}u^2-(b_j-\rho\sigma u i)D_j+\frac{\sigma^2}{2}D_j^2\\
    &=\alpha-\beta D_j+\frac{1}{2}\sigma^2D_j^2,
\end{align*}
with:
\begin{align*}
    \alpha = u_j ui - \frac{1}{2}u^2 \quad \text{and} \quad \beta=b_j-\rho \sigma u i.
\end{align*}
The solution given in (9) can then be written exactly as:
\begin{align*}
    D_j(\tau;u)=-\frac{\frac{\partial E_j(\tau;u)}{\partial \tau}}{\frac{1}{2}\sigma^2E_j(\tau;u)}.
\end{align*}
If we now differentiate the RHS of (9) wrt. $\tau$ we achieve:
\begin{align*}
        \frac{\partial D_j}{\partial \tau}&=-\frac{\frac{\partial^2 E_j}{\partial\tau^2 }E_j\frac{1}{2}\sigma^2- \left ( \frac{\partial E_j}{\partial \tau}\right )^2 \frac{1}{2}\sigma^2}{\left ( \frac{1}{2}\sigma^2 E_j\right )^2}\\
        &=\frac{\left ( \frac{\partial E_j}{\partial \tau}\right )^2-\frac{\partial^2 E_j}{\partial\tau^2 }E_j}{\frac{1}{2}\sigma^2 E_j^2}.
\end{align*}
Substituting (9) and its just found derivatives into the Ricatti equation
yields:
\begin{align*}
    \frac{\partial D_j}{\partial \tau}=\alpha-\beta D_j + \frac{1}{2}\sigma^2D_j^2\\
    &\iff\\
    \frac{\left ( \frac{\partial E_j}{\partial \tau}\right )^2-\frac{\partial^2 E_j}{\partial\tau^2 }E_j}{\frac{1}{2}\sigma^2 E_j^2}&=\alpha-\beta\frac{-\frac{\partial E_jD}{\partial \tau}}{\frac{1}{2}\sigma^2 E_j}+\frac{1}{2}\sigma^2 \left ( \frac{-\frac{\partial E_j}{\partial \tau}}{\frac{1}{2}\sigma^2 E_j}\right )^2\\
    &\iff\\
    \frac{\left ( \frac{\partial E_j}{\partial \tau}\right )^2-\frac{\partial^2 E_j}{\partial\tau^2 }E_j}{\frac{1}{2}\sigma^2 E_j^2}&=\alpha+\beta\frac{\frac{\partial E_jD}{\partial \tau}}{\frac{1}{2}\sigma^2 E_j}+ \frac{\left ( \frac{\partial E_j}{\partial \tau}\right ) ^2}{\frac{1}{2}\sigma^2 E_j}\\
    &\iff\\
    \left ( \frac{\partial E_j}{\partial \tau} \right )^2 - \frac{\partial^2 E_j}{\partial \tau^2} &= \alpha \frac{1}{2} \sigma^2 E_j^2 + \beta \frac{ \frac{\partial E_j}{\partial \tau} }{\frac{1}{2}\sigma^2E_j} \frac{1}{2}\sigma^2 E_j^2+ \frac{\left ( \frac{\partial E_j}{\partial \tau}\right ) ^2}{\frac{1}{2}\sigma^2 E_j}\frac{1}{2}\sigma^2 E_j^2\\
    &\iff\\
    \left ( \frac{\partial E_j}{\partial \tau} \right )^2 - \frac{\partial^2 E_j}{\partial \tau^2} &= \alpha \frac{1}{2} \sigma^2 E_j^2 + \beta \frac{\partial E_j}{\partial \tau}E_j+\left ( \frac{\partial E_j }{\partial \tau} \right )^2\\
    &\iff\\
    - \frac{\partial^2 E_j}{\partial \tau^2}E_j &= \alpha \frac{1}{2} \sigma^2 E_j^2 + \beta \frac{\partial E_j}{\partial \tau}E_j\\
    &\iff\\
    - \frac{\partial^2 E_j}{\partial \tau^2} &= \alpha \frac{1}{2} \sigma^2 E_j + \beta \frac{\partial E_j}{\partial \tau}\\
    &\iff\\
    \frac{\partial^2 E_j}{\partial \tau^2}+\alpha \frac{1}{2} \sigma^2 E_j + \beta \frac{\partial E_j}{\partial \tau}&=0\\
    &\overset{\dagger}{\iff}\\
    \frac{\partial^2 E_j}{\partial \tau^2} - (\rho u \sigma i - b_j) \frac{\partial E_j}{\partial \tau} + \frac{\sigma^2}{2} \left( -\frac{1}{2}u^2 + u_j ui \right) E_j &= 0,
\end{align*}
where $\dagger$ follows from $\alpha = u_j ui - \frac{1}{2}u^2$ and
$\beta=b_j-\rho \sigma u i$ and rearranging.

$\textcolor{blue}{\star}$  We showed the initial sufficient condition for $\tau=0$ was
specified as
$D_j(0;u)=0$. We exploit this to show that the respective sufficient conditions for $E_j$ are
given by:
\begin{align*}
    D_j(0;u)=-\frac{\left. \frac{\partial E_j(\tau;u)}{\partial \tau}\right|_{\tau=0}}{\frac{1}{2}\sigma^2E_j(0;u)}&=0\\
    &\iff\\
    \left. \frac{\partial E_j(\tau;u)}{\partial \tau}\right|_{\tau=0}&=0.
\end{align*}
Using this result and that we just showed the resulting ODE for $E_j$ we now have that:
\begin{align*}
    \left ( \left. \frac{\partial^2 E_j(\tau;u)}{\partial \tau^2}\right|_{\tau=0} \right )-(\rho u \sigma i-b_j)\underbrace{\left (\left. \frac{\partial E_j(\tau;u)}{\partial \tau}\right|_{\tau=0} \right )}_{= 0} + \frac{\sigma^2}{2} \left( -\frac{1}{2}u^2 + u_j ui \right) E_j(0;u)&= 0\\
    &\iff\\
    \left ( \left. \frac{\partial^2 E_j(\tau;u)}{\partial \tau^2}\right|_{\tau=0} \right ) + \frac{\sigma^2}{2} \left( -\frac{1}{2}u^2 + u_j ui \right) E_j(0;u) &= 0\\
    &\iff\\
    -\frac{2}{\sigma^2}\frac{\left.\frac{\partial^2 E_j(\tau;u)}{\partial \tau^2}\right|_{\tau=0}}{\left( -\frac{1}{2}u^2 + u_j ui \right)}&=E_j(0;u)\\
    &\neq 0,
\end{align*}
yielding the respective sufficient condition for $E_j$ at $\tau=0$.

$\textcolor{blue}{\star}$ For $\tau=0$, we achieve the following form of
equation (10):
\begin{align*}
    E_j(0;u)&=A_je^{x_{j,+}\cdot 0}+B_je^{x_{j,-}\cdot 0}\\&
    =A_j\cdot 1+B_j \cdot 1\\
    &=A_j + B_j,
\end{align*}
showing the first property.

We now find the derivative of equation (10) to find the initial condition of
said derivative:
\begin{align*}
    \frac{\partial E_j(\tau;u)}{\partial \tau}=x_{j,+}A_je^{x_{j,+}\cdot\tau}+x_{j,-}B_je^{x_{j,-}\cdot\tau}.
\end{align*}
This now yields the sufficient initial condition of the derivative of $E_j$:
\begin{align*}
    \left. \frac{\partial E_j(\tau;u)}{\partial \tau} \right|_{\tau=0}&=x_{j,+}A_je^{x_{j,+}\cdot 0}+x_{j,-}B_j e^{x_{j,-}\cdot 0}\\
    &=x_{j,+} \cdot 1 \cdot A_j+x_{j,-} \cdot 1 \cdot B_j \\
    &=x_{j,+} A_j+x_{j,-} B_j.
\end{align*}

$\textcolor{blue}{\star}$ Once again, we use the sufficient initial conditions for $D_j$,
namely, $D_j(0;u)=0$, for $\tau=0$. For equation (9) to hold, it must then be the case
that:
\begin{align*}
    \left. \frac{\partial E_j(\tau;u)}{\partial \tau} \right|_{\tau=0}=0.
\end{align*}
This gives us the bi-implications:
\begin{align*}
    \left. \frac{\partial E_j(\tau;u)}{\partial \tau} \right|_{\tau=0}&=0\\
    &\:\iff\\
    x_{j,+} A_j+x_{j,-} B_j&=0\\
    \:&\:\overset{\dagger}{\iff}\\
    -A_j=\frac{x_{j,-}}{x_{j,+}}B_j=g_jB_j &\quad \text{and}\quad B_j=\frac{x_{j,+}}{x_{j,-}}A_j=-g_j^{-1}A_j,
\end{align*}
where $\dagger$ follows from definition, namely, $g_j=\frac{x_{j,-}}{x_{j,+}}$.
Using our just showed result, $E_j(0;u)=A_j+B_j$, we find that:
\begin{align*}
    -A_j=g_jB_j&=g_j(E_j(0;u)-A_j)\\
    &\iff\\
    A_j(g_j-1)&=g_jE_j(0;u)\\
    &\iff\\
    A_j&=\frac{g_jE_j(0;u)}{g_j-1},
\end{align*}
implying by the found relations of $A_j,B_j$:
\begin{align*}
    B_j=-g_j^{-1}A_j=-g_j^{-1}\frac{g_jE_j(0;u)}{g_j-1}=-\frac{E_j(0;u)}{g_j-1}.
\end{align*}

$\textcolor{blue}{\star}$ It now follows quite simply and nicely from
substitution of the
previous results that:
\begin{align*}
    E_j(\tau;u)&=A_je^{x_{j,+}\cdot \tau}+B_j e^{x_{j,-}\cdot \tau}\\
    &=\frac{g_jE_j(0;u)}{g_j-1}e^{x_{j,+}\cdot \tau}-\frac{E_j(0;u)}{g_j-1}e^{x_{j,-}\cdot \tau}\\
    &=\frac{E_j(0;u)}{g_j-1}\left ( g_je^{x_{j,+}\cdot \tau}-e^{x_{j,-}\cdot \tau}\right ).
\end{align*}

$\textcolor{blue}{\star}$ We use the found relations for $A_j,B_j$ to find: 
\begin{align*}
    \frac{\partial E_j(\tau;u)}{\partial \tau}&=x_{j,+}A_je^{x_{j,+}\cdot \tau}+x_{j,-}B_je^{x_{j,-}\cdot\tau}\\
    &=x_{j,+}\frac{g_jE_j(0;u)}{g_j-1}e^{x_{j,+}\tau}-x_{j,-}\frac{E_j(0;u)}{g_j-1}e^{x_{j,-}\cdot\tau}\\
    &=\frac{E_j(0;u)}{g_j-1}\left (x_{j,+}g_je^{x_{j,+}\cdot \tau} -x_{j,-}e^{x_{j,-}\cdot \tau}\right ).
\end{align*}

$\textcolor{blue}{\star}$ Substituting $E_j$ and $\partial E_j/\partial
\tau$ into equation (9) for $D_j$
now yields the altered expression:
\begin{align*}
    D_j(\tau;u)&=-\frac{\frac{\partial E_j(\tau;u)}{\partial \tau}}{\frac{1}{2}\sigma^2E_j(\tau;u)}\\
    &=-\frac{\frac{E_j(0;u)}{g_j-1}\left (x_{j,+}g_je^{x_{j,+}\cdot \tau} -x_{j,-}e^{x_{j,-}\cdot \tau}\right )}{\frac{1}{2}\sigma^2 \frac{E_j(0;u)}{g_j-1}\left ( g_je^{x_{j,+}\cdot \tau}-e^{x_{j,-}\cdot \tau}\right )}
\end{align*}
\newpage
Now, substituting in equation (11) and (15) into the altered equation (9) using
$\exp\lc\rc$ for readability yields:
\begin{align*}
    D_j(\tau ; u)&=-\frac{\frac{E_j(0 ; u)}{g_j-1}\left(x_{j,+} g_j \exp\left\{x_{j,+} \cdot \tau\right\}-x_{j,-} \exp\left\{x_{j,-} \cdot \tau\right\}\right)}{\frac{\sigma^2}{2} \frac{E_j(0;u)}{g_j-1}\left(g_j \exp\left\{x_{j,+} \cdot \tau\right\}-\exp\left\{x_{j,-} \cdot \tau\right\}\right)} \\
    & =-\frac{\frac{E_j(0 ; u)}{g_j-1}\left(\frac{\rho \sigma u i-b_j+d_j}{2} g_j \exp\left\{\frac{\rho \sigma u i-b_j+d_j}{2} \tau\right\}-\frac{\rho \sigma u i-b_j-d_j}{2} \exp\left\{\frac{\rho \sigma u i-b_j-d_j}{2} \tau\right\}\right)}{\frac{\sigma^2}{2} \frac{E_j(0 ; u)}{g_j-1}\left(g_j \exp\left\{\frac{\rho \sigma u i-b_j+d_j}{2} \tau\right\}-\exp\left\{\frac{\rho \sigma u i-b_j-d_j}{2} \tau\right\}\right)} \\
    & =-\frac{\frac{\rho \sigma u i-b_j+d_j}{2} g_j \exp\left\{\frac{\rho \sigma u i-b_j+d_j}{2} \tau\right\}-\frac{\rho \sigma u i-b_j-d_j}{2} \exp\left\{\frac{\rho \sigma u i-b_j-d_j}{2} \tau\right\}}{\frac{\sigma^2}{2}\left(g_j \exp\left\{\frac{\rho \sigma u i-b_j+d_j}{2} \tau\right\}-\exp\left\{\frac{\rho \sigma u i-b_j-d_j}{2} \tau\right\}\right)} \\
    & =-\frac{\frac{\rho \sigma u i-b_j+d_j}{2} g_j \exp\left\{\frac{d_j}{2} \tau\right\}-\frac{\rho \sigma u i-b_j-d_j}{2} \exp\left\{-\frac{d_j}{2} \tau\right\}}{\frac{\sigma^2}{2}\left(g_j \exp\left\{\frac{d_j}{2} \tau\right\}-\exp\left\{-\frac{d_j}{2} \tau\right\}\right)} \\
    & =-\frac{\left(\rho \sigma u i-b_j+d_j\right) g_j \exp\left\{\frac{d_j}{2} \tau\right\}-\left(\rho \sigma u i-b_j-d_j\right) \exp\left\{-\frac{d_j}{2} \tau\right\}}{\sigma^2\left(g_j \exp\left\{\frac{d_j}{2} \tau\right\}-\exp\left\{-\frac{d_j}{2} \tau\right\}\right)} \\
    & \overset{\dagger}{=}-\frac{\left(\rho \sigma u i-b_j+d_j\right) g_j \exp\left\{\frac{d_j}{2} \tau\right\} \exp\left\{\frac{d_j}{2} \tau\right\}-\left(\rho \sigma u i-b_j-d_j\right) \exp\left\{-\frac{d_j}{2} \tau\right\} \exp\left\{\frac{d_j}{2} \tau\right\}}{\sigma^2\left(g_j \exp\left\{\frac{d_j}{2} \tau\right\} \exp\left\{\frac{d_j}{2} \tau\right\}-\exp\left\{-\frac{d_j}{2} \tau\right\} \exp\left\{\frac{d_j}{2} \tau\right\}\right)} \\
    & =-\frac{\left(\rho \sigma u i-b_j+d_j\right) g_j \exp\left\{d_j \tau\right\}-\left(\rho \sigma u i-b_j-d_j\right)}{\sigma^2\left(g_j \exp\left\{d_j \tau\right\}-1\right)} \\
    & \overset{\dagger\dagger}{=}-\frac{\left(\rho \sigma u i-b_j+d_j\right) \frac{\rho \sigma u i-b_j-d_j}{\rho \sigma u i-b_j+d_j} \exp\left\{d_j \tau\right\}-\left(\rho \sigma u i-b_j-d_j\right)}{\sigma^2\left(g_j \exp\left\{d_j \tau\right\}-1\right)} \\
    & =-\frac{\left(\rho \sigma u i-b_j-d_j\right) \exp\left\{d_j \tau\right\}-\left(\rho \sigma u i-b_j-d_j\right)}{\sigma^2\left(g_j \exp\left\{d_j \tau\right\}-1\right)} \\
    & \overset{\dagger\dagger\dagger}{=}-\frac{\left(b_j-\rho \sigma u i+d_j\right) \exp\left\{d_j \tau\right\}-\left(b_j-\rho \sigma u i+d_j\right)}{\sigma^2\left(1-g_j \exp\left\{d_j \tau\right\}\right)} \\
    & =\frac{\left(b_j-\rho \sigma u i+d_j\right)\left(1-\exp\left\{d_j \tau\right\}\right)}{\sigma^2\left(1-g_j \exp\left\{d_j \tau\right\}\right)} \\
    & =\frac{b_j-\rho \sigma u i + d_j}{\sigma^2} \frac{1-\exp\left\{d_j \tau\right\}}{1-g_j \exp\left\{d_j \tau\right\}},
\end{align*}
where $\dagger$ follows by multiplication of $\exp{\left \{\frac{d_j}{2}\tau
\right \}}$, $\dagger\dagger$ from $g:=x_{j,-}/x_{j,+}$ and
$\dagger\dagger\dagger$ by multiplication of $-1$.
\newpage
$\textcolor{blue}{\star}$  As stated at the very start, finding the solution of
$D_j$ is advantageous to derive the solution for $C_j$ as the ODE for $C_j$ was
given by $(*)$ containing $D_j$. So, by integration:
\begin{align*}
    \frac{\partial C_j}{\partial \tau}&=rui+aD_j\\
    &\iff\\
    C_j&=\int_0^\tau rui+aD_j(s;u)\:ds\\
    &\overset{\dagger}{=}rui\int_0^\tau \:ds+a\int_0^\tau D_j(s;u)\:ds\\
    &=rui \left ( \tau-0 \right )+a\int_0^\tau D_j(s;u)\:ds\\
    &=rui\tau+a\int_0^\tau D_j(s;u)\:ds.
\end{align*}
where $\dagger$ follows from the linearity of the integral. As wished, solving the ODE of $D_j$ first, we now only have to find the integral
involving $D_j$ to find the solution of $C_j$. We proceed with finding the integral involving $D_j$:
\begin{align*}
    \int_0^\tau D_j(s;u)\:ds&\overset{\dagger}{=}\int_0^\tau \frac{b_j-\rho \sigma u i + d_j}{\sigma^2} \frac{1-\exp\left\{d_j \tau\right\}}{1-g_j \exp\left\{d_j \tau\right\}} \:ds\\
    &= \frac{b_j-\rho \sigma u i + d_j}{\sigma^2}\cdot \int_0^\tau\frac{1-\exp\left\{d_j \tau\right\}}{1-g_j \exp\left\{d_j \tau\right\}} \:ds\\
    &\overset{\dagger\dagger}{=}\frac{b_j-\rho \sigma u i+d j}{\sigma^2} \frac{1}{d_j} \cdot\int_{1}^{\exp\left\{d_j \tau\right\}} \frac{1-y}{1-g_j y} \frac{1}{y} \:dy \\
    & =\frac{b_j-\rho \sigma u i+d j}{\sigma^2} \frac{1}{d_j} \cdot\int_{1}^{\exp\left\{d_j \tau\right\}} \frac{1-y}{y\left(1-g_j y\right)} \:dy \\
    & =\frac{b_j-\rho \sigma u i+d j}{\sigma^2} \frac{1}{d_j}\cdot \int_{1}^{\exp\left\{d_j \tau\right\}} \frac{1-g_j y-\left(y-g_j y\right)}{y\left(1-g_j y\right)} \:dy \\
    & =\frac{b_j-\rho \sigma u i+d j}{\sigma^2} \frac{1}{d_j} \cdot\int_{1}^{\exp\left\{d_j \tau\right\}} \frac{1}{y}-\frac{1-g_j}{1-g_j y} \:dy \\
    & =\frac{b_j-\rho \sigma u i+d j}{\sigma^2} \frac{1}{d_j}\cdot\left(\int_{1}^{\exp\left\{d_j \tau\right\}} \frac{1}{y} d y-\int_{1}^{\exp\left\{d_j \tau\right\}} \frac{1-g_j}{1-g_j y} d y\right)\\
    &\overset{\dagger\dagger\dagger}{=}\frac{b_j-\rho \sigma u i+d j}{\sigma^2} \frac{1}{d_j}\cdot \left ( \left [ \log{(y)}\right ]_1^{\exp\left\{d_j \tau\right\}}+\frac{1-g_j}{g_j} \left [ \log{(1-g_jy)}\right ]_1^{\exp\left\{d_j \tau\right\}}\right )\\
    & =\frac{b_j-\rho \sigma u i+d j}{\sigma^2} \frac{1}{d_j}\\&\quad \cdot \left(\left[\log \left(\exp\left\{d_j \tau\right\} \right)-\log{(1)}\right]+\frac{1-g_j}{g_j}\left[\log \left(1-g_j \exp\left\{d_j \tau\right\}\right)-\log \left(1-g_j \right)\right]\right) \\
    &=\frac{b_j-\rho \sigma u i+d j}{\sigma^2} \frac{1}{d_j}\cdot\left(d_j \tau+\frac{1-g_j}{g_j} \log \left(\frac{1-g_j \exp\left\{d_j \tau\right\}}{1-g_j}\right)\right),
\end{align*}
where $\dagger$ follows from our just found expression for $D_j$,
$\dagger\dagger$ from substition using:
\begin{align*}
    y=\exp\left\{d_j s\right\}, \quad dy=d_j\exp\left\{d_j s\right\}ds,\quad ds=\frac{dy}{d_j y},
\end{align*}
and the fact that when at $\tau \Rightarrow \exp\left\{d_j \tau\right\}=\exp\left\{d_j \tau\right\}$ and at
zero $\Rightarrow \exp\left\{d_j \cdot 0 \right\}=1$. Lastly,
$\dagger\dagger\dagger$ follows from:
\begin{align*}
    \frac{d}{dy}\frac{1-g_j}{g_j}\log{(1-g_j y)}&=\frac{1}{1-g_jy}\frac{1-g_j}{g_j}(-g_j)\\
    &=-\frac{1-g_j}{1-g_jy}\\
    &\Rightarrow\\
    -\int\frac{1-g_j}{1-g_jy}\:dy&=\frac{1-g_j}{g_j}\log{(1-g_jy)}+K, \quad K\in \RR.
\end{align*}
where in our case the integral is definite and hence no constant $K$ added. We are not quite done as we still need to substitute the found integral back
into the expression for $C_j$:
\begin{equation*}
    \hspace*{-1.4cm}
    \begin{aligned}
        C_j&=rui\tau+a\int_0^\tau D_j(s;u)\:ds.\\
        & =rui\tau+a \frac{b_j-\rho \sigma u i+d_j}{\sigma^2} \frac{1}{d_j}\cdot\left(d_j \tau+\frac{1-g_j}{g_j} \log \left(\frac{1-g_j \exp\left\{d_j \tau\right\}}{1-g_j}\right)\right) \\
        & =rui\tau+\frac{a}{\sigma^2}\cdot\left(\left(b_j-\rho \sigma u i+d_j\right) \tau+\frac{1}{d_j}\left(b_j-\rho \sigma u i+d_j\right) \frac{1-g_j}{g_j} \log \left(\frac{1-g_j \exp\left\{d_j \tau\right\}}{1-g_j}\right)\right) \\
        & \overset{\dagger}{=}rui\tau+\frac{a}{\sigma^2}\cdot\left(\left(b_j-\rho \sigma u i+d_j\right) \tau+\frac{1}{d_j}\left(b_j-\rho \sigma u i+d_j\right) \frac{1-\frac{\rho \sigma u i-b_j-d_j}{\rho \sigma u i-b_j+d_j}}{\frac{\rho \sigma u i-b_j-d_j}{\rho \sigma u i-b_j+d_j} }\log \left(\frac{1-g_j \exp\left\{d_j \tau\right\}}{1-g_j}\right)\right) \\
        &=rui\tau+\frac{a}{\sigma^2}\cdot\left(\left(b_j-\rho \sigma u i+d_j\right) \tau+\frac{1}{d_j}\left(b_j-\rho \sigma u i+d_j\right) \frac{\left(\rho \sigma u i-b_j+d_j\right)-\left(\rho \sigma u i-b_j-d_j\right)}{\rho \sigma u i-b_j-d_j} \log \left(\frac{1-g_j \exp\left\{d_j \tau\right\}}{1-g_j}\right)\right) \\
        & =rui\tau+\frac{a}{\sigma^2}\cdot\left(\left(b_j-\rho \sigma u i+d_j\right) \tau+\frac{1}{d_j}\left(b_j-\rho \sigma u i+d_j\right) \frac{2 d_j}{\rho \sigma u i-b_j-d_j} \log \left(\frac{1-g_j \exp\left\{d_j \tau\right\}}{1-g_j}\right)\right) \\
        & =rui\tau+\frac{a}{\sigma^2}\cdot\left(\left(b_j-\rho \sigma u i+d_j\right) \tau-\frac{1}{d_j} 2 d_j \frac{\rho \sigma u i-b_j-d_j}{\rho \sigma u i-b_j-d_j} \log \left(\frac{1-g_j \exp\left\{d_j \tau\right\}}{1-g_j}\right)\right) \\
        &=rui\tau+\frac{a}{\sigma^2}\cdot\left(\left(b_j-\rho \sigma u i+d_j\right) \tau-2 \log \left(\frac{1-g_j \exp\left\{d_j \tau\right\}}{1-g_j}\right)\right),
    \end{aligned}
\end{equation*}
where $\dagger$ follows from $g_j:=x_{j,-}/x_{j,+}$.
\newpage
\subsection{Numerical implementation of characteristic functions
}
$\textcolor{blue}{\star}$ All the code can be viewed in \autoref{chap:code}. 

$\textcolor{blue}{\star}$ From \cite{Havrylenko2024}, pp. 59-65 we define a
function to compute the characteristic functions $\Psi_j$, $j=1,2$, given by:
\begin{align*}
    \Psi_{x(T)}^j(x,v,\tau ;u)=\exp \lc C_j (\tau; u) + D_j (\tau; u) \cdot v + i  u x \rc,
\end{align*}
with:
\begin{align*}
    C(\tau ; u) &= rui\tau + \frac{a}{\sigma^2}\left( (b_j - \rho\sigma ui + d_j) \tau - 2 \log \left( \frac{1 - g_j e^{d_j \tau}}{1 - g_j} \right) \right), \\
    D_j(\tau ; u) &= \frac{b_j - \rho \sigma ui + d_j}{\sigma^2} \frac{1 - e^{d_j \tau}}{1 - g_j e^{d_j \tau}}, \\
    g_j &= \frac{b_j - \rho\sigma ui + d_j}{b_j - \rho\sigma ui - d_j}, \\
    d_j &= \sqrt{(\rho\sigma ui - b_j)^2 - \sigma^2(2u_jui - u^2)},\\
    u_1 &= \frac{1}{2}, \quad u_2 = -\frac{1}{2}, \quad a = \kappa\theta, \quad b_1 = \kappa + \lambda - \rho\sigma, \quad b_2 = \kappa + \lambda,
\end{align*}
that takes parameters seen in Table 1 with given specific values. We define the
function in \texttt{Python} as \texttt{characteristicFunctionHeston}. All the
requirements are met with the function parameter \texttt{j} to choose which
characteristic function to compute (and draw corresponding
$C_j,\:D_j,\:g_j,\:d_j$ and $u,b$ formulas from the chosen \texttt{j}).
\begin{table}[!h]
    \centering
    \begin{tabular}{lll}
    \toprule
    \textbf{Description} & \textbf{Symbol} & \textbf{Value} \\
    \midrule
    Current Stock Price & $S_t$ & 100 \\
    Current Variance & $v_t$ & 0.06 \\
    Risk-free Interest Rate & $r$ & 0.05 \\
    Speed of Mean Reversion & $\kappa$ & 1 \\
    Long-term Mean of Variance & $\theta$ & 0.06 \\
    Volatility of the Volatility & $\sigma$ & 0.3 \\
    Correlation Coefficient & $\rho$ & -0.5 \\
    Market Price of Risk & $\lambda$ & 0.01 \\
    Time to Expiry & $\tau$ & 1 \\
    \bottomrule
    \end{tabular}
    \caption{Parameters for computing the characteristic functions $\Psi_1$, $\Psi_2$.}
    \label{table:characteristic_functions}
\end{table}

\newpage
The plot seen in Figure 1 is of the real and imaginary part of
$\Psi_j$, $j=1,2$ computed using the above defined function in \texttt{Python}.
The pattern for the real and imaginary part is seen, as oscillating with peaks
(maximum amplitudes)
around the middle of the interval
for $u\in [-20,20]$, i.e near $u \approx 0$, and lower amplitude oscillations around the
ends, i.e near $u\approx \pm 20$. More specifically, we see that the real part
of the characteristic functions is even but the imaginary part is odd. The graph
of $\mathfrak{Im}(\Psi_j)$ has odd symmetry, meaning it has a rotational
symmetry about the origin, and the graph of $\mathfrak{Re}(\Psi_j)$ has even
symmetry, meaning it has symmetry about the second axis - although it requires
some \textit{imagination} to see said symmetry as the second axis is not centered.
\begin{figure}[!h]
    \begin{adjustwidth}{-2cm}{-2cm}  
      \centering
      \includegraphics[width=1\linewidth]{1.3.png} 
      \label{Figure 1}
    \end{adjustwidth}
    \captionsetup{skip=-20pt}
    \caption{Real and imaginary part of $\Psi_j$, $j=1,2$ for $u\in [-20,20]$.}
\end{figure}

\newpage
\section{Option pricing in the context of the Heston model}
We will report results with five significant figures throughout this section of
the HandIn. Computation times will be recorded in \textit{seconds (s)}. When
references are given to (x.y) (for example 2.1) it is to be understood as
Subsection (x.y) (subsection 2.1) unless otherwise stated.
\subsection{Monte Carlo and Euler discretization Scheme}
$\textcolor{blue}{\star}$ All the code can be viewed in \autoref{chap:code}.

$\textcolor{blue}{\star}$ We simulate $S_t$ and $v_t$ under $\tilde{\QQ}$ over the time interval $[0, T]$ (to
and including expiry),
which we assume to be discretized as $0 = t_1 < t_2 < \ldots < t_m = T$, where
the time increments are equally spaced with width $d t$. For $n\in\NN$ we use
the Euler discretization on an equidistan time grid $\lc t_i=\frac{iT}{n}\mid
i=0,\ldots,n\rc$, where $n$ is the number of time points after $t=0$. From
\cite{Rouah2024}, p. 4 we start with the initial
values $S_0$ for the stock price and $v_0$ for the variance. Given a value for
$v_t$ at time $t$ (remember $t_1=0$), we first obtain the discretization of $v_t$ by $v_{t+dt}$ from:
\begin{align*}
    v_{t+dt} = v_t + \kappa(\theta - v_t)dt + \sigma \sqrt{v_t dt}Z_v^\PP,
\end{align*}
and we obtain the discretization of $S_t$ by $S_{t+dt}$ from:
\begin{align*}
    S_{t+dt} = S_t + rS_tdt + \sqrt{v_t dt}S_tZ_s^\PP,
\end{align*}
where to generate $Z_v^\PP$ and $Z_s^\PP$ with correlation $\rho$, we first generate two
independent standard normal variables $Z_1^\PP$ and $Z_2^\PP$, and set $Z_v^\PP = Z_1^\PP$ and
$Z_s^\PP = \rho Z_1^\PP + \sqrt{1 - \rho^2} Z_2^\PP$ by Cholesky decomposition
\cite{Havrylenko2024}, p. 43.

For an arbitrary but fixed $\lambda$ (i.e no estimation of $\lambda$ needed) we
follow as suggested by \cite{heston1993closed} (see \cite{Havrylenko2024}, p.
45) the adjusments that preserve the structure $v$ in the Heston model as we
want to preform a change of measure from $\PP$ to $\tilde{\QQ}$ corresponding to
$\lambda$:
\begin{align*}
    \tilde{\kappa}=\kappa+\lambda \quad \text{ and } \quad \tilde{\theta}=\frac{\kappa\theta}{\kappa + \lambda},
\end{align*}
and thus:
\begin{align*}
    S_{t+dt} &= S_t + rS_tdt + \sqrt{v_t dt}S_tZ_s^{\tilde{\QQ}},\\
    v_{t+dt} &= v_t +\tilde{\kappa}(\tilde{\theta} - v_t)dt + \sigma \sqrt{v_t dt}Z_v^{\tilde{\QQ}},
\end{align*}
yielding that we simulate under the equivalent Martingale measure
$\tilde{\QQ}(\lambda)$ for that one path, i.e choosing a different arbitrary
$\lambda^\star \neq \lambda$ would yield another equivalent Martingale measure
$\tilde{\QQ}(\lambda^\star)$ corresponding to $\lambda^\star$. This leads
to analytically tractable results for option pricing.

Lastly, we note that there might occure negative $v$-values at some of the time
points due to discretization errors \textbf{even} if the Feller condition is met (see
\cite{Havrylenko2024}, p. 28). When this occurs we use a full truncation
scheme, namely, $\max\lc 0,v_t\rc$ but firstly count it towards
our 0-Variance-Count for later reporting. In the code we actually simply
check using a \texttt{if}-statement if the variance at time $t_i$ is less than
or equal to $0$ - if not - its counted and then forced to 0.
The function \texttt{generateHestonPathEulerDisc} implements the beforementioned
theory as wished in the HandIn \#3 description.
\newpage
The simulation in \texttt{Python} is done with the parameters seen in Table 2.
\begin{table}[!h]
    \centering
    \begin{tabular}{lll}
    \toprule
    \textbf{Parameter} & \textbf{Symbol} & \textbf{Value} \\
    \midrule
    Initial Stock Price & $S_0$ & 100 \\
    Initial Variance & $v_0$ & 0.06 \\
    Risk-free Interest Rate & $r$ & 0.05 \\
    Speed of Mean Reversion & $\kappa$ & 1 \\
    Long-term Mean of Variance & $\theta$ & 0.06 \\
    Volatility of the Volatility & $\sigma$ & 0.3 \\
    Correlation Coefficient & $\rho$ & -0.5 \\
    Market Price of Risk & $\lambda$ & 0.01 \\
    Expiry Time & $T$ & 1 \\
    Strike Price & $K$ & 100 \\
    Number of Time Points & $n$ & 100 \\
    Number of Paths & $N$ & 1000 \\
    \bottomrule
    \end{tabular}
    \caption{Simulation and model parameters for the Heston model to calculate the European call option price using the Monte Carlo and Euler discretization scheme.}
    \label{table:call_option_pricing_parameters1}
    \end{table}

The results from simulation using the pricing-function \texttt{priceHestonCallViaEulerMC} for three different seeds
(\texttt{np.random.seed()}) is seen in Table 3. We use the
\texttt{generateHestonPathEulerDisc}-function to simulate $N$ paths and then
estimate prices of the European call option on the underlying asset $S$
following a Heston model by discounting the mean of the payoffs of the $N$ paths as this would
yield prices near the expectation under $\tilde{\QQ}$ corresponding to $\lambda$
by the Law of Large
Numbers (LLN) - as we do in Monte Carlo estimates.

\begin{table}[ht]
    \centering
    \begin{tabular}{|l|c|c|c|}
    \hline
    \multicolumn{4}{|c|}{\textbf{Euler Monte Carlo}} \\ \hline
    \textbf{}  & \textbf{Seed A} & \textbf{Seed B} & \textbf{Seed C} \\ \hline
    \textbf{Option Price} & 12.247 & 11.849 & 11.781 \\ \hline
    \textbf{Standard Deviation} & 0.51792 & 0.51211 & 0.52947 \\ \hline
    \textbf{Computing Time} & 0.70570 & 0.61735 & 0.70212 \\ \hline
    \textbf{0-Variance-Count} & 52 & 47 & 36 \\ \hline
    \end{tabular}
    \caption{Simulation results using Euler Monte Carlo discretization for seeds $\text{\textbf{A}}=1$, $\text{\textbf{B}}=10$ and $\text{\textbf{C}}=100$.}
    \label{table:simulation_results1}
\end{table}



Observe that the option price depends upon the seed-number which implies that the
Monte Carlo, and specifically Euler discretization, scheme might not be precise
enough for actual option pricing.
Furthermore, we kept the number of simulations fixed, as requested. If this
constraint was lifted and simulation was done at a higher number of paths the
results for every seed would differ as well. These findings for Monte Carlo
follow exactly as given by
\cite{Havrylenko2024}, p. 73.
\newpage
\subsection{Monte Carlo and Milstein discretization scheme}
$\textcolor{blue}{\star}$ All the code can be viewed in \autoref{chap:code}.

$\textcolor{blue}{\star}$ We refer to the logical build up for the functions
used in 2.1 but shortly explain the implementation of Monte Carlo method but now with
the Milstein discretization scheme. Initial values $S_0$ and $v_0$ are given. From \cite{Rouah2024}, p. 8, given a value for
$v_t$ at time $t$, we first obtain the discretization of $v_t$ by $v_{t+dt}$ from:
\begin{align*}
    v_{t+dt} = v_t + \kappa(\theta - v_t)dt + \sigma \sqrt{v_tdt}Z_v^\PP+\frac{1}{4}\sigma^2dt\left ((Z_v{^\PP})^2-1 \right ),
\end{align*}
and we obtain the discretization of $S_t$ by $S_{t+dt}$ from:
\begin{align*}
    S_{t+dt} = S_t + rS_tdt + \sqrt{v_t dt}S_tZ_s^\PP+\frac{1}{4}S_t^2dt\left ((Z_s{^\PP})^2-1 \right ),
\end{align*}
where to generate $Z_v^\PP$ and $Z_s^\PP$ with correlation $\rho$, we first
generate two independent standard normal variables $Z_1^\PP$ and $Z_2^\PP$, and
set $Z_v^\PP = Z_1^\PP$ and $Z_s^\PP = \rho Z_1^\PP + \sqrt{1 - \rho^2} Z_2^\PP$ by
Cholesky decomposition \cite{Havrylenko2024}, p. 43.

For an arbitrary but fixed $\lambda$ (i.e no estimation of $\lambda$ needed) we
follow as suggested by \cite{heston1993closed} (see \cite{Havrylenko2024}, p.
45) the adjusments that preserve the structure $v$ in the Heston model as we
want to preform a change of measure $\PP$ to $\tilde{\QQ}$ corresponding to
$\lambda$ (as
explained in EulerMC (2.1)):
\begin{align*}
    \tilde{\kappa}=\kappa+\lambda \quad \text{ and } \quad \tilde{\theta}=\frac{\kappa\theta}{\kappa + \lambda},
\end{align*}
and thus:
\begin{align*}
    S_{t+dt} &= S_t + rS_tdt + \sqrt{v_t dt}S_tZ_s^{\tilde{\QQ}}+\frac{1}{4}S_t^2dt\left ((Z_s^{{\tilde{\QQ}}})^2-1 \right )\\
    v_{t+dt} &= v_t +\tilde{\kappa}(\tilde{\theta} - v_t)dt + \sigma \sqrt{v_tdt}Z_v^{\tilde{\QQ}}+\frac{1}{4}\sigma^2dt\left ((Z_v^{\tilde{\QQ}})^2-1 \right ).
\end{align*}
The simulation in \texttt{Python} is done with the parameters seen in Table 4.
\begin{table}[!h]
    \centering
    \begin{tabular}{lll}
    \toprule
    \textbf{Parameter} & \textbf{Symbol} & \textbf{Value} \\
    \midrule
    Initial Stock Price & $S_0$ & 100 \\
    Initial Variance & $v_0$ & 0.06 \\
    Risk-free Interest Rate & $r$ & 0.05 \\
    Speed of Mean Reversion & $\kappa$ & 1 \\
    Long-term Mean Variance & $\theta$ & 0.06 \\
    Volatility of the Volatility & $\sigma$ & 0.3 \\
    Correlation Coefficient & $\rho$ & -0.5 \\
    Market Price of Risk & $\lambda$ & 0.01 \\
    Expiry Time & $T$ & 1 \\
    Strike Price & $K$ & 100 \\
    Number of Time Points & $n$ & 100 \\
    Number of Paths & $N$ & 1000 \\
    \bottomrule
    \end{tabular}
    \caption{Simulation and model parameters for the Heston model to calculate the European call option price using the Monte Carlo and Milstein discretization scheme.}
    \label{table:call_option_pricing_parameters2}
    \end{table}
\newpage
The results from simulation in \texttt{Python} for three different seeds
(\texttt{np.random.seed()}) is seen in Table 5. The logical build up of
\texttt{generateHestonPathMilsteinDisc} and
\texttt{priceHestonCallViaMilsteinMC} is \textbf{exactly} thoes given in EulerMC (2.1)
but with the different discretization scheme. We therefore spare you the
explanation once more.
\begin{table}[ht]
    \centering
    \begin{tabular}{|l|c|c|c|}
    \hline
    \multicolumn{4}{|c|}{\textbf{Milstein Monte Carlo}} \\ \hline
    \textbf{} & \textbf{Seed A} & \textbf{Seed B} & \textbf{Seed C} \\ \hline
    \textbf{Option Price} & 12.253 & 11.827 & 11.778 \\ \hline
    \textbf{Standard Deviation} & 0.51785 & 0.51029 & 0.52915 \\ \hline
    \textbf{Computing Time} & 0.75349 & 0.78604 & 0.92265 \\ \hline
    \textbf{0-Variance-Count} & 0 & 0 & 0 \\ \hline
    \end{tabular}
    \caption{Simulation results using Monte Carlo and Milstein discretization for seeds $\text{\textbf{A}}=1$, $\text{\textbf{B}}=10$ and $\text{\textbf{C}}=100$.}
    \label{table:simulation_results2}
\end{table}



The option prices, standard deviations and computing time are of the same ball park as the ones
given in 2.1 (EulerMC) and thus subject to the same criticism in terms of seed-
and path-dependency. It should be noted that computation time is highly
reliant on the state of your computer at that given moment and the drag on your
CPU/GPU from other applications

However, of much importance, note that the 0-Variance-Count is $0$ for
all three seeds. This means that the Milstein discretization scheme seems to
produce roughly the same issues as the Euler discretization with the important
difference of producing far fewer (here, none) instances of a variance having to
be truncated to 0. These findings for Monte Carlo
follow as given by
\cite{Havrylenko2024}, p. 73 and the observation for the 0-Variance-Count is
exactly as mentioned by \cite{Rouah2024}, p. 7. 

The reason for the difference in $0$-Variance-Count
is that the key to the Milstein Scheme is that the accuracy of the
discretization is increased by considering expansion of both $\mu_t=\mu(S_t)$ and
$\sigma_t=\sigma(S_t)$ and - specifically -  $\mu_t=\mu(v_t)$ and
$\sigma_t=\sigma(v_t)$ by Ito's lemma. This expansion of $\mu_t=\mu(v_t)$ and
$\sigma_t=\sigma(v_t)$ adds a correction-term which differs
from EulerMC (2.1):
\begin{align*}
    v_{t+dt} = \underbrace{v_t +\tilde{\kappa}(\tilde{\theta} - v_t)dt + \sigma \sqrt{v_tdt}Z_v^{\tilde{\QQ}}}_{\text{EulerMC discretization scheme}}+\underbrace{\frac{1}{4}\sigma^2dt\left ((Z_v^{{\tilde{\QQ}}})^2-1 \right )}_{\text{correction}},
\end{align*}
\begin{align*}
    S_{t+dt} = \underbrace{S_t + rS_tdt + \sqrt{v_t dt}S_tZ_s^{\tilde{\QQ}}}_{\text{EulerMC discretization scheme}}+\underbrace{\frac{1}{4}S_t^2dt\left( (Z_s^{\tilde{\QQ}})^2-1 \right )}_{\text{correction}}.
\end{align*}
However, even though no instances required truncation of the variance to $0$ under these three seeds, the full
truncation scheme must still be applied to $v_{t+dt}$ as this is not a
guarantee even though one might be tempted to draw such conclusion based on the
above simulation results. Some seeds may (and do) produce $0$-Variance-Count
under the MilsteinMC discretization scheme.
\newpage
\subsection{Heston’s original formula}
$\textcolor{blue}{\star}$ All the code can be viewed in \autoref{chap:code}.

$\textcolor{blue}{\star}$ In this subsection we employ Heston's original formula
to compute the price of the European call option on the underlying asset $S$
following the Heston model. From \cite{Havrylenko2024}, pp. 59-66, analogous to
Subsection (1.3), we define a function \texttt{characteristicFunctionHeston} to compute the
characteristic functions $\Psi_j$, $j=1,2$, given by:
\begin{align*}
    \Psi_{x(T)}^j(x,v,\tau;u)=\exp \lc C_j (\tau; u) + D_j (\tau; u) \cdot v + i  u x \rc,
\end{align*}
with:
\begin{align*}
    C_j(\tau ; u) &= rui\tau + \frac{a}{\sigma^2}\left( (b_j - \rho\sigma ui + d_j) \tau - 2 \log \left( \frac{1 - g_j e^{d_j \tau}}{1 - g_j} \right) \right), \\
    D_j(\tau ; u) &= \frac{b_j - \rho \sigma ui + d_j}{\sigma^2} \frac{1 - e^{d_j \tau}}{1 - g_j e^{d_j \tau}}, \\
    g_j &= \frac{b_j - \rho\sigma ui + d_j}{b_j - \rho\sigma ui - d_j}, \\
    d_j &= \sqrt{(\rho\sigma ui - b_j)^2 - \sigma^2(2u_jui - u^2)},\\
    u_1 &= \frac{1}{2}, \quad u_2 = -\frac{1}{2}, \quad a = \kappa\theta, \quad b_1 = \kappa + \lambda - \rho\sigma, \quad b_2 = \kappa + \lambda.
\end{align*}
Using the characteristic function, the price of the European call option under
$\tilde{\QQ}$ in the Heston model corresponding to $\lambda$ using Heston's
original formula is given by \cite{Havrylenko2024}, p. 66 as:
\begin{align*}
    Call^{\text{OriginalFT}}(t)=S_tQ_1(\log(S(t)),v(t),t;\log(K))-Ke^{-r\tau}Q_2(\log(S(t)),v(t),t;\log(K)),
\end{align*}
where for $j=1,2$:
\begin{align*}
    Q_j(\log(S(t)),v(t),t;\log(K))=\frac{1}{2}+\frac{1}{\pi}\int_0^{+\infty} \mathfrak{Re} \left ( \frac{e^{-iu\log(K)}\Psi^j_{\log(S(T))}(\log(S(t)),v(t),t;u)}{iu}\right )du.
\end{align*}
In theory we do not need to consider the real part of the integral
but as we need (:in our implementation) to truncate the integral the need arises as there will be an
imaginary part. 

\newpage
Pricing of the call option is calculated using the parameters
seen in
Table 6.
\begin{table}[!h]
    \centering
    \begin{tabular}{lll}
    \toprule
    \textbf{Parameter} & \textbf{Symbol} & \textbf{Value} \\
    \midrule
    Initial Stock Price & $S_0$ & 100 \\
    Initial Variance & $v_0$ & 0.06 \\
    Risk-free Interest Rate & $r$ & 0.05 \\
    Speed of Mean Reversion & $\kappa$ & 1 \\
    Long-term Mean of Variance & $\theta$ & 0.06 \\
    Volatility of the Volatility & $\sigma$ & 0.3 \\
    Correlation Coefficient & $\rho$ & -0.5 \\
    Market Price of Risk & $\lambda$ & 0.01 \\
    Time to Expiry & $\tau$ & 1 \\
    Strike Price & $K$ & 100 \\
    \bottomrule
    \end{tabular}
    \caption{Model parameters for the Heston model to calculate the European call option price using Heston's original formula.}
    \label{table:call_option_pricing_parameters3}
    \end{table}

However, as noted in \cite{Havrylenko2024}, p. 66, computation of the above
integrals in $Q_j$, $j=1,2$, can be numerically tricky because of $\log$ in
$C_j$ as the $\log$ has a branch cut along the negative real
axis that most numerical integration software packages has difficulty with unlike the
root function in $d_j$ (see for example \cite{Albrecher2006LittleHeston} for
their description of the axis of evil). To overcome this obstacle and evaluate the integral numerically
we employ the \texttt{quad}-function. The \texttt{quad}-function in
\texttt{Python}'s \texttt{scipy.integrate}-module \cite{2020SciPy-NMeth} performs numerical integration
of $Q_j$, $j=1,2$ over a specified (truncated) interval  using adaptive quadrature methods
from the \texttt{QUADPACK}-library. We evaluate both integrals in increase upper
bounds and then compare the European call price and computation time with the
results obtained from 2.1 (EulerMC) and 2.2 (MilsteinMC).

The results can be seen in Table 7 with different increasing upper bounds, $u_{max}$.

\begin{table}[ht]
    \centering
    \begin{tabular}{|l|c|c|c|c|}
    \hline
    \multicolumn{5}{|c|}{\textbf{Heston's Original Formula}} \\ \hline
    \textbf{} & \textbf{Bounds $[0,10]$} & \textbf{Bounds $[0,50]$} & \textbf{Bounds $[0,100]$} & \textbf{Bounds $[0,1000]$} \\ \hline
    \textbf{Option Price} & 11.8122 & 11.936 & 11.936 & 11.936 \\
    \textbf{Computing Time} & $1.0023\times 10^{-3}$ & $2.9919 \times 10^{-3}$ &
    $4.9834 \times 10^{-3}$
    & $7.9796 \times 10^{-3}$ \\ \hline
    \end{tabular}
    \caption{Heston's original formula results using numerical integration with different (truncated) upper bounds: $[0,10]$, $[0,50]$, $[0,100]$  and $[0,1000]$.}
    \label{table:results3}
\end{table}

Table 7 results show that the option prices computed by Heston's original formula
through numerical integration are rougly the same as the ones from EulerMC (2.1)
and MilsteinMC (2.2). An important notice is with the increasing upper bound we
seem to (at seven significant figures) achieve the same option price. Furthermore
this option price is not reliant on a seed. Aiming for a unique price for the
European call option, the features of not depending on a seed and achieving a
quickly stabilizing price as the upper bound increases are advantageous.  Both
of these two properties were lacking in EulerMC (2.1)
and MilsteinMC (2.2), however, do note that the call price is in the same ball
park which does mean they could be used for exotic option pricing.

The major difference in Heston's original formula compared to EulerMC (2.1)
and MilsteinMC (2.2) lies in the computating time. We see differences of up to -
and exceeding - a
factor of $100$ between the computing times of Heston's original formula relative to EulerMC (2.1)
and MilsteinMC (2.2). Note however the
computation time is indeed, and as expected, increasing in the bounds of
integration but so would the increase in paths for EulerMC (2.1)
and MilsteinMC (2.2). Advantagously, observe how precise Heston's original formula is
for a relatively small upper bounds and increasing the upper bound
further seems to not effect the option price up to five significant figures.

It should again be noted that computation time is highly
reliant on the state of your computer at that given moment and the drag on your CPU/GPU.
\newpage
\subsection{Carr-Madan formula}
$\textcolor{blue}{\star}$ All the code can be viewed in \autoref{chap:code}.

$\textcolor{blue}{\star}$ In this subsection we employ Carr-Madan approach to
Fourier transform pricing
to compute the price of the European call option on the underlying asset $S$
following the Heston model. From \cite{Havrylenko2024}, 59-66,
analogous to Subsections
(1.3) and (2.3), we have the characteristic functions $\Psi_j$, $j=1,2$, given by:
\begin{align*}
    \Psi_{x(T)}^j(x,v,\tau;u)=\exp \lc C_j (\tau; u) + D_j (\tau; u) \cdot v + i  u x \rc,
\end{align*}
with:
\begin{align*}
    C_j(\tau ; u) &= rui\tau + \frac{a}{\sigma^2}\left( (b_j - \rho\sigma ui + d_j) \tau - 2 \log \left( \frac{1 - g_j e^{d_j \tau}}{1 - g_j} \right) \right), \\
    D_j(\tau ; u) &= \frac{b_j - \rho \sigma ui + d_j}{\sigma^2} \frac{1 - e^{d_j \tau}}{1 - g_j e^{d_j \tau}}, \\
    g_j &= \frac{b_j - \rho\sigma ui + d_j}{b_j - \rho\sigma ui - d_j}, \\
    d_j &= \sqrt{(\rho\sigma ui - b_j)^2 - \sigma^2(2u_jui - u^2)},\\
    u_1 &= \frac{1}{2}, \quad u_2 = -\frac{1}{2}, \quad a = \kappa\theta, \quad b_1 = \kappa + \lambda - \rho\sigma, \quad b_2 = \kappa + \lambda.
\end{align*}
However, using the characteristic function the price of the European call option under
$\tilde{\QQ}$ in the Heston model corresponding to $\lambda$ using Carr-Madan's
Fourier transform formula is
given on \cite{Havrylenko2024}, p. 77 by (denoting $k:=\log K$):
\begin{align*}
    Call^{\text{CarrMadan}}(k) = \frac{e^{-\alpha k}}{\pi} \mathfrak{Re} \left( \int_{0}^{+\infty} \frac{e^{-iu k} e^{-rT} \Psi_{x(T)}\left(u - (1 + \alpha)i\right)}{\alpha^{2} + \alpha - u^{2} + i(1 + 2\alpha)u} \, du \right).
\end{align*}
In other words, only one integration scheme is required in opposition to two in
Heston's original formula (2.3) and we thus need to ensure we correctly
transition from $\PP$ to $\tilde{\QQ}$ using the correct $\Psi_j$, $j=1,2$. So, from \cite{Havrylenko2024}, pp. 56-57 we
see that $Q_2$ is related to the measure $\tilde{\QQ}$ whereas for $Q_1$ we change
measure to $\QQ^S$. i.e to elaborate: we have that the price of a European call option can be
written as (by change of numeraire):
\begin{align*}
    Call(t)&=e^{-r\tau}\EE \left ( (S_T-K)^+\right )\\
    &=e^{-r\tau}\EE \left ( (S_T-K) 1_{S_T\geq K}\right )
    \\&=e^{-r\tau} \EE^{\tilde{\QQ}} \left ( S_T 1_{S_T \geq K} \right ) - K e^{-r\tau} \EE^{\tilde{\QQ}} \left ( 1_{S_T \geq K} \right )\\
    &=S_t\EE^{\tilde{\QQ}} \left ( \frac{S_T/S_t}{B_T/B_t} 1_{S_T \geq K}\right )-Ke^{-r\tau}\EE^{\tilde{\QQ}}\left ( 1_{S_T \geq K} \right )\\
    &=S_t\EE^{\QQ^S} \left ( \frac{d\tilde{\QQ}}{d\QQ^S}\frac{S_T/S_t}{B_T/B_t} 1_{S_T \geq K} \right )-Ke^{-r\tau}\EE^{\tilde{\QQ}}\left ( 1_{S_T \geq K} \right )\\
    &=S_t \EE^{\QQ^S}  \left ( 1_{S_T\geq K}\right ) - Ke^{-r\tau} \EE^{\tilde{\QQ}}\left ( 1_{S_T\geq K}\right )
    \\&=S_t\QQ^S(S_T \geq K)-Ke^{-r\tau}\tilde{\QQ}(S_T\geq K)\\
    &=S_t\QQ^S(\log{S_T} \geq \log{K})-Ke^{-r\tau}\tilde{\QQ}(\log{S_T} \geq \log{K})\\
    &=S_tQ_1-Ke^{-r\tau}Q_2,
\end{align*} 
where $\frac{\mathrm{d}\tilde{\QQ}}{\mathrm{d}\QQ^S}$ is the Radon-Nikodym derivative given by
$\frac{\mathrm{d}\tilde{\QQ}}{\mathrm{d}\QQ^S}=(B_T/B_t)/(S_T/S_t)$ as $\tilde{\QQ}$ uses the
bond $B_t$ as the numeraire, while the measure $\QQ^S$ uses the stock price $S$.
Resultingly, $Q_2$ is the correct function as it relates to $\tilde{\QQ}$ and thus $\Psi_2$
the correct characteristic function to use in the Carr-Madan European call
option formula. $\Psi_1$ can be used but has to be implemented differently (see
\cite{Havrylenko2024}, p. 71).
Consequently, when implementing the \texttt{characteristicFunctionHeston} into
the evaluation of the integral in the Carr-Madan Fourier transform pricing
formula for the European call option, we restrict \texttt{j=2} in the function
parameter. Furthermore, the
horizontal shift for the characteristic function argument is also restricted to
\texttt{(u-(1+alpha)i)} as opposed to \texttt{u} (see above formula for Carr-Madan).

In theory we do not need to consider the real part of the integral
but as we need (:in our implementation) to truncate the integral for numerical evaluation the need arises as there will be an
imaginary part. 

To evaluate the integral numerically we employ the \texttt{quad}-function in
\texttt{Python}, again (see subsection 2.4 for the same description of the axis
of evil from the $\log$-function).

Pricing of the call option is then calculated using the parameters seen in Table 8.
\begin{table}[!h]
    \centering
    \begin{tabular}{lll}
    \toprule
    \textbf{Parameter} & \textbf{Symbol} & \textbf{Value} \\
    \midrule
    Initial Stock Price & $S_0$ & 100 \\
    Initial Variance & $v_0$ & 0.06 \\
    Risk-free Interest Rate & $r$ & 0.05 \\
    Speed of Mean Reversion & $\kappa$ & 1 \\
    Long-term Mean of Variance & $\theta$ & 0.06 \\
    Volatility of the Volatility & $\sigma$ & 0.3 \\
    Correlation Coefficient & $\rho$ & -0.5 \\
    Market Price of Risk & $\lambda$ & 0.01 \\
    Time to Expiry & $\tau$ & 1 \\
    Strike Price & $K$ & 100 \\
    Damping Factor & $\alpha$ & 0.3 \\
    \bottomrule
    \end{tabular}
    \caption{Model parameters for the Heston's model European call option pricing using Carr-Madan's Fourier transfom approach.}
    \label{table:call_option_pricing_parameters4}
\end{table}

The results can be seen in Table 9  with different increasing upper bounds and
specifically $u_{max}=50$ as requested in the HandIn \#3.
\begin{table}[ht]
    \centering
    \begin{tabular}{|l|c|c|c|c|}
    \hline
    \multicolumn{5}{|c|}{\textbf{Carr-Madan's Fourier Transform Formula}} \\ \hline
    \textbf{} & \textbf{Bounds $[0,10]$} & \textbf{Bounds $[0,50]$} & \textbf{Bounds $[0,100]$} & \textbf{Bounds $[0,1000]$} \\ \hline
    \textbf{Option Price} & 11.886 & 11.936 & 11.936 & 11.936 \\
    \textbf{Computing Time} & $3.9701 \times 10^{-3}$ & $6.0089 \times 10^{-3}$ &
    $5.985 \times 10^{-3}$
    & $8.9772 \times 10^{-3}$ \\ \hline
    \end{tabular}
    \caption{Carr-Madan's Fourier transform results using numerical integration with different (truncated) upper bounds: $[0,10]$, $[0,50]$, $[0,100]$  and $[0,1000]$.}
    \label{table:results4}
\end{table}

Firstly, observe that the results of (2.4) are almost identical to thoes of
(2.3). We will however comment on why this might not be a general result when we
increase the number of call option calculations or if the computation was done
under the same CPU/GPU-drag.

If we were to price a large number of options we would expect the
Carr-madan method (2.4) to be faster in terms of computing time
compared to that of Heston's formula (2.3) and especially thoes of EulerMC (2.1) and MilsteinMC
(2.2). To see why consider both call option pricing formulas for (2.3):
\begin{align*}
    Call^{\text{OriginalFT}}(t)=S_tQ_1(\log(S(t)),v(t),t;\log(K))-Ke^{-r\tau}Q_2(\log(S(t)),v(t),t;\log(K)),
\end{align*}
and for (2.4) denoting $k:=\log K$:
\begin{align*}
    Call^{\text{CarrMadan}}(k) = \frac{e^{-\alpha k}}{\pi} \mathfrak{Re} \left( \int_{0}^{+\infty} \frac{e^{-iu k} e^{-rT} \Psi_{x(T)}\left(u - (1 + \alpha)i\right)}{\alpha^{2} + \alpha - u^{2} + i(1 + 2\alpha)u} \, du \right).
\end{align*}
Notice that the denominator of the Carr-Madan formula (2.3) decays the integral
relatively faster compared to that of Heston's original formula (2.4), because the integrand enters in the denominator squared in Carr-Madan formula (2.3)
where as it enters linearly in the denominator in Heston's original formula
(2.4). Another obvious advantageous is the fact that we have one integration
scheme in Carr-Madan formula (2.3) as opposed to two in Heston's original formula
(2.4) which for a larger number of options could be significant.

The key difference between Carr-Madan's formula and the methods EulerMC (2.1)
and MilsteinMC (2.2)  is the computation time, with Carr-Madan's formula being
faster by up to a factor of $100$ compared to EulerMC (2.1) and MilsteinMC (2.2). Despite the expected
increase in computation time with wider integration bounds Carr-Madan's (2.2) precision remains high for small
upper bounds without significantly affecting option prices by up to atleast five
decimal points.

It should again be noted as it is not negliable whatsoever that the computation time is highly
reliant on the state of your computer at that given moment and the drag on your
CPU/GPU.

In summary of question 2 we can easily conclude the following from
experimentation (and theory):

\begin{itemize}
    \item All the methods lead to prices around the same ball park.
    \item The Fourier transform based methods ((2.3) and (2.4)) were
    significantly faster at computing option prices of up to and exceeding a
    factor of 100 compared to EulerMC (2.1) and MilsteinMC (2.2).
    \item MilsteinMC (2.2) was identical to EulerMC (2.3) with the important
    difference of almost no occurences of 0-Variance-Count in the former method
    because of the expansions of $\mu(v_t)$ and $\sigma(v_t)$.
    \item EulerMC (2.1) and MilsteinMC (2.2) were both not precise enough
    methods for option pricing based on seed- and path-dependence.
    \item Carr-Madan (2.4) would be compared to Heston (2.3) as in the former
    the denominator in the integral contains $u^2$ as opposed to
    $u$ in the latter method. This could be significant for heavy computational option calculations
    that we did not consider here. Anecdotally, Yevhen's dissertation used
    Carr-Madan as it was the only sufficiently fast method.
\end{itemize}

\newpage
\section{Asset allocation under the Heston model}
\subsection{Derivation of the Hamilton-Jacobi-Bellman PDE}
$\textcolor{blue}{\star}$ Consider the set of admissible investment strategies given by
$\mathcal{A}(t,x,v)$. The wealth of the risk-averse investor then equals $x$ at
time $t$ for $t\leq T$. Consequently, the value function, $\mathcal{V}$, can be
written for any $t\in [0,T]$ as:
\begin{align*}
    \mathcal{V}(t,x,u)&=\max_{\pi\in \c{A}(t,x,v)} \EE \left [ U(X^\pi(T))\right ]\\
    &=\max_{\pi\in \mathcal{A}(t,x,v)} \EE \left [ U(X^\pi(T)) \mid X^\pi(t) =x, v(t)=v\right ].
\end{align*}
Maximizing expected utility over $\pi \in \c{A}(t, x, v)$ simplifies when $t =
T$, as the interval $t \in [0, T]$ simplifies to the single point $t = T$. At
$t=T$ the expectation vanishes as the investment horizon is reached and the
wealth and variance is thus known. This
leads directly to the terminal condition:
\begin{align*}
    \mathcal{V}(T,x,u)&=\max_{\pi\in \mathcal{A}(t,x,v)} \EE \left [ U(X^\pi(T)) \mid X^\pi(T) =x, v(T)=v\right ]\\
    &=\max_{\pi\in \c{A}(T,x,v)} \EE \left [ U(x)\right ]\\
    &=U(x).
\end{align*}
$\textcolor{blue}{\star}$ We have that the dynamics for the wealth process
$X^\pi(t + h)$ and variance $v(t + h)$ for some arbitrary constant investment strategy,
$\pi$, is given by:
\begin{align*}
    dX^\pi(t + h) &= X^\pi(t + h) \left( r + \pi \bar{\lambda} v(t + h) \right) d(t + h) + X^\pi(t + h) \pi \sqrt{v(t + h)} dW^{\PP}_1(t + h),\\
    dv(t + h) &= \kappa \left( \theta - v(t + h) \right) d(t + h) + \sigma \rho \sqrt{v(t + h)} dW^{\PP}_1(t + h) + \sigma \sqrt{v(t + h)} \sqrt{1 - \rho^2} dW^{\PP}_2(t + h).
\end{align*}
\newpage
We now assume - as we are told that $\mathcal{V}$ is sufficiently smooth - that $\mathcal{V}(t,x,u)\in C^{\text{1,2}}$ for the purpose of
applying multi-dimensional Ito Theorem 4.19 \cite{bjork2020}. This yields the
dynamics of the value process, $\mathcal{V}(t,x,u)$, for some arbitrary
constant investment strategy, $\pi$:
\begin{equation*}
    \hspace*{-1.5cm}
    \begin{aligned}
        d\mathcal{V}(t + h, X^\pi(t + h), v(t + h)) &= \mathcal{V}_t d(t + h) + \mathcal{V}_x dX^\pi(t + h) + \mathcal{V}_v dv(t + h) \\
        &\quad + \frac{1}{2} \mathcal{V}_{xx} \left(dX^\pi(t + h)\right)^2 + \frac{1}{2} \mathcal{V}_{vv} \left(dv(t + h)\right)^2 \\
        &\quad + \frac{1}{2} 2\mathcal{V}_{xv} \left(dX^\pi(t + h)\right)\left(dv(t + h)\right) \\
        &= \mathcal{V}_t d(t + h) + \mathcal{V}_x X^\pi(t + h) \left( r + \pi \bar{\lambda} v(t + h) \right) d(t + h) \\
        &\quad + \mathcal{V}_x X^\pi(t + h) \pi \sqrt{v(t + h)} dW^\PP_1(t + h) + \mathcal{V}_v \kappa \left( \theta - v(t + h) \right) d(t + h) \\
        &\quad + \mathcal{V}_v \sigma \rho \sqrt{v(t + h)} dW^\PP_1(t + h) + \mathcal{V}_v \sigma \sqrt{v(t + h)} \sqrt{1 - \rho^2} dW^\PP_2(t + h) \\
        &\quad + \frac{1}{2} \mathcal{V}_{xx} (X^\pi(t + h))^2 \pi^2 v(t + h) d(t + h) + \frac{1}{2} \mathcal{V}_{vv} \sigma^2 \rho^2 v(t + h) d(t + h) \\
        &\quad + \frac{1}{2} \mathcal{V}_{vv} \sigma^2 v(t + h) \left(1 - \rho^2\right) d(t + h) + \mathcal{V}_{xv} X^\pi(t + h) \pi v(t + h) \sigma \rho d(t + h) \\
        &= \Biggl(\mathcal{V}_t + \mathcal{V}_x X^\pi(t + h) \left( r + \pi \bar{\lambda} v(t + h) \right) + \mathcal{V}_v \kappa \left( \theta - v(t + h) \right) \\
        &\quad\quad  + \frac{1}{2} \mathcal{V}_{xx} (X^\pi(t + h))^2 \pi^2 v(t + h) + \frac{1}{2} \mathcal{V}_{vv} \sigma^2 v(t + h) + \mathcal{V}_{xv} X^\pi(t + h) \pi v(t + h) \sigma \rho  \Biggr) d(t + h) \\
        &\quad + \left( \mathcal{V}_x X^\pi(t + h) \pi \sqrt{v(t + h)} + \mathcal{V}_v \sigma \rho \sqrt{v(t + h)} \right) dW^\PP_1(t + h) \\
        &\quad + \mathcal{V}_v \sigma \sqrt{v(t + h)} \sqrt{1 - \rho^2} dW^\PP_2(t + h),
        \end{aligned}.
\end{equation*}
where we used the extended multiplication rules $dW_t^i dW_t^i=0$, $j\neq i$
\cite{bjork2020}, p. 61
and the dynamics of both $v(t+h)$ and $X^\pi(t+h)$ as they were stated in the excercise.

By the dynamics found using multi-dimensional Ito we now have that in integral
notation:
\begin{align*}
    \mathcal{V}(t + h, X^\pi(t + h), v(t + h)) &= \mathcal{V}(t, X^\pi(t), v(t)) \\
&\quad+ \int_{t}^{t+h} \Biggl( \mathcal{V}_t + \mathcal{V}_x X^\pi(s) \left( r + \pi \bar{\lambda} v(s) \right) + V_v \kappa \left( \theta - v(s) \right) \\
&\quad \quad + \frac{1}{2} \mathcal{V}_{xx} (X^\pi(s))^2 \pi^2 v(s) + \frac{1}{2} \mathcal{V}_{vv} \sigma^2 v(s) + \mathcal{V}_{xv} X^\pi(s) \pi v(s) \sigma \rho \Biggr) ds \\
&\quad+ \int_{t}^{t+h} \left( \mathcal{V}_x X^\pi(s) \pi \sqrt{v(s)} + \mathcal{V}_v \sigma \rho \sqrt{v(s)} \right) dW^\PP_1(s) \\
&\quad+ \int_{t}^{t+h} \mathcal{V}_v \sigma \sqrt{v(s)} \sqrt{1 - \rho^2} dW^\PP_2(s),
\end{align*}
as desired.
\newpage
$\textcolor{blue}{\star}$ Substitution of equation
(21) into (20) yields:
\begin{align*}
\mathcal{V}(t, x, v) &\geq \EE \left( \mathcal{V}(t+h, X^\pi(t+h), v(t+h)) \mid X^\pi(t)=x,v(t)=v \right )\\
&= \EE \left( \mathcal{V}(t, X^\pi(t), v(t)) + \int_{t}^{t+h} \Bigl( \mathcal{V}_t + \mathcal{V}_x X^\pi(s) \left( r + \pi\bar{\lambda} v(s) \right) + \mathcal{V}_v \kappa \left( \theta - v(s) \right) \right.  \\
&\quad\quad \left. + \frac{1}{2} \mathcal{V}_{xx} (X^\pi(s))^2 \pi^2 v(s) + \frac{1}{2} \mathcal{V}_{vv} \sigma^2 v(s) + \mathcal{V}_{xv} X^\pi(s) \pi v(s) \sigma \rho \Bigr) ds \right. \\
&\quad \left. + \int_{t}^{t+h} \left( \mathcal{V}_x X^\pi(s) \pi \sqrt{v(s)} + \mathcal{V}_v \sigma \rho \sqrt{v(s)} \right) dW^\PP_1(s) \right. \\
&\quad \left. + \int_{t}^{t+h} \mathcal{V}_v \sigma \sqrt{v(s)} \sqrt{1 - \rho^2} dW^\PP_2(s) \mid X^\pi(t) = x, v(t) = v \right) \\
&\overset{\dagger}{=} \mathcal{V}(t, x, v) + \EE \Bigl( \int_{t}^{t+h} \Bigl( \mathcal{V}_t + \mathcal{V}_{x} X^\pi(s)  \left( r + \pi v(s)\bar{\lambda} \right) + \mathcal{V}_v \kappa \left( \theta - v(s) \right)  \\
&\quad \quad + \frac{1}{2} \mathcal{V}_{xx}(X^\pi(s) )^2 \pi^2 v(s) + \frac{1}{2} \mathcal{V}_{vv} \sigma^2 v(s)+ \mathcal{V}_{xv} X^\pi(s) \pi v(s) \sigma \rho \Bigr) ds \mid X^\pi(t)=x,v(t)=v \Bigr)\\
&\iff\\
 0 &\geq \mathcal{V}(t, x, v) + \EE \Biggl( \int_t^{t+h}\Bigl( \mathcal{V}_t + \mathcal{V}_{x}  X^\pi(s)\left( r + \pi v(s) \bar{\lambda} \right) + \mathcal{V}_v \kappa \left( \theta - v(s) \right) \\
&\quad  \quad + \frac{1}{2} \mathcal{V}_{xx}( X^\pi(s))^2 \pi^2 v(s) + \frac{1}{2} \mathcal{V}_{vv} \sigma^2 v(s) + \mathcal{V}_{xv} X^\pi(s) \pi v(s) \sigma \rho \Bigr)ds \mid  X^\pi(t)=x,v(t)=v \Biggr) - \mathcal{V}(t, x, v) \\
&=\EE \Biggl( \int_t^{t+h}\Bigl( \mathcal{V}_t + \mathcal{V}_{x}  X^\pi(s)\left( r + \pi v(s) \bar{\lambda} \right) + \mathcal{V}_v \kappa \left( \theta - v(s) \right) \\
&\quad \quad + \frac{1}{2} \mathcal{V}_{xx}( X^\pi(s))^2 \pi^2 v(s) + \frac{1}{2} \mathcal{V}_{vv} \sigma^2 v(s) + \mathcal{V}_{xv} X^\pi(s) \pi v(s) \sigma \rho \Bigr)ds \mid  X^\pi(t)=x,v(t)=v \Biggr),
\end{align*}
where $\dagger$ follows from both that: 1) The Brownian motion process is a
Martingale implying it beloning to the class $\pounds^2$
and thus yielding conditional expectation zero by proposition 4.8
\cite{bjork2020} 2) The linearity of the expectation operator.

In equation (20), the left-hand side (LHS) delineates the value function under
an optimal investment strategy across the interval $[t, T]$. Conversely, the
right-hand side (RHS) delineates the value function when adopting any investment
strategy during the interval $[t, t + h)$, followed by a shift to an optimal
investment strategy for $[t + h, T]$. Thus, the essence of equation (20) is
that opting for an arbitrary investment strategy within the interval $[t, t+h)$,
before transitioning to an optimal strategy for $[t+h, T]$, is, on average, not
superior to consistently adhering to an optimal investment strategy throughout
the entire interval $[t, T]$.

$\textcolor{blue}{\star}$ As stated we want to divide by $h$, let $h\downarrow 0$. For
readability, define the integrand from the previous "$\textcolor{blue}{\star}$" as:
\begin{align*}
    M(s):=\mathcal{V}_t + \mathcal{V}_{x}  X^\pi(s)\left( r + \pi v(s) \bar{\lambda} \right) + \mathcal{V}_v \kappa \left( \theta - v(s) \right) + \frac{1}{2} \mathcal{V}_{xx}( X^\pi(s))^2 \pi^2 v(s) + \frac{1}{2} \mathcal{V}_{vv} \sigma^2 v(s) + \mathcal{V}_{xv} X^\pi(s) \pi v(s) \sigma \rho.
\end{align*}
In other words, rewrite the previous "$\textcolor{blue}{\star}$" inequality as:
\begin{align*}
    0\geq \EE \left ( \int_t^{t+h}M(s)\:ds \mid X^\pi(t)=x,v(t)=v \right ).
\end{align*}
Dividing now by $h$ yields:
\begin{align*}
    \frac{0}{h}&=0\\
    &\geq \frac{\EE \left ( \int_t^{t+h}M(s)\:ds \mid X^\pi(t)=x,v(t)=v \right )}{h}\\
    &\overset{\dagger}{=}\frac{\EE \left ( \int_0^{t+h}M(s)\:ds-\int_0^{t}M(s)\:ds \mid X^\pi(t)=x,v(t)=v \right )}{h}\\
    &\overset{\dagger\dagger}{=}\EE \left ( \frac{\int_0^{t+h}M(s)\:ds-\int_0^{t}M(s)\:ds}{h} \mid X^\pi(t)=x,v(t)=v \right ),
\end{align*}
where $\dagger$ follows from the additivity and linearity of the integral (as
$M(s)$ is a shorthand) and $\dagger \dagger$
from the linearity of the expectation operator.
Letting $h\downarrow 0$ now yields:
\begin{align*}
    \lim_{h\downarrow 0}0&=0\\
    &\geq \lim_{h\downarrow 0}\EE \left ( \frac{\int_0^{t+h}M(s)\:ds-\int_0^{t}M(s)\:ds}{h} \mid X^\pi(t)=x,v(t)=v \right )\\
    &\overset{\dagger}{=}\EE \left (\lim_{h\downarrow 0} \frac{\int_0^{t+h}M(s)\:ds-\int_0^{t}M(s)\:ds}{h} \mid X^\pi(t)=x,v(t)=v \right )\\
    &\overset{\dagger\dagger}{=}\EE \left (\lim_{h\downarrow 0} \frac{f(t+h)-f(t)}{h} \mid X^\pi(t)=x,v(t)=v \right )\\
    &=\EE \left ( f'(t) \mid X^\pi(t)=x,v(t)=v \right )\\
    &=\EE \left ( M(t) \mid X^\pi(t)=x,v(t)=v \right )\\
    &\overset{\dagger\dagger\dagger}{=}\mathcal{V}_t + \frac{1}{2}\sigma^2 \mathcal{V}_{vv}+\kappa(\theta-v)\mathcal{V}_v+ x\left(r + \pi\bar{\lambda} \right)V_x + \frac{1}{2}\pi^2x^2vV_{xx} + \pi x \sigma v \rho V_{xv},
\end{align*}
where $\dagger$ follows from Lebesgue's Dominated Convergence (LDC) as the
function $M(s)$ is bounded and
$\dagger\dagger$ (and forward) from the definition of a derivative and the
Fundamental Theorem of Calculus (FToC) with $f(t)=\int_0^t
M(s)\:ds$, implying $f'(t)=M(t)$. $\dagger\dagger\dagger$ is
simply from definition using the conditional expectation and unwinding the shorthand notation defined at the start.

$\textcolor{blue}{\star}$ Let $\pi^*$ represent the optimal investment strategy. It follows that $\pi^*$ is uniquely determined as the solution to the following optimization problem:
\begin{align*}
    g(\pi):=\sup_\pi \left\{ x\left(r + \pi\bar{\lambda}v\right)V_x + \frac{1}{2}\pi^2x^2vV_{xx} + \pi x \sigma v \rho V_{xv} \right\},
\end{align*}
which signifies that $\pi^*$ is the supremum of $\pi^*$ across all conceivable
or permissible investment strategies. Furthermore, since the previously found
inequality holds for any constant investment strategy $\pi$, it follows from the
previously found inequality that:
\begin{align*}
    0\geq \mathcal{V}_t + \frac{1}{2}\sigma^2 \mathcal{V}_{vv}+\kappa(\theta-v)\mathcal{V}_v+\sup_\pi \left\{ x\left(r + \pi\bar{\lambda}v\right)V_x + \frac{1}{2}\pi^2x^2vV_{xx} + \pi x \sigma v \rho V_{xv} \right\}.
\end{align*}
Given that $\pi^*$ is the optimal investment strategy, defined as the supremum
over $\pi$, it follows that for this $\pi^*$, equation (20) must be satisfied
with equality. Recall that in equation (20), the left-hand side (LHS) delineates the value function under
an optimal investment strategy across the interval $[t, T]$. Conversely, the
right-hand side (RHS) delineates the value function when adopting any investment
strategy during the interval $[t, t + h)$, followed by a shift to an optimal
investment strategy for $[t + h, T]$. Thus, the essence of equation (20) is
that opting for an arbitrary investment strategy within the interval $[t, t+h)$,
before transitioning to an optimal strategy for $[t+h, T]$, is, on average, not
superior to consistently adhering to an optimal investment strategy throughout
the entire interval $[t, T]$. Currently, we find ourselves in a scenario where
an optimal strategy is employed over the intervals $[t, t + h)$ and $[t + h,
T]$. This approach is tantamount to adhering to an optimal investment strategy
throughout the entire period $[t, T]$. Consequently, it is imperative that
equation (20) is satisfied with equality for $\pi^*$.

We adhere to the mercy of the court to let us re-use the just found dynamics the
the previous question for $\mathcal{V}$ (or the fact that Yevhen said it was
allowed at the Q\&A). Doing the exact same calculations
- other than the exact same deriviation for the dynamics - using the optimal investment strategy $\pi^*$ now yields:
\begin{align*}
    \mathcal{V}(t + h, X^{\pi^*}(t + h), v(t + h)) =\ &\mathcal{V}(t, X^{\pi^*}(t), v(t)) \\
    &+ \int_{t}^{t+h} \left( \mathcal{V}_t + \frac{1}{2} \sigma^2 v (s) \mathcal{V}_{vv} + \kappa \left( \theta - v(s) \right) \mathcal{V}_v \right. \\
    &\quad \left. + \sup_\pi \lc X^{\pi^*} \left( r + \pi \bar{\lambda} v(s) \right) \mathcal{V}_x + \frac{1}{2} \pi^2 (X^{\pi^*})^2 v(s) \mathcal{V}_{xx} + \pi X^{\pi^*} \sigma v(s) \rho \mathcal{V}_{xv} \rc \right) ds \\
    &+ \int_{t}^{t+h} \left( \sup_\pi \lc \mathcal{V}_x X^{\pi}(s) \pi \sqrt{v(s)} \rc + \mathcal{V}_v \sigma \rho \sqrt{v(s)} \right) dW^\PP_1(s) \\
    &+ \int_{t}^{t+h} \mathcal{V}_v \sigma \sqrt{v(s)} \sqrt{1 - \rho^2} dW^\PP_2(s).
\end{align*}
\newpage
As $\pi^*$ is the optimal investment strategy, as argued above, we now have that
equation (20) must hold with equality:
\begin{equation*}
    \hspace*{-1.7cm}
    \begin{aligned}
    \mathcal{V}(t, x, v) &= \EE \left( \mathcal{V}(t+h, X^{\pi^*}(t+h), v(t+h)) \mid X^{\pi^*}(t)=x,v(t)=v \right )\\
    &= \EE \Biggl( \mathcal{V}(t, X^{\pi^*}(t), v(t)) + \int_{t}^{t+h} \left( \mathcal{V}_t + \frac{1}{2} \sigma^2 v (s) \mathcal{V}_{vv} + \kappa \left( \theta - v(s) \right) \mathcal{V}_v \right. \\
    &\quad \quad \left. + \sup_\pi \lc X^{\pi^*}(s) \left( r + \pi \bar{\lambda} v(s) \right) \mathcal{V}_x + \frac{1}{2} \pi^2 (X^{\pi^*}(s))^2 v \mathcal{V}_{xx} + \pi X^{\pi^*}(s) \sigma v(s) \rho \mathcal{V}_{xv} \rc \right) ds \\
    &\quad + \int_{t}^{t+h} \left( \sup_\pi \lc \mathcal{V}_x X^{\pi^*}(s) \pi \sqrt{v(s)} \rc + \mathcal{V}_v \sigma \rho \sqrt{v(s)} \right) dW^\PP_1(s) \\
    &\quad + \int_{t}^{t+h} \mathcal{V}_v \sigma \sqrt{v(s)} \sqrt{1 - \rho^2} dW^\PP_2(s) \mid X^{\pi^*}(t)=x,v(t)=v  \Biggr) \\
    &\overset{\dagger}{=} \mathcal{V}(t, x, v) + \EE \Biggl( \int_{t}^{t+h} \left( \mathcal{V}_t + \frac{1}{2} \sigma^2 v (s) \mathcal{V}_{vv} + \kappa \left( \theta - v(s) \right) \mathcal{V}_v \right. \\
    &\quad \quad  \left. + \sup_\pi \lc X^{\pi^*}(s) \left( r + \pi \bar{\lambda} v(s) \right) \mathcal{V}_x + \frac{1}{2} \pi^2 (X^{\pi^*}(s))^2 v(s) \mathcal{V}_{xx} + \pi X^{\pi^*}(s) \sigma v(s) \rho \mathcal{V}_{xv} \rc \right) ds \mid X^{\pi^*}(t)=x,v(t)=v \Biggr)\\
    &\iff\\
    0 &= \mathcal{V}(t, x, v) + \EE \Biggl( \int_{t}^{t+h} \left( \mathcal{V}_t + \frac{1}{2} \sigma^2 v (s) \mathcal{V}_{vv} + \kappa \left( \theta - v(s) \right) \mathcal{V}_v \right. \\
    &\quad \quad  \left. + \sup_\pi \lc X^{\pi^*}(s) \left( r + \pi \bar{\lambda} v(s) \right) \mathcal{V}_x + \frac{1}{2} \pi^2 (X^{\pi^*})^2(s) v(s) \mathcal{V}_{xx} + \pi X^{\pi^*}(s) \sigma v(s) \rho \mathcal{V}_{xv} \rc \right) ds \mid X^{\pi^*}(t)=x,v(t)=v \Biggr) - \mathcal{V}(t, x, v) \\
    &=\EE \Biggl( \int_{t}^{t+h} \left( \mathcal{V}_t + \frac{1}{2} \sigma^2 v (s) \mathcal{V}_{vv} + \kappa \left( \theta - v(s) \right) \mathcal{V}_v \right. \\
    &\quad\quad  \left. + \sup_\pi \lc X^{\pi^*}(s) \left( r + \pi \bar{\lambda} v(s) \right) \mathcal{V}_x + \frac{1}{2} \pi^2 (X^{\pi^*})^2 v(s) \mathcal{V}_{xx} + \pi X^{\pi^*}(s) \sigma v(s) \rho \mathcal{V}_{xv} \rc \right) ds \mid X^{\pi^*}(t)=x,v(t)=v \Biggr),
    \end{aligned}
\end{equation*}
where $\dagger$ follows from both that: 1) The Brownian motion process is a
Martingale, beloning to the class $\pounds^2$,
and thus yielding conditional expectation zero by proposition 4.8 \cite{bjork2020}, 2) The linearity of the expectation operator.

In the same manner as the previous "$\textcolor{blue}{\star}$", define:
\begin{align*}
    J(s):=\mathcal{V}_t + \frac{1}{2} \sigma^2 v (s) \mathcal{V}_{vv} + \kappa \left( \theta - v(s) \right) \mathcal{V}_v + \sup_\pi \lc X^{\pi^*}(s) \left( r + \pi \bar{\lambda} v(s) \right) \mathcal{V}_x + \frac{1}{2} \pi^2 (X^{\pi^*}(s))^2 v(s) \mathcal{V}_{xx} + \pi X^{\pi^*}(s) \sigma v(s) \rho \mathcal{V}_{xv} \rc.
\end{align*}
In other words:
\begin{align*}
    0 = \EE \left ( \int_t^{t+h}J(s)\:ds \mid X^{\pi^*}(t)=x,v(t)=v \right ).
\end{align*}
Dividing now by $h$ yields:
\begin{align*}
    \frac{0}{h}&=0\\
    &= \frac{\EE \left ( \int_t^{t+h}J(s)\:ds \mid X^{\pi^*}(t)=x,v(t)=v \right )}{h}\\
    &\overset{\dagger}{=}\frac{\EE \left ( \int_0^{t+h}J(s)\:ds-\int_0^{t}J(s)\:ds \mid X^{\pi^*}(t)=x,v(t)=v \right )}{h}\\
    &\overset{\dagger\dagger}{=}\EE \left ( \frac{\int_0^{t+h}J(s)\:ds-\int_0^{t}J(s)\:ds}{h} \mid X^{\pi^*}(t)=x,v(t)=v \right ),
\end{align*}
where $\dagger$ follows from the additivity and linearity of the integral and $\dagger \dagger$
from the linearity of the expectation.
Letting $h\downarrow 0$ now yields:
\begin{align*}
    \lim_{h\downarrow 0}0&=0\\
    &= \lim_{h\downarrow 0}\EE \left ( \frac{\int_0^{t+h}J(s)\:ds-\int_0^{t}J(s)\:ds}{h} \mid X^{\pi^*}(t)=x,v(t)=v \right )\\
    &\overset{\dagger}{=}\EE \left (\lim_{h\downarrow 0} \frac{\int_0^{t+h}J(s)\:ds-\int_0^{t}J(s)\:ds}{h} \mid X^{\pi^*}(t)=x,v(t)=v \right )\\
    &\overset{\dagger\dagger}{=}\EE \left (\lim_{h\downarrow 0} \frac{f(t+h)-f(t)}{h} \mid X^{\pi^*}(t)=x,v(t)=v \right )\\
    &=\EE \left ( f'(t) \mid X^{\pi^*}(t)=x,v(t)=v \right )\\
    &=\EE \left ( J(t) \mid X^{\pi^*}(t)=x,v(t)=v \right )\\
    &\overset{\dagger\dagger\dagger}{=}\mathcal{V}_t + \frac{1}{2} \sigma^2 v \mathcal{V}_{vv} + \kappa \left( \theta - v \right) \mathcal{V}_v + \sup_\pi \lc x \left( r + \pi \bar{\lambda} v \right) \mathcal{V}_x + \frac{1}{2} \pi^2 x^2 v \mathcal{V}_{xx} + \pi x \sigma v \rho \mathcal{V}_{xv} \rc,
\end{align*}
where $\dagger$ follows from Lebesgue's Dominated Convergence (LDC) as the
function $J(s)$ is bounded and
$\dagger\dagger$ (and forward) from the definition of a derivative and the
Fundamental Theorem of Calculus (FToC) with $f(t)=\int_0^t
J(s)\:ds$, implying $f'(t)=J(t)$. $\dagger\dagger\dagger$ is
simply by definition using the conditional expectation and unwinding the shorthand notation defined at the start. The last equality
is exactly the Hamilton-Jacobi-Bellman partial differential equation HJB PDE,
equation (22)

\newpage
\subsection{Solution to the Hamilton-Jacobi-Bellman PDE}
$\textcolor{blue}{\star}$ The definition of the function $g(\pi)$ is seen in the
underbrace of equation (22). We find the FOC of $g$:
\begin{align*}
    g'(\pi) = x\bar{\lambda} v \mathcal{V}_x + \frac{1}{2} 2\pi x^2 v \mathcal{V}_{xx} + x\sigma v\rho \mathcal{V}_{xv} &= 0 \\
    &\overset{\dagger}{\iff}\\
     \bar{\lambda} \mathcal{V}_x + \pi x \mathcal{V}_{xx} + \sigma\rho \mathcal{V}_{xv} &= 0 \\
    &\iff\\
      -\bar{\lambda} \mathcal{V}_x - \sigma\rho \mathcal{V}_{xv}&=\mathcal{V}_{xx}\pi x  \\
     &\iff\\
     \pi&=-\frac{\bar{\lambda} \mathcal{V}_x}{x \mathcal{V}_{xx}} - \frac{\sigma\rho \mathcal{V}_{xv}}{x \mathcal{V}_{xx}} \\
     &\overset{\dagger\dagger}{=}-\frac{\bar{\lambda} \mathcal{V}_x(t, x, v)}{x \mathcal{V}_{xx}(t, x, v)} - \frac{\sigma\rho \mathcal{V}_{xv}(t, x, v)}{x \mathcal{V}_{xx}(t, x, v)} 
    \\&=: \pi^*(t, x, v) \\&\overset{\dagger\dagger\dagger}{>} 0,
\end{align*}
where $\dagger$ follows from dividing through by $xv$, $\dagger\dagger$ was simply undwinding the shorthand notation and
$\dagger\dagger\dagger$ from the assumption  $\mathcal{V}_{xx}<0$. Lastly, we have:
\begin{align*}
    g''(\pi)&=x\mathcal{V}_{xx}\\
    &\overset{\dagger}{<}0.
\end{align*}
where $\dagger$ follows from $\mathcal{V}_{xx}<0$ and $x>0$. It is thus shown that equation (23), $\pi^*$, is a unique
maximizer to $g(\pi)$.

\newpage
$\textcolor{blue}{\star}$ By substitution of equation (23) into equation (22)
with trial by fire and algebra now yields the PDE in $\mathcal{V}$ with terminal
condition $\mathcal{V}(T,x,v)=U(x)$:
\begin{equation*}
    \hspace*{-1.3cm}
    \begin{aligned}
        0 &= \mathcal{V}_t + \frac{1}{2} \sigma^2 v \mathcal{V}_{vv} + \kappa (\theta - v) \mathcal{V}_v + \sup_\pi \lc x \left ( r + \pi \bar{\lambda}v \right) \mathcal{V}_x + \frac{1}{2} \pi^2 x^2 v \mathcal{V}_{xx} + \pi x \sigma v \rho \mathcal{V}_{xv} \rc \\
        &= \mathcal{V}_t + \frac{1}{2} \sigma^2 v \mathcal{V}_{vv} + \kappa (\theta - v) \mathcal{V}_v + x \left ( r + \pi^* (t, x, v) \bar{\lambda} v\right) \mathcal{V}_x + \frac{1}{2} \left( \pi^* (t, x, v) \right)^2 x^2 v \mathcal{V}_{xx} + \pi^* (t, x, v) x \sigma v \rho \mathcal{V}_{xv} \\
        &= \mathcal{V}_t + \frac{1}{2} \sigma^2 v \mathcal{V}_{vv} + \kappa \theta \mathcal{V}_v - \kappa v \mathcal{V}_v + x r \mathcal{V}_x + \pi^* (t, x, v) x\bar{\lambda}v \mathcal{V}_x + \frac{1}{2} \left( \pi^* (t, x, v) \right)^2 x^2 v \mathcal{V}_{xx} + \pi^* (t, x, v) x \sigma v \rho \mathcal{V}_{xv} \\
        &= \mathcal{V}_t + \kappa \theta \mathcal{V}_v + x r \mathcal{V}_x + v \left( \frac{1}{2} \sigma^2 \mathcal{V}_{vv} - \kappa \mathcal{V}_v + \pi^* (t, x, v) \left( x \bar{\lambda} \mathcal{V}_x + \frac{1}{2} \pi^* (t, x, v) x^2 \mathcal{V}_{xx} + x \sigma \rho \mathcal{V}_{xv} \right) \right) \\
        &= \mathcal{V}_t + \kappa \theta \mathcal{V}_v + x r \mathcal{V}_x + v \left( \frac{1}{2} \sigma^2 \mathcal{V}_{vv} - \kappa \mathcal{V}_v + \left( -\bar{\lambda} \frac{\mathcal{V}_x}{x \mathcal{V}_{xx}} - \sigma \rho \frac{\mathcal{V}_{xv}}{x \mathcal{V}_{xx}} \right) \left ( x \bar{\lambda} \mathcal{V}_x + \frac{1}{2} \left( -\bar{\lambda} \frac{\mathcal{V}_x}{x \mathcal{V}_{xx}} - \sigma \rho \frac{\mathcal{V}_{xv}}{x \mathcal{V}_{xx}} \right) x^2 \mathcal{V}_{xx} + x \sigma \rho \mathcal{V}_{xv} \right) \right )\\
        &=\mathcal{V}_t + \kappa \theta \mathcal{V}_v + x r \mathcal{V}_x + v \left( \frac{1}{2} \sigma^2 \mathcal{V}_{vv} - \kappa \mathcal{V}_v + \left( -\bar{\lambda} \frac{\mathcal{V}_x}{x \mathcal{V}_{xx}} - \sigma \rho \frac{\mathcal{V}_{xv}}{x \mathcal{V}_{xx}} \right) \left ( x \bar{\lambda} \mathcal{V}_x - \frac{1}{2} \bar{\lambda}\mathcal{V}_xx - \frac{1}{2} x\sigma \rho  \mathcal{V}_{xv} + x \sigma \rho \mathcal{V}_{xv} \right) \right )\\
        &= \mathcal{V}_t + \kappa \theta \mathcal{V}_v + x r \mathcal{V}_x + v \left( \frac{1}{2} \sigma^2 \mathcal{V}_{vv} - \kappa \mathcal{V}_v + \frac{1}{\mathcal{V}_{xx}} \left( -\bar{\lambda} \mathcal{V}_x \frac{1}{x} - \sigma \rho \frac{1}{x} \mathcal{V}_{xv} \right) \left ( \frac{1}{2} \bar{\lambda} \mathcal{V}_{x}x + \frac{1}{2} x \sigma \rho \mathcal{V}_{xv} \right) \right) \\
        &= \mathcal{V}_t + \kappa \theta \mathcal{V}_v + x r \mathcal{V}_x + v \left( \frac{1}{2} \sigma^2 \mathcal{V}_{vv} - \kappa \mathcal{V}_v + \frac{1}{\mathcal{V}_{xx}} \left( - \frac{1}{2} \left( \bar{\lambda} \mathcal{V}_x \right)^2 - \frac{1}{2} \sigma \rho \bar{\lambda} \mathcal{V}_x \mathcal{V}_{xv} - \frac{1}{2} \bar{\lambda} \sigma \rho \mathcal{V}_x \mathcal{V}_{xv} - \frac{1}{2} \left( \sigma \rho \mathcal{V}_{xv} \right)^2 \right) \right) \\
        &= \mathcal{V}_t + \kappa \theta \mathcal{V}_v + x r \mathcal{V}_x + v \left( \frac{1}{2} \sigma^2 \mathcal{V}_{vv} - \kappa \mathcal{V}_v + \frac{1}{\mathcal{V}_{xx}} \left( - \frac{1}{2} \left( \bar{\lambda} \mathcal{V}_x \right)^2 - \sigma \rho \bar{\lambda} \mathcal{V}_x \mathcal{V}_{xv} - \frac{1}{2} \left( \sigma \rho \mathcal{V}_{xv} \right)^2 \right) \right) \\
        &= \mathcal{V}_t + \kappa \theta \mathcal{V}_v + x r \mathcal{V}_x + v \left( \frac{1}{2} \sigma^2 \mathcal{V}_{vv} - \kappa \mathcal{V}_v - \frac{1}{2} \left( \bar{\lambda} \mathcal{V}_x + \sigma \rho \mathcal{V}_{xv} \right)^2 \frac{1}{\mathcal{V}_{xx}} \right)\\
        &=\mathcal{V}_t + \kappa \theta \mathcal{V}_v + x r \mathcal{V}_x + v \left( \frac{1}{2} \sigma^2 \mathcal{V}_{vv} - \kappa \mathcal{V}_v - \frac{1}{2} \frac{\left( \bar{\lambda} \mathcal{V}_x + \sigma \rho \mathcal{V}_{xv} \right)^2}{\mathcal{V}_{xx}} \right),
        \end{aligned}
\end{equation*}
which is equation (24) as desired.

$\textcolor{blue}{\star}$ We recall the terminal condition is
$\mathcal{V}(T,x,v)=U(x)=x^p/p$. The seperation guessed solution (ansatz) for
the optimal solution is equation (25). Firstly, we find the respective
derivatives of $\mathcal{V}$ from equation (25):
\begin{align*}
    \mathcal{V}_x=ph\frac{x^{p-1}}{p},\quad \mathcal{V}_{xx}=(p-1)x^{p-2}h, \quad \mathcal{V}_{xv}=x^{p-1}h_v
\end{align*}
Substituting in the found derivatives into equation (23) now yields that the
optimal investment strategy is given by:
\begin{align*}
    \pi^*(t,x,v)&=-\frac{\bar{\lambda} \mathcal{V}_x}{x \mathcal{V}_{xx}} - \frac{\sigma\rho \mathcal{V}_{xv}}{x \mathcal{V}_{xx}}\\
    &=-\frac{\bar{\lambda} x^{p-1}h}{x(p-1)x^{p-2}h}-\sigma\rho\frac{x^{p-1}h_v}{x(p-1)x^{p-2}h}\\
    &=-\frac{\bar{\lambda} }{p-1}-\frac{\sigma\rho}{p-1}\frac{h_v}{h}\\
    &=\frac{\bar{\lambda} }{1-p}+\frac{\sigma\rho}{1-p}\frac{h_v}{h},
\end{align*}
which is exactly equation (26).

\newpage
$\textcolor{blue}{\star}$ Firstly, we find the respective
derivatives of $\mathcal{V}$ from equation (25):
\begin{align*}
    \mathcal{V}_t &= \frac{x^p}{p} h_t, & \mathcal{V}_x &=ph\frac{x^{p-1}}{p} = x^{p-1}h, & \mathcal{V}_{xx} &= (p - 1)x^{p-2}h, \\
    \mathcal{V}_{xv} &= x^{p-1}h_v, & \mathcal{V}_v &= \frac{x^p}{p} h_v, & \mathcal{V}_{vv} &= \frac{x^p}{p} h_{vv}.
    \end{align*}
Substituting the found derivatives into equation (24) yields the PDE for $h$:
\begin{align*}
    0 &= \mathcal{V}_t + \kappa\theta\mathcal{V}_v + xr\mathcal{V}_x + v \left( \frac{1}{2} \sigma^2 \mathcal{V}_{vv} - \kappa \mathcal{V}_v - \frac{1}{2}\frac{ \left( \bar{\lambda}\mathcal{V}_x + \sigma\rho\mathcal{V}_{xv} \right)^2 }{\mathcal{V}_{xx}} \right) \\
    &= \frac{x^p}{p} h_t + \kappa\theta \frac{x^p}{p} h_v + xrx^{p-1}h + v \left( \frac{1}{2} \sigma^2 \frac{x^p}{p} h_{vv} - \kappa \frac{x^p}{p} h_v - \frac{1}{2} \frac{\left ( \bar{\lambda}x^{p-1}h+\sigma\rho x^{p-1}h_v\right )^2}{(p-1)x^{p-2}h}\right) \\
    &= \frac{x^p}{p} h_t + \kappa\theta \frac{x^p}{p} h_v + rx^p h + v \left( \frac{1}{2} \sigma^2 \frac{x^p}{p} h_{vv} - \kappa \frac{x^p}{p} h_v - \frac{1}{2} \frac{x^{2p-2}\left (\bar{\lambda} h+\sigma\rho h_v\right )^2}{(p-1)x^{p-2}h} \right) \\
    &= \frac{x^p}{p} h_t + \kappa\theta \frac{x^p}{p} h_v + rx^p h + v \left( \frac{1}{2} \sigma^2 \frac{x^p}{p} h_{vv} - \kappa \frac{x^p}{p} h_v - \frac{1}{2} \frac{x^p\left (\bar{\lambda} h+\sigma\rho h_v\right )^2}{(p-1)h} \right) \\
    &= \frac{x^p}{p} h_t + \kappa\theta \frac{x^p}{p} h_v + rx^p h + v \left( \frac{1}{2} \sigma^2 \frac{x^p}{p} h_{vv} - \kappa \frac{x^p}{p} h_v + \frac{1}{2} \frac{x^p\left (\bar{\lambda} h+\sigma\rho h_v\right )^2}{(1-p)h} \right)  \\
    &= \frac{x^p}{p} \left( h_t + \kappa\theta h_v + prh + v \left( \frac{1}{2} \sigma^2 h_{vv} - \kappa h_v + \frac{1}{2}\frac{p\left(h\bar{\lambda}+\rho \sigma h_v \right)^2}{(1-p)h}\right) \right) \\
    &\iff\\
    0 &= h_t + \kappa\theta h_v + prh + v \left( \frac{1}{2} \sigma^2 h_{vv} - \kappa h_v + \frac{1}{2} \frac{p\left(h\bar{\lambda}+\rho \sigma h_v \right)^2}{(1-p)h}\right),
\end{align*}
which is exactly equation (27).

\newpage
$\textcolor{blue}{\star}$ Firstly, we find the derivatives of $h$ given in
equation (28):
\begin{align*}
    h_t &= \left(a'(\tau) \cdot (-1) + vb'(\tau) \cdot (-1)\right)h = -a'(\tau)h - b'(\tau)h_v, \\
    h_v &= b(\tau)h,\quad h_{vv} = b^2(\tau)h.
\end{align*}
Substituting the found derivatives into equation (27), rearranging the terms to
emphasize the linearity in $v$ as asked to, yields:
\begin{align*}
    0 &= h_t + \kappa\theta h_v + prh + v \left( \frac{1}{2} \sigma^2 h_{vv} - \kappa h_v + \frac{1}{2}  \frac{p\left( \bar{\lambda}h + \sigma\rho h_v \right)^2}{(1-p)h} \right) \\
    &= h_t + \kappa\theta h_v + prh + v \left( \frac{1}{2} \sigma^2 h_{vv} - \kappa h_v + \frac{p\bar{\lambda}^2h^2}{2(1 - p)h} + \frac{p\sigma^2\rho^2h_v^2}{2(1 - p)h} + \frac{2p\bar{\lambda}\sigma\rho h h_v}{2(1 - p)h} \right) \\
    &= h_t + \kappa\theta h_v + prh + v \left( \frac{1}{2} \sigma^2 h_{vv} - \kappa h_v + \frac{p\bar{\lambda}^2 h}{2(1 - p)} + \frac{p\sigma^2\rho^2h_v^2}{2(1 - p)h} + \frac{p \bar{\lambda} \sigma\rho  h_v}{1 - p} \right) \\
    &= -a'(\tau)h - b'(\tau)h_v + \kappa\theta b(\tau)h + prh + v \left( \frac{1}{2} \sigma^2 b^2(\tau)h - \kappa b(\tau)h + \frac{p\bar{\lambda}^2h}{2(1 - p)}  + \frac{p\sigma^2\rho^2 b^2(\tau)h^2}{2(1 - p)h} + \frac{p\bar{\lambda}\sigma\rho b(\tau)h}{1 - p} \right) \\
    &= -a'(\tau)h + b(\tau)\kappa\theta h + prh + v \left( -b'(\tau)h + b^2(\tau) \left( \frac{1}{2}\sigma^2h + \frac{p\sigma^2\rho^2h}{2(1 - p)} \right)  + b(\tau) \left( -\kappa h + \frac{p \bar{\lambda}\sigma\rho h}{1 - p} \right) + \frac{p \bar{\lambda}^2 h}{2(1 - p)} \right),
    \end{align*}
exactly as we wanted to show.

\newpage
$\textcolor{blue}{\star}$ The previously derived Partial Differential Equation
(PDE) is satisfied if and only if both conditions $(***)$ and $(***\:*)$ are
equal to zero. This condition is necessary for the PDE to be valid for all $\tau
> 0$. As a result, these conditions lead to a set of Ordinary Differential
Equations (ODEs) derived from $(***)$ and $(***\:*)$, which can be tackled
independently (similar to our approach in Subsection 1.2). Eliminating the terms involving $h$ and reorganizing the PDE, we
deduce two separate ordinary differential equations concerning the functions $a$
and $b$:
\begin{align*}
    0 &= -a'(\tau)h + b(\tau)\kappa\theta h + prh \\
    &\quad+ v\left( -b'(\tau)h + b^2(\tau) \left( \frac{1}{2} \sigma^2 h + \frac{p\sigma^2\rho^2 h^2}{2(1 - p)} \right) \right. \left. + b(\tau) \left( -\kappa h + \frac{p\bar{\lambda}\sigma\rho h}{1 - p} \right) + \frac{p\bar{\lambda}^2h}{2(1 - p)} \right) \\
    &\iff\\
     0 &= h\left(\underbrace{-a'(\tau) + b(\tau)\kappa\theta + pr}_{(***)}\right) \\
     &\quad + vh \bigg( \underbrace{-b'(\tau) + b^2(\tau) \left( \frac{1}{2} \sigma^2 + \frac{p\sigma^2\rho^2}{2(1 - p)} \right)+ b(\tau) \left( -\kappa + \frac{p\bar{\lambda}\sigma\rho}{1 - p} \right) + \frac{p\bar{\lambda}^2}{2(1 - p)}}_{(****)} \bigg)\\
    &\iff\\
    &\begin{cases}
    \underbrace{- a'(\tau) + b(\tau)\kappa\theta + pr}_{(***)} = 0, &  \\
    \underbrace{- b'(\tau) + b^2(\tau) \left( \frac{1}{2} \sigma^2 + \frac{p\sigma^2\rho^2}{2(1-p)} \right) + b(\tau) \left( -\kappa + \frac{p\bar{\lambda}\sigma\rho }{1-p} \right) + \frac{p\bar{\lambda}^2}{2(1-p)}}_{(****)} = 0 &
    \end{cases} \\
    &\overset{\dagger}{\iff}\\
    &\begin{cases}
    a'(\tau) = \kappa\theta b(\tau) + pr, \\
    b'(\tau) = \frac{1}{2} \underbrace{\left( \sigma^2 + \frac{p\sigma^2\rho^2}{1 - p} \right)}_{=:k_2}b^2(\tau) - \underbrace{\left( \kappa - \frac{p\bar{\lambda}\sigma\rho }{1 - p} \right)}_{=:k_1} b(\tau)+\frac{1}{2}\underbrace{\frac{p\bar{\lambda}^2}{1-p}}_{=:k_0}
    \end{cases} \\
    &\iff\\
    &\begin{cases}
    a'(\tau) = \kappa\theta b(\tau) + pr, \\
    b'(\tau) = \frac{1}{2} k_2 b^2(\tau) - k_1 b(\tau) + \frac{1}{2} k_0,
    \end{cases},
\end{align*}
where the underbraces in $\dagger$ follow from the Handin \#3 description. Furthermore, we are told that the boundary condition for $h$ is exactly given by
$h(T,z)=1$, $\forall z>0$, which yields:
\begin{align*}
    h(T,z)&=\exp{\lc a(\tau(T))+b(\tau(T))z\rc}\\
    &=1\\
    &\iff\\
    a(0)=a(\tau(T))=0 \quad     &\: \: \text{and} \quad b(0)=b(\tau(T))=0,
\end{align*}
which defines the sufficient boundary conditions for the functions $a,\:b$.
\newpage
$\textcolor{blue}{\star}$  By equation (28) we can rewrite to:
\begin{align*}
    \mathcal{V}_{xx}=(p-1)x^{p-2}\exp{\lc a(\tau)+b(\tau)v \rc}
\end{align*}
Using the definitions for $k_i$ (and not $\kappa$), $i\in \{0,1,2\}$ that are underbraced in the
first equality in equation (30) now yields:
\begin{align*}
    k^2 (1 - k_0k_2) &= \left( \kappa - \frac{p\bar{\lambda}\sigma\rho h}{1 - p} \right)^2 - \left( \frac{p\bar{\lambda}^2}{1 - p} \right) \left( \sigma^2 + \frac{p\sigma^2\rho^2}{1 - p} \right) \\
    &= \kappa^2 + \frac{p^2\bar{\lambda}^2\sigma^2\rho^2}{(1 - p)^2} - 2\kappa \frac{p\bar{\lambda}\sigma\rho}{1 - p} - \frac{p\bar{\lambda}^2}{1 - p} \sigma^2 - \frac{p^2\bar{\lambda}^2\sigma^2\rho^2}{(1 - p)^2} \\
    &= \kappa^2 - 2\kappa \frac{p \bar{\lambda}\sigma\rho }{1 - p} - \frac{p\bar{\lambda}^2}{1 - p} \sigma^2 \\
    &= \kappa^2 -  \frac{2\kappa p \bar{\lambda}\sigma\rho-p\bar{\lambda}^2}{1 - p} \sigma^2 \\
    &=\frac{\kappa^2 (1 - p) - 2\kappa p \bar{\lambda}\sigma\rho h - p\bar{\lambda}^2\sigma^2 }{1 - p} \\
    &> 0 \\
    &\iff\\
    1 - p&> 0 \\
    &\Rightarrow\\
    \mathcal{V}_{xx}&=(p-1)x^{p-2}\exp{\lc a(\tau)+b(\tau)v \rc}\\
    & < 0,
\end{align*}
where we used the fact that $p-1<0$ and then also $1-p>0$ remembering $p\in
(-\infty,0) \cup (0,1)$. Using now the previously found derivative, $h_v=b(\tau)h$,
of $h$ means we can write equation (26) as:
\begin{align*}
    \pi^*(t,x,v)&=\frac{\bar{\lambda}}{1-p}+\frac{\sigma \rho}{1-p}\frac{h_v}{h}\\
    &=\frac{\bar{\lambda}}{1-p}+\frac{\sigma \rho}{1-p}\frac{b(\tau)h}{h}\\
    &=\frac{\bar{\lambda}}{1-p}+\frac{\sigma \rho}{1-p}b(\tau),
\end{align*}
and thus conclude that the candidate for the optimal investment strategy
is indeed given by equation (33) with $b(t)$ given by equation (32) and
$\tau=T-t$ (such that $b(\tau(t)$), yielding:
\begin{align*}
    \pi^*(t,x,v)=\frac{\bar{\lambda}}{1-p}+\frac{\sigma \rho}{1-p}b(\tau(t)).
\end{align*}
\newpage
\subsection{Interpretation of the results}
$\textcolor{blue}{\star}$ All the code can be viewed in \autoref{chap:code}. 

To perform a sensitivity analysis on the optimal investment strategy $\pi^*$ in
relation to the model parameters, $\pi^*$ has been expressed as a function of
time $t\in [0,T]$, utilizing equations (33) and (32). As such, we define a
\texttt{optimalpi}-function defined as in the Handin \#3 description.  This representation enables the
generation of plots for $\pi^*(t)$ from $t=0$ to expiry-$T$.

Firstly, expiry is set to $T = 10$. We define $t$
to ensure that there is a time point for each business day within the year,
resulting in a vector $t = T - \tau$ that includes $252\cdot 10$ time points.

Initially, we will determine the values of the optimal investment strategy
$\pi^*$ as a function of time using selected parameter values. This strategy
will act as our benchmark. For the sensitivity analysis, we will examine the
impact of altering each parameter on $\pi^*$ over time and compare these effects
against the benchmark. This analysis will involve creating a separate plot for
each parameter, focusing on one parameter at a time.

The analysis in \texttt{Python}½ is done with the benchmark parameters seen in
Table 10 with the sensitivity analysis utilizing varying degrees of the
benchmark. When a varying parameter is scrutinize all others are held constant
at their benchmark.

\begin{table}[!h]
    \centering
    \begin{tabular}{lll}
    \toprule
    \textbf{Description} & \textbf{Symbol} & \textbf{Value} \\
    \midrule
    Speed of Mean Reversion & $\kappa$ & 1 \\
    Volatility of the Volatility & $\sigma$ & 0.3 \\
    Correlation Coefficient & $\rho$ & -0.5 \\
    Risk-Aversion Parameter & $p$ & 0.5  \\
    Risk Premium per Unit of Variance & $\bar{\lambda}$ & 0.4 \\
    Expiry Time & $T$ & 10 \\
    \bottomrule
    \end{tabular}
    \caption{Model parameters used for the sensitivity analysis of $\pi^*$.}
    \label{table:table10}
\end{table}
\newpage
\begin{figure}[!h]
    \centering
    \hspace*{-2cm}
    \includegraphics[width=0.95\linewidth]{sigma.png}
    \label{Figure 2}
    \setlength{\abovecaptionskip}{-5pt}
    \caption{The optimal investing strategy for varying $\sigma$.}
\end{figure}

In Figure 2, the benchmark investment strategy is illustrated, where $\sigma$
has been set to $0.3$. Additionally, the Volatility of the Volatility
(Vol-of-Vol) parameter $\sigma$ is
varied, and the optimal investment strategies for $\sigma \in \{0.05,\: 0.5,\: 0.8\}$
are computed, with all other parameters held constant.

At the expiry, all values of $\sigma$ converge to an optimal investment strategy value of
$0.8$, indicating uniformity at this point. The graph shows that for the
benchmark case $0.3$, the optimal investment strategy increases gradually over time
for the given parameters. As $\sigma$ increases, the initial value of the
optimal strategy decreases, which aligns with intuitive financial principles;
higher $\sigma$ typically leads to a more conservative investment stance as
uncertainty arises at investment-time. The
end-state uniformity across all values of $\sigma$ implies that higher initial
values result in a steeper curve as the strategy approaches the expiry,
reflecting a more aggressive investment strategy over time. Conversely, when
$\sigma$ is very low, it becomes sensible to increase the optimal investments at
$t=0$ due to reduced
uncertainty, suggesting that the optimal investment strategy remains quite
stable in this scenario. This is entirely reasonable since $\sigma$ represents
the vol-of-vol, essentially quantifying the variability of
the variance. As expected, $\pi^*$ will fluctuate more with an increase in the
vol-of-vol, where uncertainty is higher, and will be more
consistent under conditions of low volatility. 

Therefore, it is evident that
$\sigma$ does not influence the optimal investment level at expiry, but affects
it in earlier periods untill nearing expiry-$T$.
\newpage
\begin{figure}[!h]
    \centering
    \hspace*{-2cm}
    \includegraphics[width=0.95\linewidth]{rho.png}
    \label{Figure 3}
    \setlength{\abovecaptionskip}{-5pt}
    \caption{The optimal investing strategy for varying $\rho$.}
\end{figure}

In Figure 3, the benchmark investment strategy is illustrated, where $\rho$ has
been set to $-0.5$. Additionally, the Correlation Coefficient $\rho$ is varied, and the
optimal investment strategies for $\rho \in \{-0.05,\: -0.2,\: -0.8\}$ are computed,
with all other parameters held constant. Note that from empirical observations of
the financial markets $\rho<0$.

As previously noted, we observe that at expiry, the optimal investment strategy
$\pi^*$ consistently attains a value of $0.8$, independent of the chosen $\rho$.
The variable $\rho$ characterizes the correlation between two Brownian motions.
Analysis of the plot indicates that a small numerical value of $\rho$
leads to a higher initial investment.
Furthermore, the value of the optimal investment strategy remains relatively
stable over time. Conversely, when $\rho$ is numerically larger, the enhanced negative correlation between the
Brownian motions necessitates a lower initial optimal investment strategy. This
scenario results in the optimal investment strategy gradually increasing over
time, culminating uniformly at $0.8$ by the expiry.  This behavior is a
consequence of the model dynamics.

The explanation hinges on understanding the relationship between the correlation
coefficient, $\rho$, and the contributions of the two Brownian motions, $W_1$
and $W_2^\PP$, to the dynamics of the asset price $S_1$ and its variance $v$.
If $\rho$ is numerically large, the contribution from $W_2^\PP$ becomes negligible
for movements in the variance, and thus it is only $W_1^\PP$ that drives both $dv$ and
$dS_1$. This means that the optimal investment strategy is Initially low as
there is a strong dependency between movements in both directions of the asset
price and the variance, leading to a more cautious approach.

Conversely, if $\rho$ is approximately zero, the contribution $W_1^\PP$ to the
variance is negligible and the variance is driven almost solely by $W_2^\PP$, but the
price is still driven by $W_1^\PP$. A lower correlation $\rho \approx 0$ means
that the relationship between
$S_1$ and $v$ is almost independent which allows for more aggressive investment strategies due to more
predictable risk management. Furthermore, the lack of uncertainty allows for
only minor corrections in investment near expiry. Conversely, a higher correlation ($\rho$ numerically large)
necessitates a more conservative approach due to increased perceived risk and uncertainty
in the asset’s price movements relative to its variance because of the near
independent relationsship between the risky asset $S_1$ and $v$.

In conclusion, $\rho$ does not influence the ultimate level of the optimal
investment strategy at expiry, but is affecting only the strategy in the earlier
years.

\begin{figure}[!h]
    \centering
    \hspace*{-2cm}
    \includegraphics[width=0.95\linewidth]{kappa.png}
    \label{Figure 4}
    \setlength{\abovecaptionskip}{-5pt}
    \caption{The optimal investing strategy for varying $\kappa$.}
\end{figure}

In Figure 4, the benchmark iznvestment strategy is illustrated, where $\kappa$ has
been set to $1$. Additionally, the Mean Reversaion Rate $\kappa$ is varied, and the
optimal investment strategies for $\kappa \in \{0.5,\: 2, \:3\}$ are computed,
with all other parameters held constant.

It is clear that at expiration, the optimal investment strategy $\pi^*$ is
consistently 0.8, unaffected by the value of $\kappa$. The parameter $\kappa$
denotes the mean reversion rate, reflecting how swiftly the variance of the
asset returns to its long-term average, $\theta$. According to the curves in
Figure 4, a
higher $\kappa$ corresponds to an increased initial investment strategy. This
makes sense because quicker mean reversion suggests a faster stabilization to
the equilibrium state, allowing investors to predict future variances more
confidently and adjust their portfolios aggressively in the short term with a
higher optimal investment strategy.

On the other hand, a lower $\kappa$ implies a longer duration for the variance to return to the mean, resulting in a more unpredictable variance process and thereby elevating the risk as uncertainty is more predominant until expiry. Consequently, a more conservative initial investment strategy is justified under these circumstances. Such conservative behavior mitigates the risk of exposure to high variance, thus safeguarding the investment against the potential for large fluctuations that could lead to substantial losses.

Therefore, while $\kappa$ has no impact on the optimal investment level at expiration, it significantly influences the investment strategy during the initial years.

\newpage
\begin{figure}[!h]
    \centering
    \hspace*{-2cm}
    \includegraphics[width=0.95\linewidth]{lambda.png}
    \label{Figure 5}
    \setlength{\abovecaptionskip}{-5pt}
    \caption{The optimal investing strategy for varying $\bar{\lambda}$.}
\end{figure}

In Figure 5, the benchmark investment strategy is illustrated, where $\bar{\lambda}$ has
been set to $0.4$. Additionally, the Risk Premium per Unit of Variance $\bar{\lambda}$ is varied, and the
optimal investment strategies for $\bar{\lambda} \in \{0.1,\: 0.9, \:3\}$ are computed,
with all other parameters held constant. Note that from empirical observations of
the financial markets $\bar{\lambda}>0$.

Contrary to the earlier three plots, we now see that $\bar{\lambda}$
significantly influences the optimal investment strategy at expiry.
$\bar{\lambda}$ serves as the indicator for volatility risk premium, providing
insights into traders' risk tolerance and their decision-making aggressiveness.
It is observed that a high $\bar{\lambda}$ correlates with an equally high
optimal investment strategy. Essentially, a high $\bar{\lambda}$ suggests that
the market compensates more for risk-taking, which in turn implies that
investors are likely to invest more heavily and accept greater risks. According
to \cite{kraft2005optimal}, there must be an upper limit to $\bar{\lambda}$;
exceeding this limit would lead to excessively high market rewards for risk,
rendering equations (31) and (32) invalid solutions for our model. This scenario leads to
traders being overly optimistic about future gains, which logically results in
observing elevated levels of optimal investment under high $\bar{\lambda}$.
Additionally, a lower $\bar{\lambda}$ tends to stabilize the optimal investment
strategy over time, unlike when $\bar{\lambda} = 3$, where the initial
investment is significantly high and continues to rise at expiry, suggesting an
overly generous market reward for risk.

Thus, while a higher $\bar{\lambda}$ generally encourages more aggressive
investment strategies, reflecting a direct response to the higher rewards for
risk-taking offered by the market, it is crucial to recognize the limits of this
parameter. Adhering to realistic and sustainable investment practices requires
careful calibration of $\bar{\lambda}$ within the models to ensure that the
investment strategies formulated are both optimal and practical, aligning with
genuine market dynamics and risk tolerance levels.
\newpage
\begin{figure}[!h]
    \centering
    \hspace*{-2cm}
    \includegraphics[width=0.95\linewidth]{p.png}
    \label{Figure 6}
    \setlength{\abovecaptionskip}{-5pt}
    \caption{The optimal investing strategy for varying $p$.}
\end{figure}

In Figure 6, the benchmark investment strategy is illustrated, where $p$ has
been set to $0.5$. Additionally, the Risk-Aversion Parameter $p$ is varied, and the
optimal investment strategies for $p \in \{-5,\:-2,\:0.1,\: 0.8, \:0.9\}$ are computed,
with all other parameters held constant. Note $p \in (−\infty, 0) \cup (0, 1)$ by assumption.

Similar to $\bar{\lambda}$, we now find that $p$ indeed
impacts the optimal investment strategy at expiry and, consequently, over the
long term. The plot reveals that when $p \in (-\infty, 0)$, the optimal
investment level is quite low and remains consistently stable - or rather constant. Conversely, for
$p \in (0, 1)$, the optimal investment strategy starts off low when $p$ is near
0, but is still higher than when $p$ is negative. As $p$ approaches 1, the
initial optimal investment strategy begins to increase. Notably, as $p$ nears
its upper limit ($p \approx 1$), the initial optimal investment strategy
substantially increases, escalating further as we near expiry.

$p$ represents the risk-aversion coefficient, indicating a trader's willingness to accept risk. When $p \in (0, 1)$, it characterizes a trader who is cautious yet open to taking risks, moving towards risk-neutrality as $p$ nears 1. The plot confirms that with such traders, initial investments increase as perceived risks diminish or as potential returns appear more favorable. Specifically, we notice that at $p \approx 1$, optimal investment increases as expiry approaches. This correlates with decreased risk due to the shorter time to maturity, leading to more precise expectations. Additionally, this aligns well with the trader approaching risk-neutrality, allowing for slight variations in the optimal investment strategy over time, unlike the nearly constant strategy observed at lower $p$ values. When $p \in (-\infty, 0)$, traders exhibit an even greater aversion to risk. This reflects precisely what the plot indicates: such traders maintain low and stable initial investments over time. In summary, as anticipated, traders more willing to accept risk ($p \in (0, 1)$) demonstrate more aggressive optimal investment behaviors as expiry approaches, optimizing their expected returns in line with their risk preferences. Conversely, traders who avoid risk ($p \in (-\infty, 0)$) exhibit low and stable optimal investment strategies.

\newpage
\begin{figure}[!h]
    \centering
    \hspace*{-2cm}
    \includegraphics[width=0.95\linewidth]{T.jpg}
    \label{Figure 7}
    \setlength{\abovecaptionskip}{-5pt}
    \caption{The optimal investing strategy for varying $T$.}
\end{figure}

In Figure 7, the benchmark investment strategy is illustrated, where $T$ has
been set to $10$. Additionally, the Expiry Time $T$ is varied, and the
optimal investment strategies for $T \in \{5, \:30,\: 100\}$ are computed,
with all other parameters held constant.

We see that changing the expiry time $T$ results in horizontal shifts for the
curves, meaning, it has no effect on the optimal investment strategy in the
interval $[0,T]$. This is as expected as - when all other parameters held
constant - as the investment decision process does not inherently depend on the
duration of the investment but rather on the underlying asset dynamics and the
parameters controlling the investment strategy itself. Therefore, the expiry
time $T$ simply extends or shortens the investment horizon without altering the
nature or direction of the investments within that period.
\newpage
\appendix
\section{Appendix: Code}\label{chap:code}
\href{https://github.com/YoussefRaad-mathecon/Handin-3}{GitHub profile
with the code in \texttt{Python} for \textit{Option Pricing in the Context of the Heston
Model} and \textit{Asset Allocation Under the Heston Model}. We frequently refer
to and use the parameter $"\lambda"$. It's crucial to understand that this
$"\lambda"$ parameter should not be confused with \texttt{Python}'s
built-in \texttt{lambda}-function in subsection 2.3 and 2.4. \texttt{Python}'s \texttt{lambda} function
is a feature that allows for creating small, anonymous functions at runtime.
They are defined using the \texttt{lambda} keyword, followed by a list of
arguments etc.}

\newpage
\bibliographystyle{abbrvnat}

\bibliography{mybib.bib}
\end{document}
